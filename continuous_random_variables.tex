% Author Alfredo Sánchez Alberca (asalber@ceu.es)

\newproblem{vac-1}{gen}{}
%ENUNCIADO
{The probability density function of a continuous random variable $X$ is
\[
f(x)=
\begin{cases}
k(6-3x) & \mbox{if $0\leq x\leq 2$,} \\
0 & \mbox{if  $x<0$ or $x>2$.}
\end{cases}
\]
\begin{enumerate}
\item Compute the value for $k$.
\item Compute $P(X\leq 1)$; $P(X>2)$; $P(X=1/4)$; $P(1/3\leq X\leq 2/3)$.
\item Compute the mean and the standard deviation.
\item Compute the distribution function $F(x)$.
\end{enumerate}
}
%SOLUCIÓN
{
\begin{enumerate}
\item $k=1/6$.
\item $P(X\leq 1)=0.75$, $P(X>2)=0$, $P(X=1/4)=0$ and $P(1/3\leq X\leq 2/3)=1/4$.
\item $\mu=2/3$, $\sigma^2=2/9$ and $\sigma=\sqrt{2}/3$.
\item \[
F(x)=
\begin{cases}
0 & \text{if $x<0$,}\\
x-\frac{x^2}{4} & \mbox{if $0\leq x\leq 2$,}\\
1 & \text{if $2<x$.}
\end{cases}
\]
\end{enumerate}
}
%RESOLUCIÓN
{}


\newproblem{vac-2}{gen}{*}
%ENUNCIADO
{La función de densidad de la variable aleatoria continua $X$ viene dada por la gráfica siguiente:
\begin{center}
\scalebox{0.6}{% Created by tikzDevice version 0.10.1 on 2016-05-10 11:29:27
% !TEX encoding = UTF-8 Unicode
\begin{tikzpicture}[x=1pt,y=1pt]
\definecolor{fillColor}{RGB}{255,255,255}
\path[use as bounding box,fill=fillColor,fill opacity=0.00] (0,0) rectangle (289.08,289.08);
\begin{scope}
\path[clip] ( 42.00, 42.00) rectangle (277.08,253.08);
\definecolor{drawColor}{RGB}{5,161,230}

\path[draw=drawColor,line width= 0.8pt,line join=round,line cap=round] ( 50.71, 42.00) --
	(159.54,233.89) --
	(268.37, 42.00);
\end{scope}
\begin{scope}
\path[clip] (  0.00,  0.00) rectangle (289.08,289.08);
\definecolor{drawColor}{RGB}{0,0,0}

\path[draw=drawColor,line width= 0.4pt,line join=round,line cap=round] ( 50.71, 42.00) -- (268.37, 42.00);

\path[draw=drawColor,line width= 0.4pt,line join=round,line cap=round] ( 50.71, 42.00) -- ( 50.71, 36.00);

\path[draw=drawColor,line width= 0.4pt,line join=round,line cap=round] (105.12, 42.00) -- (105.12, 36.00);

\path[draw=drawColor,line width= 0.4pt,line join=round,line cap=round] (159.54, 42.00) -- (159.54, 36.00);

\path[draw=drawColor,line width= 0.4pt,line join=round,line cap=round] (213.96, 42.00) -- (213.96, 36.00);

\path[draw=drawColor,line width= 0.4pt,line join=round,line cap=round] (268.37, 42.00) -- (268.37, 36.00);

\node[text=drawColor,anchor=base,inner sep=0pt, outer sep=0pt, scale=  1.00] at ( 50.71, 25.20) {0.0};

\node[text=drawColor,anchor=base,inner sep=0pt, outer sep=0pt, scale=  1.00] at (105.12, 25.20) {0.5};

\node[text=drawColor,anchor=base,inner sep=0pt, outer sep=0pt, scale=  1.00] at (159.54, 25.20) {1.0};

\node[text=drawColor,anchor=base,inner sep=0pt, outer sep=0pt, scale=  1.00] at (213.96, 25.20) {1.5};

\node[text=drawColor,anchor=base,inner sep=0pt, outer sep=0pt, scale=  1.00] at (268.37, 25.20) {2.0};

\path[draw=drawColor,line width= 0.4pt,line join=round,line cap=round] ( 42.00, 42.00) -- ( 42.00,233.89);

\path[draw=drawColor,line width= 0.4pt,line join=round,line cap=round] ( 42.00, 42.00) -- ( 36.00, 42.00);

\path[draw=drawColor,line width= 0.4pt,line join=round,line cap=round] ( 42.00, 80.38) -- ( 36.00, 80.38);

\path[draw=drawColor,line width= 0.4pt,line join=round,line cap=round] ( 42.00,118.76) -- ( 36.00,118.76);

\path[draw=drawColor,line width= 0.4pt,line join=round,line cap=round] ( 42.00,157.13) -- ( 36.00,157.13);

\path[draw=drawColor,line width= 0.4pt,line join=round,line cap=round] ( 42.00,195.51) -- ( 36.00,195.51);

\path[draw=drawColor,line width= 0.4pt,line join=round,line cap=round] ( 42.00,233.89) -- ( 36.00,233.89);

\node[text=drawColor,anchor=base east,inner sep=0pt, outer sep=0pt, scale=  1.00] at ( 34.80, 38.56) {0.0};

\node[text=drawColor,anchor=base east,inner sep=0pt, outer sep=0pt, scale=  1.00] at ( 34.80, 76.93) {0.2};

\node[text=drawColor,anchor=base east,inner sep=0pt, outer sep=0pt, scale=  1.00] at ( 34.80,115.31) {0.4};

\node[text=drawColor,anchor=base east,inner sep=0pt, outer sep=0pt, scale=  1.00] at ( 34.80,153.69) {0.6};

\node[text=drawColor,anchor=base east,inner sep=0pt, outer sep=0pt, scale=  1.00] at ( 34.80,192.07) {0.8};

\node[text=drawColor,anchor=base east,inner sep=0pt, outer sep=0pt, scale=  1.00] at ( 34.80,230.45) {1.0};

\path[draw=drawColor,line width= 0.4pt,line join=round,line cap=round] ( 42.00, 42.00) --
	(277.08, 42.00) --
	(277.08,253.08) --
	( 42.00,253.08) --
	( 42.00, 42.00);
\end{scope}
\begin{scope}
\path[clip] (  0.00,  0.00) rectangle (289.08,289.08);
\definecolor{drawColor}{RGB}{0,0,0}

\node[text=drawColor,anchor=base,inner sep=0pt, outer sep=0pt, scale=  1.00] at (159.54,  6.00) {$X$};

\node[text=drawColor,rotate= 90.00,anchor=base,inner sep=0pt, outer sep=0pt, scale=  1.00] at ( 13.20,147.54) {Probability density $f(x)$};
\end{scope}
\end{tikzpicture}
}
\end{center}
Calcular:
\begin{enumerate}
\item $P(\frac{1}{2}\leq X\leq \frac{5}{4})$.
\item Función de distribución.
\item Media y desviación típica.
\end{enumerate}
}
%SOLUCIÓN
{
\begin{enumerate}
\item $P(\frac{1}{2}\leq X\leq \frac{5}{4})=19/32$.
\item \[
F(x)=
\begin{cases}
0 & \text{si $x<0$,}\\
\frac{x^2}{2} & \text{si $0\leq x\leq 1$,}\\
-\frac{x^2}{2}+2x-1 & \text{si $1<x\leq 2$,}\\
1 & \text{si $2<x$.}
\end{cases}
\]
\item $\mu=1$, $\sigma^2=1/6$ y $\sigma=1/\sqrt{6}$.
\end{enumerate}
}
%RESOLUCIÓN
{}


\newproblem{vac-3}{gen}{*}
%ENUNCIADO
{Given the continuous random variable $X$ with the probability density function chart below,
\begin{center}
\scalebox{0.7}{% Created by tikzDevice version 0.10.1 on 2016-05-10 11:29:28
% !TEX encoding = UTF-8 Unicode
\begin{tikzpicture}[x=1pt,y=1pt]
\definecolor{fillColor}{RGB}{255,255,255}
\path[use as bounding box,fill=fillColor,fill opacity=0.00] (0,0) rectangle (289.08,289.08);
\begin{scope}
\path[clip] ( 42.00, 42.00) rectangle (277.08,253.08);
\definecolor{drawColor}{RGB}{5,161,230}

\path[draw=drawColor,line width= 0.8pt,line join=round,line cap=round] ( 50.71, 42.00) --
	(195.82,233.89) --
	(268.37,233.89) --
	(268.37, 42.00);
\end{scope}
\begin{scope}
\path[clip] (  0.00,  0.00) rectangle (289.08,289.08);
\definecolor{drawColor}{RGB}{0,0,0}

\path[draw=drawColor,line width= 0.4pt,line join=round,line cap=round] ( 50.71, 42.00) -- (268.37, 42.00);

\path[draw=drawColor,line width= 0.4pt,line join=round,line cap=round] ( 50.71, 42.00) -- ( 50.71, 36.00);

\path[draw=drawColor,line width= 0.4pt,line join=round,line cap=round] (123.26, 42.00) -- (123.26, 36.00);

\path[draw=drawColor,line width= 0.4pt,line join=round,line cap=round] (195.82, 42.00) -- (195.82, 36.00);

\path[draw=drawColor,line width= 0.4pt,line join=round,line cap=round] (268.37, 42.00) -- (268.37, 36.00);

\node[text=drawColor,anchor=base,inner sep=0pt, outer sep=0pt, scale=  1.00] at ( 50.71, 25.20) {0.0};

\node[text=drawColor,anchor=base,inner sep=0pt, outer sep=0pt, scale=  1.00] at (123.26, 25.20) {0.5};

\node[text=drawColor,anchor=base,inner sep=0pt, outer sep=0pt, scale=  1.00] at (195.82, 25.20) {1.0};

\node[text=drawColor,anchor=base,inner sep=0pt, outer sep=0pt, scale=  1.00] at (268.37, 25.20) {1.5};

\path[draw=drawColor,line width= 0.4pt,line join=round,line cap=round] ( 42.00, 42.00) -- ( 42.00,233.89);

\path[draw=drawColor,line width= 0.4pt,line join=round,line cap=round] ( 42.00, 42.00) -- ( 36.00, 42.00);

\path[draw=drawColor,line width= 0.4pt,line join=round,line cap=round] ( 42.00, 80.38) -- ( 36.00, 80.38);

\path[draw=drawColor,line width= 0.4pt,line join=round,line cap=round] ( 42.00,118.76) -- ( 36.00,118.76);

\path[draw=drawColor,line width= 0.4pt,line join=round,line cap=round] ( 42.00,157.13) -- ( 36.00,157.13);

\path[draw=drawColor,line width= 0.4pt,line join=round,line cap=round] ( 42.00,195.51) -- ( 36.00,195.51);

\path[draw=drawColor,line width= 0.4pt,line join=round,line cap=round] ( 42.00,233.89) -- ( 36.00,233.89);

\node[text=drawColor,anchor=base east,inner sep=0pt, outer sep=0pt, scale=  1.00] at ( 34.80, 38.56) {0.0};

\node[text=drawColor,anchor=base east,inner sep=0pt, outer sep=0pt, scale=  1.00] at ( 34.80, 76.93) {0.2};

\node[text=drawColor,anchor=base east,inner sep=0pt, outer sep=0pt, scale=  1.00] at ( 34.80,115.31) {0.4};

\node[text=drawColor,anchor=base east,inner sep=0pt, outer sep=0pt, scale=  1.00] at ( 34.80,153.69) {0.6};

\node[text=drawColor,anchor=base east,inner sep=0pt, outer sep=0pt, scale=  1.00] at ( 34.80,192.07) {0.8};

\node[text=drawColor,anchor=base east,inner sep=0pt, outer sep=0pt, scale=  1.00] at ( 34.80,230.45) {1.0};

\path[draw=drawColor,line width= 0.4pt,line join=round,line cap=round] ( 42.00, 42.00) --
	(277.08, 42.00) --
	(277.08,253.08) --
	( 42.00,253.08) --
	( 42.00, 42.00);
\end{scope}
\begin{scope}
\path[clip] (  0.00,  0.00) rectangle (289.08,289.08);
\definecolor{drawColor}{RGB}{0,0,0}

\node[text=drawColor,anchor=base,inner sep=0pt, outer sep=0pt, scale=  1.00] at (159.54,  6.00) {$X$};

\node[text=drawColor,rotate= 90.00,anchor=base,inner sep=0pt, outer sep=0pt, scale=  1.00] at ( 13.20,147.54) {Probability density $f(x)$};
\end{scope}
\end{tikzpicture}
}
\end{center}
\begin{enumerate}
\item Check that $f(x)$ is a probability density function.
\item Compute the following probabilities
\begin{enumerate}
\item $P(X<1)$
\item $P(X>0)$
\item $P(X=1/4)$
\item $P(1/2\leq X\leq 3/2)$
\end{enumerate}
\item Compute the distribution function.
\end{enumerate}
}
%SOLUCIÓN
{
\begin{enumerate}[start=2]
\item $P(X<1)=0.5$ , $P(X>0)=1$, $P(X=1/4)=0$ and $P(1/2\leq X\leq 3/2)=0.875$.
\item \[
F(x)=
\begin{cases}
0 & \mbox{if $x<0$,}\\
\frac{x^2}{2} & \mbox{if $0\leq x\leq 1$,}\\
x-0.5 & \mbox{if $1<x\leq 1.5$,}\\
1 & \mbox{if $1.5<x$.}
\end{cases}
\]
\end{enumerate}
}
%RESOLUCIÓN
{}


\newproblem{vac-4}{amb}{*}
%ENUNCIADO
{It is known that the battery lifetime of a cardiac pacemaker (in years) is a random variable with the following probability density function
\[
f(x)=
\begin{cases}
0 & \mbox{if $x<0$,}\\
\dfrac{e^{-x/10}}{10} & \mbox{if $x\geq 0$.}
\end{cases}
\]
\begin{enumerate}
\item Check that $f(x)$ is a probability density function.
\item Compute the distribution function.
\item Compute the probability that the battery lifetime is less than 5 years, and from 5 to 10 years.
\item Compute the expected value for the battery lifetime.
\end{enumerate}
}
%SOLUCIÓN
{
\begin{enumerate}
\item $\int_{-\infty}^\infty f(x)\, dx = [-e^{-x/10}]_{-\infty}^\infty = 1$.
\item \[
F(x)=
\begin{cases}
0 & \mbox{if $s<0$,}\\
1-e^{-x/10} & \mbox{if $x\geq 0$.}
\end{cases}
\]
\item $P(X<5)=0.3935$ and $P(5<X<10)=0.2387$.
\item $\mu=10$ years.
\end{enumerate}
}
%RESOLUCIÓN
{}


\newproblem*{vac-5}{amb}{*}
% ENUNCIADO
{La proporción de cierto aditivo en la gasolina determina su precio. Si en la producción de gasolina la proporción de
aditivo es una variable aleatoria $X$ con función de densidad $f(x)=6x(1-x)$ para $0\leq x\leq 1$, de manera que si
$x<0,5$ la gasolina es del tipo I y se vende a $0.6$ \euro/l, si $0.5\leq x\leq 0.8$ la gasolina es del tipo II y se
vende a $0.7$ \euro/l, y si $x>0.8$ la gasolina es del tipo III que se vende a $0.8$ \euro/l, se pide:
\begin{enumerate}
\item Calcular la función de distribución de $X$.
\item Calcular los porcentajes de producción de cada tipo de gasolina.
\item Calcular el precio medio por litro.
\end{enumerate}
}
%SOLUCIÓN
{}
%RESOLUCIÓN
{}


\newproblem{vac-6}{gen}{}
%ENUNCIADO
{A worker can arrive to the workplace in any instant between 6 and 7 in the morning with the same likelihood.
\begin{enumerate}
\item Compute and plot the probability density function of the variable that measures the arrival time.
\item compute and plot the distribution function.
\item Compute the probability of arriving quarter past six and half past six.
\item What is the expected arrival time?
\end{enumerate}
}
%SOLUCIÓN
{
\begin{enumerate}
\item \[
f(x)=
\begin{cases}
0 & \mbox{if $x<6$,}\\
1 & \mbox{if $6\leq x\leq 7$,}\\
0 & \mbox{if $7<x$.}
\end{cases}
\]
\item \[
F(x)=
\begin{cases}
0 & \mbox{if $x<6$,}\\
x-6 & \mbox{if $6\leq x\leq 7$,}\\
1 & \mbox{if $7<x$.}
\end{cases}
\]
\item $P(6.25<X<6.5)=0.25$.
\item $\mu=6.5$.
\end{enumerate}
}
%RESOLUCIÓN
{}


\newproblem{vac-7}{gen}{}
%ENUNCIADO
{Sea $Z$ una variable aleatoria que sigue una distribución $N(0,1)$.
Determinar el valor de $t$ en cada uno de los siguientes casos:
\begin{enumerate}
\item El área entre $0$ y $t$ es $0.4783$.
\item El área a la izquierda de $t$ es $0.6406$.
\item El área entre $-1.5$ y $t$ es $0.2313$.
\end{enumerate}
}
%SOLUCIÓN
{
\begin{enumerate}
\item $t=2.02$.
\item $t=0.36$.
\item $t=-0.53$.
\end{enumerate}
}
%RESOLUCIÓN
{}


\newproblem*{vac-8}{gen}{}
%ENUNCIADO
{Hallar las siguientes probabilidades:

\begin{enumerate}
\item  $P(-2.4$ $\leq Z\leq -1.2)$ si $Z$ es $N(0,1)$.
\item  $P(\left| Z\right| >1.2)$ si $Z$ es $N(0,1)$.
\item  $P(1.3\leq X\leq 3.3)$ si $X$ es $N(2,1)$.
\item  $P(\left| X-3\right| >2)$ si $X$es $N(3,4)$.
\end{enumerate}
}
%SOLUCIÓN
{}
%RESOLUCIÓN
{}


\newproblem*{vac-9}{gen}{*}
%ENUNCIADO
{Supongamos una variable aleatoria $X$ que sigue una distribución normal de desviación típica 1 ($\sigma =1)$, y media ($\mu )$ desconocida. Si nos dan como dato que la función de densidad en $x$=1 vale $\dfrac{1}{\sqrt{2\pi }},$ y teniendo en cuenta que la función de densidad viene dada por:
\[
f(x)=\dfrac{1}{\sigma \sqrt{2\pi }}\ e^{-\dfrac{(x-\mu )^{2}}{2\sigma ^{2}}},
\]
calcular $P(\left| X-2\right| <3)$.
}
%SOLUCIÓN
{}
%RESOLUCIÓN
{}


\newproblem{vac-10}{med}{}
%ENUNCIADO
{It is known that the cholesterol level in males 30 years old follows a normal probability distribution model with mean 220 mg/dl and standard deviation 30 mg/dl.
If there are 20000 males 30 years old in the population,
\begin{enumerate}
\item how many of them have a cholesterol level between 210 and 240 mg/dl?
\item If a cholesterol level greater than 250 mg/dl can provoke a thrombosis, how many of them are in risk of thrombosis?
\item Compute the cholesterol level above which 20\% of the males are?
\end{enumerate}
}
%SOLUCIÓN
{
\begin{enumerate}
\item $P(210\leq X\leq 240)=0.3781\rightarrow 7561.3$ males.
\item $P(X>250)=0.1587\rightarrow 3173.1$ males.
\item $P_{80}=245.2486$ mg/dl.
\end{enumerate}
}
%RESOLUCIÓN
{}


\newproblem{vac-11}{med}{}
%ENUNCIADO
{It is known that the glucose level in blood of diabetic persons follows a normal probability distribution model with
mean 106 mg/100 ml and standard deviation 8 mg/100 ml.
\begin{enumerate}
\item Compute the probability that a random diabetic person has a glucose level less than 120 mg/100 ml.
\item What percentage of persons have a glucose level between 90 and 120 mg/100 ml?
\item Compute and interpret the first quartile of the glucose level.
\end{enumerate}
}
%SOLUCIÓN
{
\begin{enumerate}
\item $P(X\leq 120)=0.9599$.
\item $P(90< X<120)=0.9371$, that is, $93.71\%$.
\item $100.64$ mg/$100$ ml.
\end{enumerate}
}
%RESOLUCIÓN
{}


\newproblem*{vac-12}{gen}{}
%ENUNCIADO
{Se sabe que la distribución de probabilidad de una variable aleatoria $X$ sigue la ley normal. Si conocemos $P(X>3.6)=0.3821$ y $P(X<5.8)=0.9192$, calcular $\mu $ y $\sigma $ de la distribución normal.
}
%SOLUCIÓN
{}
%RESOLUCIÓN
{}


\newproblem{vac-13}{med}{}
%ENUNCIADO
{In a population with 40000 persons, 2276 have between 0.80 and 0.84 milligrams of bilirubin per deciliter of blood, and 11508 have more than 0.84.
Assuming that the level of bilirubin in blood follows a normal probability distribution model,
\begin{enumerate}
\item Compute the mean and the standard deviation.
\item How many persons have more than 1 mg of bilirubin per dl of blood?
\end{enumerate}
}
%SOLUCIÓN
{
\begin{enumerate}
\item $\mu=0.7$ mg and $\sigma=0.25$ mg.
\item $P(X>1)=0.1151 \rightarrow 4604$ persons.
\end{enumerate}
}
%RESOLUCIÓN
{}


\newproblem{vac-14}{gen}{*}
%ENUNCIADO
{En un examen, el 63\% de los alumnos ha obtenido una nota superior a 5, y el 44\% entre 5 y 7.
Suponiendo que las notas siguen una distribución normal:

\begin{enumerate}
\item  Calcular la media y la desviación típica de las notas.
\item  Calcular el porcentaje de alumnos con nota superior a 8.
\item  ¿Cuál es la nota por encima de la cual está el 5\% de los alumnos?
\end{enumerate}
}
%SOLUCIÓN
{Llamando $X$ a la nota obtenida en el examen, se tiene que $X\sim N(\mu,\sigma)$.
\begin{enumerate}
\item $\mu=5.55$ puntos y $\sigma=1.65$ puntos.
\item $P(X>8)=0.0648$, es decir, el $6.48\%$ de los alumnos han tenido una nota superior a 8.
\item $8.21$ puntos.
\end{enumerate}
}
%RESOLUCIÓN
{}


\newproblem*{vac-15}{med}{*}
%ENUNCIADO
{Se realiza un estudio sobre la tensión arterial máxima en varones de edad comprendida entre 30 y 40 años. Como
consecuencia del estudio se sabe que el 40\% la tiene por encima de 140 mmHg y el 20\% por debajo de 110 mmHg.
Suponiendo que la tensión arterial máxima sigue una distribución normal, se pide:
\begin{enumerate}
\item Calcular la media y la desviación típica.
\item Calcular el porcentaje de hipertensos, si se considera hipertenso a una persona cuya tensión arterial máxima sea mayor que 150 mmHg.
\item ¿Cuál es el valor de la tensión arterial máxima por encima del cual se espera que esté el 10\% de la población?
\end{enumerate}
}
%SOLUCIÓN
{}
%RESOLUCIÓN
{}


\newproblem{vac-16}{gen}{*}
%ENUNCIADO
{In an exam done by 100 students, the average degree was $4.2$ and only 32 students passed.
Assuming that the grade follows a normal probability distribution model, how many students got a grade greater than 7?
}
%SOLUCIÓN
{$P(X>7)=0.0508\rightarrow 5.1$ students.
}
%RESOLUCIÓN
{}


\newproblem{vac-17}{med}{*}
%ENUNCIADO
{It is known that the blood pressure of people in a population with 20000 persons follows a normal distribution
model with mean 13 mm Hg and interquartile range 4 mm Hg.
\begin{enumerate}
\item How many persons have a blood pressure above 16 mm Hg?
\item How much has to decrease the blood pressure of a person with 16 mm Hg in order to be below the 40\% of
people with lowest blood pressure?
\end{enumerate}
}
%SOLUCIÓN
{
\begin{enumerate}
\item $P(X>16)=0.1587 \rightarrow 3174$ persons.
\item It must decrease at least $3.75$ mmHg.
\end{enumerate}
}
%RESOLUCIÓN
{}


\newproblem{vac-18}{med}{*}
%ENUNCIADO
{En una población de 30000 individuos se está interesado en medir el volumen sanguíneo de sus individuos.
Se sabe que la desviación típica de la población es $0.4$ litros y que el 50\% de los individuos tienen un volumen
superior a $4.8$ litros.
¿Cuántos individuos presentarán un volumen menor de $4.3$ litros?}
%SOLUCIÓN
{El número de individuos con menos de $4.3$ litros de sangre es $3168$.}
%RESOLUCIÓN
{}


\newproblem*{vac-19}{med}{}
%ENUNCIADO
{Se sabe que el nivel de potasio en personas sanas sigue una distribución normal de media $4.4$ mg/l y desviación típica $0.4$ mg/l. En un estudio realizado sobre 20000 personas sanas:

\begin{enumerate}
\item  ¿Cuántos se espera que tengan un nivel de potasio entre $3.7$ y $5$ mg/l?
\item  ¿Cuántos se espera que tengan un nivel de potasio por encima de $5.3$ mg/l?
\item  ¿Cuál será el nivel de potasio por encima del cual está la cuarta parte de la población?
\end{enumerate}
}
%SOLUCIÓN
{}
%RESOLUCIÓN
{}


\newproblem{vac-20}{gen}{}
% ENUNCIADO
{Se consideran las variables aleatorias $X_1$ y $X_2$. La variable $X_1$ sigue una distribución normal de media $\mu$ y
desviación típica $\sigma$, mientras que la variable $X_2$ sigue también una distribución normal de media $\mu + 1$ y
desviación típica $\sigma$. Si la probabilidad de que $X_1$ tome valores superiores a $14.2$ es $0.5636$, y la de que
$X_2$ tome valores inferiores a $17.4$ es $0.6103$, se pide:
\begin{enumerate}
\item Hallar los valores de $\mu$ y $\sigma$.
\item Si se rechazan los individuos que están fuera del intervalo $(12,18)$, hallar los porcentajes de rechazo
correspondientes a $X_1$ y $X_2$.
\item Si se desea seleccionar el $20\%$ de individuos que tengan los valores más altos de $X_1$, ¿cuál será el valor
de $X_1$ a partir del cuál se seleccionarán?
\end{enumerate}
}
%SOLUCIÓN
{
\begin{enumerate}
\item $\mu=15$ y $\sigma=5$.
\item $P(12<X_1<18)=0.4515$, luego el porcentaje de rechazos es $100\%-45.15\%=54.85\%$.\\
$P(12<X_2<18)=0.4436$, luego el porcentaje de rechazos es $100\%-44.36\% = 55.64\%$.
\item $19.21$.
\end{enumerate}
}
%RESOLUCIÓN
{}


\newproblem{vac-21}{med}{*}
% ENUNCIADO
{El peso de los recién nacidos no prematuros en una ciudad sigue una distribución normal de media y desviación típica
desconocidas.
Teniendo en cuenta que, de un total de 200 recién nacidos no prematuros, 15 han pesado más de 4 kg y 25 menos de $2,5$
kg:
\begin{enumerate}
\item ¿Cuáles son la media y la desviación típica del peso?
\item ¿Cuántos niños no prematuros habrán nacido con un peso entre $3$ y $3.5$ kg?
\item Si los médicos consideran peligrosos los pesos por debajo del percentil 10, ¿cuál será dicho peso?, ¿cuántos
niños habrán nacido con un peso por debajo de dicho percentil?
\end{enumerate}
}
%SOLUCIÓN
{Llamando $X$ al peso de los recién nacidos no prematuros, se tiene que $X\sim N(\mu,\sigma)$.
\begin{enumerate}
\item $\mu=3.17$ kg y $\sigma=0.58$ kg.
\item 66 niños.
\item $P_{10}=2.43$ kg y habrán nacido 20 niños por debajo de este peso.
\end{enumerate}
}
%RESOLUCIÓN
{}


\newproblem*{vac-22}{amb}{}
%ENUNCIADO
{La cantidad de fuel en aguas costeras próximas al hundimiento de un petrolero sigue una distribución normal de media 20 $\mu$g/l y desviación típica 8 $\mu$g/l. Si se considera que existe contaminación si el nivel de fuel excede los 5 $\mu$g/l, ¿Cual es la probabilidad de que al realizar una medición al azar no se detecte la contaminación? ¿Y si en lugar de realizar una medición, realizamos dos y tomamos la media?
}
%SOLUCIÓN
{}
%RESOLUCIÓN
{}


\newproblem*{vac-23}{med}{*}
%ENUNCIADO
{Una solución contiene virus bacteriófagos $T_4$ en una concentración de $4\cdot10^6$ por mm$^3$.
En la misma solución hay $2\cdot10^6$ bacterias por mm$^3$.
Suponiendo que todos los virus infectan bacterias y que se distribuyen al azar entre las mismas, se pide:
\begin{enumerate}
\item ¿Cuál es el porcentaje de bacterias que no están infectadas por el virus?
\item ¿Qué porcentaje de bacterias tendrá al menos 2 virus fijados sobre ellas?
\item Si tomamos un volumen pequeño de dicha solución en el que hay 4 bacterias, ¿cuál es la probabilidad de que alguna
esté infectada?
\item Si tomamos un volumen en el que hay 10000 bacterias, ¿cuál es la probabilidad de que estén infectadas al menos 8600?
\end{enumerate}
}
%SOLUCIÓN
{}
%RESOLUCIÓN
{}


\newproblem{vac-24}{psi}{}
%ENUNCIADO
{El coeficiente intelectual es una puntuación derivada de los test de inteligencia que tiene media 100 puntos y
desviación típica 15.
Si se considera que una persona por encima de 145 es una superdotada, ¿qué porcentaje de superdotados habrá en la
población?
Si se considera que el 1\% de las personas con menor coeficiente intelectual son deficientes, ¿por debajo de qué
coeficiente estarán dichas personas? }
%SOLUCIÓN
{Llamando $X$ al coeficiente intelectual, se tiene que $X\sim N(100,15)$.\\
$P(X>145)=0.0013$, luego el $0.13\%$ de las personas son superdotadas.\\
$P_1=65.10$, por debajo de este coeficiente las personas son deficientes.}
%RESOLUCIÓN
{}


\newproblem*{vac-25}{amb}{*}
%ENUNCIADO
{Si se supone que la concentración de plomo en suelo sigue una distribución normal de media 320 mg/Kg y desviación típica desconocida, y sabiendo que el 25\% de los suelos presenta una concentración de plomo por abajo de 240 mg/Kg:
\begin{enumerate}
\item ¿Cuánto vale la desviación típica de la distribución?
\item Suponiendo que niveles de Plomo por arriba de 250 mg/Kg son peligrosos para la salud, y se catalogan como contaminados todos los suelos que sobrepasen ese nivel, ¿cuál es la probabilidad de que un suelo analizado esté contaminado?
\item Si analizamos 10000 suelos diferentes, ¿cuántos esperamos que no estén contaminados?
\item ¿Cuánto vale el rango intercuartílico de la distribución? ¿Qué interpretación tienen?
\end{enumerate}
}
%SOLUCIÓN
{}
%RESOLUCIÓN
{}


\newproblem*{vac-26}{amb}{*}
%ENUNCIADO
{Para la recuperación de una lesión en encinas parasitadas por un tipo de hongo se dispone de dos tratamientos alternativos A y B. El tiempo de recuperación con el procedimiento A sigue una distribución normal de media 16 meses y desviación típica 3 meses, mientras que con el procedimiento B la media es de 20 meses y la desviación típica de 1.
\begin{enumerate}
\item ¿Qué tanto por ciento de recuperaciones se producen con cada tratamiento entre 18 y 21 meses?
\item ¿Qué tratamiento consigue antes la recuperación del 90\% de las encinas tratadas?. Justificar adecuadamente la respuesta.
\item Si a una encina le aplicamos los dos tratamientos, que suponemos que actúan de forma independiente, ¿cuál es la
probabilidad de que cure antes de 19 meses?
\end{enumerate}
}
%SOLUCIÓN
{}
%RESOLUCIÓN
{}


\newproblem*{vac-27}{gen}{}
%ENUNCIADO
{Calcular:
\begin{enumerate}
\item  $P(T\leq 1.476)$ si $T\sim T(5)$.
\item  $P(T\geq 0.69)$ si $T\sim T(16)$.
\item  El valor $t_{0}$ tal que $P(T<t_{0})=0.995$, con $T\sim T(12)$.
\item  El valor $t_{0}$ tal que $P(T>t_{0})=0.01$, con $T\sim T(8)$.
\end{enumerate}
}
%SOLUCIÓN
{}
%RESOLUCIÓN
{}


\newproblem*{vac-28}{gen}{}
%ENUNCIADO
{Calcular:

\begin{enumerate}
\item  $P(X\leq 5.23)$ si $X\sim \chi ^{2}(12)$.
\item  $P(X\geq 1.65)$ si $X\sim \chi ^{2}(8)$.
\item  El valor $x_{0}$ tal que $P(X<x_{0})=0.995$, con $X\sim \chi ^{2}(18)$.
\item  El valor $x_{0}$ tal que $P(X>x_{0})=0.25$, con $X\sim \chi ^{2}(7)$.
\end{enumerate}
}
%SOLUCIÓN
{}
%RESOLUCIÓN
{}


\newproblem*{vac-29}{gen}{}
%ENUNCIADO
{Calcular:

\begin{enumerate}
\item  El valor $f_{0}$ tal que $P(F<f_{0})=0.9$, con $F\sim F(12,8)$.
\item  El valor $f_{0}$ tal que $P(F>f_{0})=0.025$, con $F\sim F(5,7)$.
\end{enumerate}
}
%SOLUCIÓN
{}
%RESOLUCIÓN
{}


\newproblem*{vac-30}{med}{*}
%ENUNCIADO
{Suponiendo que la duración del embarazo sigue una distribución normal de media y desviación típica desconocidas, y
teniendo en cuenta que el 80\% de las mujeres dan a luz antes de 40 semanas, y que el 70\% lo hacen después de 38
semanas y 2 días, se pide:

\begin{enumerate}
\item Calcular la media y la desviación típica de la distribución dadas en número de semanas.\\
\item Suponiendo un hospital en el que se han producido 2000 nacimientos, ¿cuántos lo habrán hecho con más de 282 días de gestación?
\item Suponiendo una mujer que ha tenido 2 embarazos y que la duración del segundo ha sido independiente del primero, ¿cuál es la probabilidad de que alguno de los partos se haya producido antes de los 275 días de embarazo?
\end{enumerate}
}
%SOLUCIÓN
{}
%RESOLUCIÓN
{}


\newproblem{vac-31}{far}{*}
%ENUNCIADO
{El gasto mensual en medicamentos de las familias españolas antes de la crisis seguía una distribución normal $X\sim
N(160,\sigma)$, mientras que ahora sigue una distribución normal $Y\sim N(\mu,2\sigma)$.
Sabiendo que antes de la crisis el $10\%$ de las familias gastaba más de 200 euros y que después el $40\%$ gastaba
menos de 100 euros, se pide:
\begin{enumerate}
\item ¿Qué porcentaje de familias gastarán ahora entre 110 y 120 euros?
\item ¿Qué percentil de la distribución actual se corresponde con el tercer decil de la distribución de antes de la crisis?
\end{enumerate}
}
%SOLUCIÓN
{$X\sim N(160,\,31.25)$ y $Y\sim N(115.625,\,62.50)$.
\begin{enumerate}
\item $P(110\leq Y\leq 120)=0.0638$.
\item El decil tercero en $X$ es $D_3=143.75$ euros y se corresponde aproximadamente con el percentil 67 de $Y$.
\end{enumerate}
}
%RESOLUCIÓN
{\begin{enumerate}
\item Teniendo en cuenta que el gasto antes de la crisis seguía una distribución normal de media 160 euros y desviación típica desconocida, pero nos dan el dato de que antes de la crisis el $10\%$ de las familias gastaba más de 200, fácilmente podemos aprovechar este dato para calcular la desviación típica, y utilizamos esta desviación típica y el dato de que el $40\%$ de las familias gastan ahora menos de 100 euros para calcular la media del gasto después. Con ello, podemos calcular la probabilidad que nos piden: $P(110 \leq Y \leq 120)$.
Utilizamos el dato de que $P(X>200)=0.1000$:
\begin{align*}
P(X > 200) &= 1 - P(X \le 200) = 1 - P\left( {Z \le \frac{{200 - 160}}{\sigma }} \right) = 1 - P\left( {Z \le \frac{{40}}{\sigma }} \right) =\\
&= 1 - F\left( {\frac{{40}}{\sigma }} \right)=0.1000 \Leftrightarrow F\left( {\frac{{40}}{\sigma }} \right) = 0.9000.
\end{align*}

Mediante la tabla de la distribución normal tipificada vemos que $F(1.28)=0.8997$, que es la probabilidad acumulada más cercana al $0.9000$ que buscamos. Por lo tanto:
\[
\frac{{40}}{\sigma } = 1.28 \Leftrightarrow \sigma  = \frac{{40}}{{1.28}} = 31.25\;\mbox{euros}.
\]

Para calcular la media $\mu$ del gasto después, utilizamos el dato de que $P(Y<100)=0.4000$:
\begin{align*}
P(Y < 100) &= P\left({Z \le \frac{{100 - \mu }}{{2\sigma }}} \right) = P\left( {Z \le \frac{{100 - \mu }}{{2 \cdot 31.25}}} \right) = P\left( {Z \le \frac{{100 - \mu }}{{62.50}}} \right)=\\
&=F\left( {\frac{{100 - \mu }}{{62.50}}} \right)=0.4000.
\end{align*}
De nuevo, buscamos en la tabla de la normal tipificada y observamos que $F(-0.25)=0.4013$, que es la probabilidad acumulada más cercana al $0.4000$ que buscamos. Por lo tanto:
\[
\frac{{100 - \mu }}{{62.50}} =  - 0.25 \Leftrightarrow \mu  = 100 + 15.625 = 115.625\,\mbox{euros}.
\]

Por lo tanto, concluimos que $X\sim N(160\,,\,31.25)$ y $Y\sim N(115.625\,,\,62.50)$.

Por último, calculamos la probabilidad (el porcentaje) de que el gasto ahora esté entre 110 y 120 euros:
\begin{align*}
P(110 \le Y \le 120) &= P\left( {\frac{{110 - 115.625}}{{62.50}} \le Z \le \frac{{120 - 115.625}}{{62.50}}} \right) = P\left( { - 0.09 \le Z \le 0.07} \right) =\\
&= F(0.07) - F( - 0.09) \stackrel{1}{=} 0.5279 - 0.4641 = 0.0638=6.38\%.
\end{align*}
\begin{quotation}
\footnotesize (1) Mirando en la tabla de la función de distribución de la normal estándar.
\end{quotation}

\item Para calcular el tercer decil de antes de la crisis, sabemos que $P(X\leq D_3)=0.3000$. Por lo tanto:
\[
P\left( {Z \le \frac{{D_3  - 160}}{{31.25}}} \right) = F\left( {\frac{{D_3  - 160}}{{31.25}}} \right) = 0.3000
\]
Mediante la tabla de la normal tipificada, la probabilidad acumulada más cercana es: $F(-0.52)=0.3015$. Por lo tanto:
\[
\frac{{D_3  - 160}}{{31.25}} =  - 0.52 \Leftrightarrow D_3  = 143.75\,\mbox{euros},
\]
y para ver qué porcentaje de familias consumen ahora, durante la crisis, menos de $143.75$ euros:
\[
P(Y \le 143.75) = P\left( {Z \le \frac{{143.75 - 115.625}}{{62.50}}} \right) = P(Z \le 0.45) = F(0.45) = 0.6736.
\]
Por lo tanto, los $143.75$ euros son el percentil 67 de la distribución de gasto durante la crisis.
\end{enumerate}
}


\newproblem{vac-32}{psi}{}
%ENUNCIADO
{Un test diagnóstico para niños de 10 años con problemas de lectura da puntuaciones que se distribuyen normalmente con
una media de 80 y una varianza de 100.
Se pide:
\begin{enumerate}
\item Dar la probabilidad de que un niño seleccionado al azar tenga una puntuación de:
\begin{enumerate}
\item menos de 68;
\item entre 75 y 90;
\end{enumerate}
\item El test indica que el niño tiene un problema del lenguaje si su puntuación está por debajo del 10\% de la
población que realizó el test.
¿Por debajo de qué puntuación el test diagnosticará problemas de lenguaje?
\item Si se selecciona una muestra de 16 niños:
\begin{enumerate}
\item ¿Cuantos se espera que tengan una puntuación por encima de 68?
\item ¿Cuál es la probabilidad de que su puntuación media supere los 84 puntos?
\end{enumerate}
\end{enumerate}
}
%SOLUCIÓN
{Llamando $X$ a la puntuación del test diagnóstico, se tiene que $X\sim N(80,10)$.
\begin{enumerate}
\item $P(X<68)=0.1151$ y $P(75<X<90)=0.5328$.
\item $P_{10}=67.18$ puntos.
\item $P(X>68)=0.8849$ y el número esperado niños con puntuaciones por encima de 68 es \mbox{$0.8849\cdot 16=14.16$}.\\
$\bar x\sim N(80,\,2.5)$ y $P(\bar x>84)=0.0548$.
\end{enumerate}
}
%RESOLUCIÓN
{}


\newproblem{vac-33}{fis}{*}
%ENUNCIADO
{The curing time of a knee injury in soccer players follows a normal distribution model with mean 50 days and standard deviation 10 days.
If there is a final match in 65 days, what is the probability that a player that has just injured his knee will miss the final?
If the semifinal match is in 40 days, and 4 players has just injured the knee, what is the probability that some of them can play the semifinal?
}
%SOLUCIÓN
{Naming $X$ to the curing time, $P(X>65)=0.0668$.\\
Naming $Y$ to the number of injured players that could play the semifinal, $P(Y\geq 1)=0.4989$.
}
%RESOLUCIÓN
{Llamemos $X$ a la variable aleatoria que mide el número de días que tarda en recuperarse un futbolista con la lesión de rodilla. Entonces,
según el enunciado, $X\sim N(50,10)$. La probabilidad de que un futbolista tarde más de 65 días en curarse es $P(X>65)$. Para calcularla, tipificamos
\[
P(X>65)=P(\frac{X-50}{10}>\frac{65-50}{10})=P(Z>1.5)=1-P(Z\leq 1.5)=1-F(1.5),
\]
y mirando en la tabla de la función de distribución de la normal estándar, la probabilidad acumulada hasta 1.5 es 0.9332. Por tanto,
$P(X>65)=1-0.9332=0.0668.$

Para responder a la segunda pregunta, necesitamos otra variable aleatoria que mida el número de jugadores que tardan menos de 40 días en
curarse de entre 4 que tenemos lesionados. Llamemos $Y$ a dicha variable. Puesto que $Y$ responde a un esquema de repetición de pruebas (una
por jugador), donde cada prueba consiste en ver si el jugador tarda menos de 40 días en curarse (éxito) o no (fracaso), entonces sigue una
distribución binomial de parámetros $n=4$ jugadores y $p=P(X<40)$.

Necesitamos, en primer lugar, calcular esta probabilidad. Al igual que antes, tipificando y mirando en la tabla de la función de
distribución de la normal estándar obtenemos
\[
P(X<40)=P(\frac{X-50}{10}<\frac{40-50}{10})=P(Z<-1)=F(-1)=0.1587.
\]
Así pues, $Y\sim B(4,0.1587)$, y la probabilidad que nos piden es
\[
P(Y\geq 1)=1-P(Y<1)=1-P(Y=0)=1-\binom{4}{0}0.1587^0 (1-0.1587)^4=1-0.8413^4=0.499.
\]
}


\newproblem{vac-34}{fis}{*}
%ENUNCIADO
{According to the central limit theorem, for big samples ($n\geq 30$) the sample mean $\bar x$ follows a normal
distribution model $N(\mu,\sigma/\sqrt{n})$, where $\mu$ is the population mean and $\sigma$ the population standard
deviation.

It is known that in a population the sural triceps elongation has a mean $60$ cm and a standard deviation $15$ cm.
If you draw a sample of 30 individuals from this population, what is the probability of having a sample mean greater
than 62 cm?
If a sample is atypical if its mean is below the 5th percentile, is atypical a sample of 60 individuals with $\bar
x=57$?
}
%SOLUCIÓN
{
\begin{enumerate}
\item Naming $\bar X$ to the variable that measures the sural triceps elongation in samples of size 30, $P(\bar X>62)=0.2327$.
\item Naming $\bar Y$ to the variable that measures the sural triceps elongation in samples of size 60, $P_5=56.8$ cm, so the sample is not atypical.
\end{enumerate}
}
%RESOLUCIÓN
{Sea $\bar X$ la variable que mide la elongación media del triceps sural en muestras de tamaño 30. Por el teorema central del límite,
puesto que tomamos una muestra grande, tenemos que $\bar X\sim N(\mu,\sigma/\sqrt{n})=N(60,15/\sqrt{30})=N(60\,,\,2.74)$. Así pues, la
probabilidad que nos piden es
\begin{align*}
P(\bar X>62)& \stackrel{(1)}{=} P\left(\frac{\bar X-60}{2.74}>\frac{62-60}{2.74}\right)= P(Z>0.73) =\\
&= 1 - P(Z\leq 0.73)=1-F(0.73)\stackrel{(2)}{=}1-0.7673=0.2327.
\end{align*}
\begin{quote}
    \footnotesize
    (1) Tipificando.\\
    (2) Mirando en la tabla de la función de distribución de la normal estándar.
\end{quote}

Para responder a la segunda pregunta, llamemos $\bar Y$ a la variable que mide la elongación media del triceps sural en muestras de tamaño
60. Al igual que antes, por el teorema central del límite, $N(\mu,\sigma/\sqrt{n})=N(60,15/\sqrt{60})=N(60\,,\,1.94)$. Para saber si una
muestra es atípica, tenemos que calcular el percentil 5 de $\bar Y$.
Dicho valor cumplirá $P(\bar Y\leq y_0)=0.05$. Tipificando tenemos
\[
P(\bar Y\leq y_0)=
P\left(\frac{\bar Y-60}{1.94}\leq\frac{y_0-60}{1.94}\right)=
P\left(Z\leq\frac{y_0-60}{1.94}\right)=F\left(\frac{y_0-60}{1.94}\right)=0.05.
\]
Y buscando en la tabla de la función de distribución de la normal estándar, tenemos que
\[
\frac{y_0-60}{1.94}=-1.65 \Leftrightarrow y_0=-1.65\cdot 1.94+60= 56.8 \text{ cm}.
\]
Como la media muestral obtenida es 57 y no está por debajo de $56.8$ que es el percentil 5, concluimos que la muestra obtenida no se puede
considerar atípica.
}


\newproblem{vac-35}{far}{*}
%ENUNCIADO
{Para el tratamiento de una enfermedad se utilizan 3 fármacos diferentes: $A$ en un 20\% de los casos, $B$ en un 30\%, y $C$ en un 50\%. Los
tratados con $A$ curan en media al cabo de $6.2$ días con una desviación típica de $1.1$ días; los tratados con $B$ en $7.1$ días con una
desviación típica de $1.3$ y los tratados con $C$ en $7.4$ días con una desviación típica de $0.9$. Con ello:
\begin{enumerate}
\item En qué fármaco es mayor el percentil $90$, en $B$ o en $C$?
\item Qué probabilidad total hay de que tomado un enfermo al azar haya curado antes de $8.1$ días?
\item Sabiendo que un enfermo ha tardado en curar más de $7.2$ días, con qué fármaco es más probable que haya sido tratado? Justificar
adecuadamente la respuesta.
\end{enumerate}
}
%SOLUCIÓN
{Llamando $X$ al tiempo de curación y $A$, $B$ y $C$ a los sucesos consistentes en haber sido tratado respectivamente con cada uno de los
fármacos:
\begin{enumerate}
\item El percentil 90 en el fármaco $B$ es $8.7660$ días y en $C$ es $8.5534$ días, así que es mayor en el fármaco $B$.
\item $P(X<8.1)=0.8161$.
\item $P(A/X>7.2)=0.0771$, $P(B/X>7.2)=0.2989$ y $P(C/X>7.2)=0.624$, así que es más probable que haya sido tratado con el fármaco $C$.
\end{enumerate}
}
%RESOLUCIÓN
{}


\newproblem{vac-36}{far}{*}
%ENUNCIADO
{Se sabe que la concentración de urea láctica, en mg/cm$^3$, en vacas sanas sigue una distribución normal de media $28$ y desviación típica
$1.5$, mientras que en vacas con una determinada enfermedad sigue una distribución normal de media $34.5$ y desviación típica $2$. Para
detectar la enfermedad se realiza un test diagnóstico que da positivo cuando el nivel de urea láctica está por encima de $32$. Se pide:
\begin{enumerate}
\item ¿Qué sensibilidad $P(+/E)$ y qué especificidad $P(-/\overline{E})$ tiene el test diagnóstico?
\item Si el test da positivo en un 5\% de los casos analizados, ¿cuál será el porcentaje de vacas enfermas en la población?
\item Teniendo en cuenta el porcentaje de vacas enfermas anterior, ¿cuál será la probabilidad de diagnóstico acertado con este test?
\end{enumerate}
}
%SOLUCIÓN
{\begin{enumerate}
\item $P(+/E)=0.8944$ y $P(-/\overline E)=0.9962$.
\item $P(E)=0.0465$.
\item $P(E\cap +)+P(\overline E\cap -)= 0.0416 + 0.9498 = 0.99143$.
\end{enumerate}
}
%RESOLUCIÓN
{}


\newproblem{vac-37}{gen}{*}
%ENUNCIADO
{Let $Z$ be a random variable following a standard normal distribution model.
Calculate the following probabilities using the table of the distribution function:
\begin{enumerate}
\item $P(Z<1.24)$
\item $P(Z>-0.68)$
\item $P(-1.35\leq Z\leq 0.44)$
\end{enumerate}
}
%SOLUCIÓN
{
\begin{enumerate}
\item $P(Z<1.24)=0.8925$.
\item $P(Z>-0.68)=0.7517$.
\item $P(-1.35\leq Z\leq 0.44)=0.5815$.
\end{enumerate}
}
%RESOLUCIÓN
{}


\newproblem{vac-38}{gen}{*}
%ENUNCIADO
{Let $Z$ be a random variable following a standard normal distribution model.
Determine the value of $x$ in the following cases using the table of the distribution function:
\begin{enumerate}
\item $P(Z<x)=0.6406$.
\item $P(Z>x)=0.0606$.
\item $P(0\leq Z\leq x)=0.4783$.
\item $P(-1.5\leq Z\leq x)=0.2313$.
\item $P(-x\leq Z\leq x)=0.5467$.
\end{enumerate}
}
%SOLUCIÓN
{
\begin{enumerate}
\item $x=0.3601$.
\item $x=1.5498$.
\item $x=2.0198$.
\item $x=-0.5299$.
\item $x=0.7499$.
\end{enumerate}
}
%RESOLUCIÓN
{}


\newproblem{vac-39}{gen}{*}
%ENUNCIADO
{Let $X$ be a random variable following a normal distribution model $N(10,2)$.
\begin{enumerate}
\item Compute $P(X\leq 10)$.
\item Compute $P(8\leq X\leq 14)$.
\item Compute the interquartile range.
\item Compute the third decile.
\end{enumerate}
}
%SOLUCIÓN
{
\begin{enumerate}
\item $P(X\leq 10)=0.5$.
\item $P(8\leq X\leq 14)=0.8186$.
\item $RI=2.698$.
\item $D_3=8.9512$.
\end{enumerate}
}
%RESOLUCIÓN
{
}


\newproblem{vac-40}{nut}{*}
%ENUNCIADO
{A study tries to determine the effect of a low fat diet in the lifetime of rats.
The rats where divided into two groups, one with a normal diet and another with a low fat diet.
It is assumed that the lifetime of both groups follows normal distribution model with the same variance but different
mean.
If 20\% of rats with normal diet lived more than 12 months, 5\% less than 8 moth, and 85\% of rats with low fat diet
lived more than 11 moths,
\begin{enumerate}
\item what is the mean and the standard deviation of lifetime of rats following a low fat diet?
\item If there was 40\% of rats with normal diet, and 60\% of rats with low fat diet, what is the probability that a
random rat die before 9 months?
\end{enumerate}
}
%SOLUCIÓN
{Naming $X$ to the lifetime of rats.
\begin{enumerate}
\item $\mu=12.6673$ months and $s=1.6087$ months.
\item $P(X<9)=0.068$.
\end{enumerate}
}
%RESOLUCIÓN
{}


\newproblem{vac-41}{med}{*}
%ENUNCIADO
{A diagnostic test to determine doping of athletes returns a positive outcome when the concentration of a substance
in blood is greater than 4 $\mu$g/ml.
If the distribution of the substance concentration in doped athletes follows a normal distribution model with mean $4.5$
$\mu$g/ml and standard deviation $0.2$ $\mu$g/ml, and in non-doped athletes follow a normal distribution model with mean
$3$ $\mu$g/ml and standard deviation $0.3$ $\mu$g/ml,
\begin{enumerate}
\item what is the sensitivity and specificity of the test?
\item If there are a 10\% of doped athletes in a competition, what is the positive predicted value?
\end{enumerate}
}
%SOLUCIÓN
{Naming $D$ to the event of being doped, $X$ to the concentration of the substance in doped athletes and $Y$ to the concentration of the substance in non-doped athletes.
\begin{enumerate}
\item Sensitivity $P(+|D)=P(X>4)=0.9938$ and specificity $P(-|\overline D)=P(Y<4)=0.9996$.
\item PPV $P(D|+)=0.9961$.
\end{enumerate}
}
%RESOLUCIÓN
{}
