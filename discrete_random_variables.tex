% Author Alfredo Sánchez Alberca (asalber@ceu.es)

\newproblem{vad-1}{gen}{}
%STATEMENT
{Let $X$ be a discrete random variable with the following probability distribution
\[
\begin{array}{|c|c|c|c|c|c|}
\hline
X & 4 & 5 & 6 & 7 & 8 \\ 
\hline
f(x) & 0.15 & 0.35 & 0.10 & 0.25 & 0.15 \\ 
\hline
\end{array}
\]
\begin{enumerate}
\item  Compute and represent graphically the distribution function.
\item  Compute the following probabilities
\begin{enumerate}
\item  $P(X<7.5)$.
\item  $P(X>8)$.
\item  $P(4\leq X\leq 6.5)$.
\item  $P(5<X<6)$.
\end{enumerate}
\end{enumerate}
}
%SOLUTION
{
\begin{enumerate}
\item \[
F(x)=
\begin{cases}
0 & \text{if $x<4$,}\\
0.15 & \text{if $4\leq x<5$,}\\
0.5 & \text{if $5\leq x<6$,}\\
0.6 & \text{if $6\leq x<7$,}\\
0.85 & \text{if $7\leq x<8$,}\\
1 & \text{if $8\leq x$.}
\end{cases}
\]
\item $P(X<7.5)=0.85$, $P(X>8)=0$, $P(4\leq x\leq 6.5)=0.6$ and $P(5<X<6)=0$.
\end{enumerate}
}
%RESOLUTION
{}


\newproblem{vad-2}{gen}{}
%STATEMENT
{Let $X$ be a discrete random variable with the following probability distribution
\[
F(x)=
\begin{cases}
0 & \text{if $x<1$,} \\
1/5 & \text{if $1\leq x< 4$,} \\
3/4 & \text{if $4\leq x<6$,} \\
1 & \text{if $6\leq x$.}
\end{cases}
\]

\begin{enumerate}
\item  Compute the probability function.
\item  Compute the following probabilities
\begin{enumerate}
\item  $P(X=6)$.
\item  $P(X=5)$.
\item  $P(2<X<5.5)$.
\item  $P(0\leq X<4)$.
\end{enumerate}
\item Compute the mean. 
\item Compute the standard deviation. 
\end{enumerate}
}
%SOLUTION
{
\begin{enumerate}
\item \[
\begin{array}{|c|c|c|c|}
\hline
X & 1 & 4 & 6 \\
\hline
f(x) & 1/5 & 11/20 & 1/4\\
\hline
\end{array}
\]
\item $P(X=6)=1/4$, $P(X=5)=0$, $P(2<X<5.5)=11/20$ and $P(0\leq X<4)=1/5$.
\item $\mu=3.9$.
\item $\sigma=1.6703$.
\end{enumerate}
}
%RESOLUTION
{}


\newproblem{vad-3}{med}{*}
%STATEMENT
{An experiment consist in injecting a virus to three rats and checking if they survive or not. 
It is known that the probability of surviving is $0.5$ for the first rat, $0.4$ for the second and $0.3$
for the third.
\begin{enumerate}
\item Compute the probability function of the variable $X$ that measures the number of surviving rats.
\item Compute the distribution function.
\item Compute $P(X\leq 1)$, $P(X\geq 2)$ and $P(X=1.5)$.
\item Compute the mean and the standard deviation. 
Is representative the mean?
\end{enumerate}
}
%SOLUTION
{
\begin{enumerate}
\item \[
\begin{array}{|c|c|c|c|c|}
\hline
X & 0 & 1 & 2 & 3 \\
\hline
f(x) & 0.21 & 0.44 & 0.29 & 0.06\\
\hline
\end{array}
\]
\item \[
F(x)=
\begin{cases}
0 & \text{if $x<0$,}\\
0.21 & \text{if $0\leq x<1$,}\\
0.65 & \text{if $1\leq x<2$,}\\
0.94 & \text{if $2\leq x<3$,}\\
1 & \text{if $3\leq x$.}
\end{cases}
\]
\item $P(X\leq 1)=0.65$, $P(X\geq 2)=0.35$ y $P(X=1.5)=0$.
\item $\mu=1.2$ ratas, $\sigma^2=0.7$ ratas$^2$ y $\sigma=0.84$ ratas.
\end{enumerate}
}
%RESOLUTION
{}


\newproblem*{vad-4}{gen}{}
%STATEMENT
{Una tómbola asegura que en cada 1000 boletos hay 500 con ``siga intentándolo'', 100 con un premio de 1\euro, 60 con un premio de 2\euro, 30 con un premio de 3\euro, y 10 con un premio de 10\euro. Un individuo decide comprar un boleto que cuesta 1\euro. Se pide:

\begin{enumerate}
\item Construir una variable aleatoria que mida la ganancia (o pérdida) con la compra de un boleto.
\item ¿Cual es la probabilidad de que pierda dinero?
\item ¿Qué ganancia espera obtener?
\end{enumerate}
}
%SOLUTION
{}
%RESOLUTION
{}


\newproblem{vad-5}{med}{}
%STATEMENT
{The chance of being cured with certain treatment is $0.85$. 
If we apply the treatment to 6 patients,
\begin{enumerate}
\item What is the probability that half of them get cured?
\item What is the probability that a least 4 of them get cured?
\end{enumerate}
}
%SOLUTION
{Let $X$ be the number of cured patients in the sample of 6 treated patients, we have $X\sim B(6,0.85)$.
\begin{enumerate}
\item $P(X=3)=0.041$.
\item $P(X\geq 4)=0.9526$.
\end{enumerate}
}
%RESOLUTION
{}


\newproblem*{vad-6}{gen}{*}
%STATEMENT
{En una mesa de juego, en 1654, Meré propuso a Pascal la siguiente afirmación: ``es más probable obtener al menos un as con cuatro dados, que al menos un doble as en veinticuatro tiradas de dos dados''.

Demostrar que Meré tenía razón.
}
%SOLUTION
{}
%RESOLUTION
{}


\newproblem{vad-7}{med}{}
%STATEMENT
{It is known that the probability having a bacteria in one mm$^3$ of a dissolution is $0.002$.
Assuming that in one mm$^3$ can not be more than one bacteria, compute the probability of having 5 bacteria at most in one cm$^3$ of the dissolution.}
%SOLUTION
{Let $X$ be the number of bacteria in one cm$^3$ of dissolution, we have that $X\sim B(1000,0.002)\approx P(2)$.\\
$P(X\leq 5)=0.9834$.
}
%RESOLUTION
{}


\newproblem{vad-8}{med}{}
%STATEMENT
{Ten persons came into contact with a person infected with tuberculosis. 
The probability of being infected after contacting a person with tuberculosis is $0.10$.
\begin{enumerate}
\item What is the probability that nobody is infected?
\item What is the probability that at least 2 persons are infected?
\item What is the expected number of infected persons? 
\end{enumerate}
}
%SOLUTION
{Let $X$ be the number of persons infected with tuberculosis, we have $X\sim B(10,0.1)$. 
\begin{enumerate}
\item $P(X=0)=0.3487$. 
\item $P(X\geq 2)=0.2639$.
\item $\mu=1$.
\end{enumerate}
}
%RESOLUTION
{}


\newproblem*{vad-9}{gen}{*}
%STATEMENT
{Las matrículas de los coches constan de una parte numérica formada por cuatro cifras, y una parte literal. Se pide:

\begin{enumerate}
\item  Hallar la probabilidad de que al pasar 30 coches haya menos de dos cuya parte numérica sea capicúa.
\item  ¿Cuántos coches deben pasar para que la probabilidad de que alguno tenga la parte numérica capicúa sea mayor
que 0.1?
\end{enumerate}
}
%SOLUTION
{}
%RESOLUTION
{}


\newproblem{vad-10}{med}{}
%STATEMENT
{The probability of suffering an adverse reaction to a vaccine is $0.001$. 
If 2000 persons are vaccinated, what is the probability of suffering some adverse reaction?
}
%SOLUTION
{Let $X$ be the number of adverse reactions, we have that $X\sim B(2000,0.001)\approx P(2)$, and $P(X\geq 1)=0.8648$.
}
%RESOLUTION
{}


\newproblem{vad-11}{gen}{*}
%STATEMENT
{The average number of calls per minute received by a telephone switchboard is 120. 
\begin{enumerate}
\item What is the probability of receiving less than 4 calls in 2 seconds?
\item What is the probability of receiving at least 3 calls in 3 seconds?
\end{enumerate}
}
%SOLUTION
{
\begin{enumerate}
\item If $X$ is the number of calls in 2 seconds, then $X\sim P(4)$ and $P(X<4)=0.4335$.
\item If $Y$ is the number of calls in 3 seconds, then $Y\sim P(6)$ and $P(Y\geq 3)=0.938$.
\end{enumerate}
}
%RESOLUTION
{}


\newproblem*{vad-12}{nut}{}
%STATEMENT
{En un laboratorio de análisis de alimentos se sabe, de experiencias anteriores, que la probabilidad de que una muestra de café contenga plomo en cantidades superiores a las permitidas por la legislación vigente es $0.2$. Si se reciben $50$ muestras de café, ¿cuál es la probabilidad de rechazar al menos $5$?
¿Cuál será el número medio de muestras que rechazaremos?
}
%SOLUTION
{}
%RESOLUTION
{}


\newproblem{vad-13}{far}{}
%STATEMENT
{The fabrication process of a drug produces 6 defective units by hour on average.
What is the probability of producing less than 3 defective units in one hour?
And what is the probability of producing more than one defective units in half an hour?}
%SOLUTION
{Let $X$ be the number of defective units in one hour, we have that $X\sim P(6)$ and $P(X<3)=0.062$.\\
Let $Y$ be the number of defective units in half an hour, we have that $Y\sim P(3)$ and $P(Y>1)=0.8009$.}
%RESOLUTION
{}


\newproblem{vad-14}{gen}{}
%STATEMENT
{Una mecanógrafa comete, en promedio, una errata cada 2000 caracteres que escribe.
Suponiendo que escribe un folio con treinta líneas y setenta caracteres por línea, ¿cuál es la probabilidad de que
cometa más de un error en dicho folio?}
%SOLUTION
{Llamando $X$ al número de errores en un folio, se tiene que $X\sim B(2100,1/2000)\approx P(1.05)$, y $P(X>1)=0.2826$.
}
%RESOLUTION
{}


\newproblem{vad-15}{gen}{}
%STATEMENT
{A test contains 10 questions with 3 possible options each. 
For every question you get a point if you give the right answer and lose half a point if the answer is wrong. 
A student knows the right answer for 3 of the 10 questions and answers the rest randomly. 
What is the probability of passing the exam?}
%SOLUTION
{Let $X$ be the number of right questions in the 7 questions randomly answered, we have that $X\sim B(7,1/3)$ and $P(X\geq 4)=0.1733$.}
%RESOLUTION
{}


\newproblem*{vad-16}{gen}{*}
%STATEMENT
{Un equipo de fútbol tiene 7 delanteros. Se sabe que por término medio, cada delantero se pierde 5 partidos por lesión en una temporada de 40 partidos. Suponiendo que todos los delanteros tienen la misma probabilidad de lesionarse, se pide:
\begin{enumerate}
\item  ¿Cuál es la probabilidad de que en un partido determinado tenga menos de 5 delanteros en condiciones de jugar?
\item  ¿Cuál es la probabilidad de que a lo largo de la temporada haya más de un partido en que tenga menos de 5  delanteros en condiciones de jugar?
\end{enumerate}
}
%SOLUTION
{}
%RESOLUTION
{}


\newproblem*{vad-17}{amb}{}
%STATEMENT
{Para reforestar un bosque se compran árboles a un vivero en el que el 4\% de los árboles suele morir debido a una enfermedad. Si la repoblación se efectúa por parcelas en las que se ponen 10 árboles, se pide:
\begin{enumerate}
\item Calcular la probabilidad de que no muera ningún árbol en una parcela.
\item Calcular la probabilidad de no mueran más de 2 árboles en una parcela.
\item Si en total se reforestan 3000 parcelas, ¿cuál es la probabilidad de que haya alguna en la que mueran más de dos árboles?
\end{enumerate}
}
%SOLUTION
{}
%RESOLUTION
{}


\newproblem{vad-18}{med}{*}
%STATEMENT
{In a study about a parasite that attacks the kidney of rats it is known that the average number of parasites per
kidney is 3. 
\begin{enumerate}
\item Compute the probability that a rat has more than 3 parasites.
\item Compute the probability of, in a sample of 10 rats, at least 9 are infected. 
\end{enumerate}
}
%SOLUTION
{
\begin{enumerate}
\item If $X$ is the number of parasites in a rat, then $X\sim P(6)$ and $P(X>8)=0.1528$.
\item If $Y$ is the number of infected rats in a sample of 10 rats, then $Y\sim B(10,0.9975)$ and $P(Y\geq
9)=0.9997$.
\end{enumerate}
}
%RESOLUTION
{}


\newproblem{vad-19}{med}{*}
% ENUNCIADO
{It has been observed experimentally that 1 of every 20 trillions of cells exposed to radiation mutates becoming
carcinogenic. 
We know that the human body has approximately 1 trillion of cells by kilogram ot tissue. 
Compute the probability that a 60 kg person exposed to radiation develops cancer.
If the radiation affects 3 persons weighing 60 kg, what is the probability that a least one of them develops cancer? }
% SOLUCIÓN
{Let $X$ be the number of mutations, we have that $X\sim B(60\cdot 10^{12},1/20\cdot 10^{-12})\approx P(3)$ and $P(X>0)=0.9502$.\\ 
Let $Y$ be the number of persons that develops cancer in the group of 3 persons, we have that $Y\sim B(3,0.9502)$ and $P(Y\geq 1)=0.9999$.
}
% RESOLUCIÓN
{}


\newproblem{vad-20}{med}{*}
%STATEMENT
{En un servicio de urgencias de cierto hospital se sabe que, en media, llegan 2 pacientes a la hora.
Calcular:
\begin{enumerate}
\item Si los turnos en urgencias son de 8 horas, ¿cuál será la probabilidad de que en un turno lleguen más de 5 pacientes?
\item Si el servicio de urgencias tiene capacidad para atender adecuadamente como mucho a 4 pacientes a la hora, ¿cuál
es la probabilidad de que a lo largo de un turno de 8 horas el servicio de urgencias se vea desbordado en alguna de las
horas del turno?
\end{enumerate}
}
%SOLUTION
{
\begin{enumerate}
\item Llamando $X$ al número de pacientes en un turno, se tiene que $X\sim P(16)$ y $P(X>5)=0.9986$.
\item Llamando $Y$ al número de horas en el que el servicio se vea desbordado porque lleguen más de 4 pacientes, se
tiene que $Y\sim B(8,0.0527)$ y $P(Y\geq 1)=0.3515$.
\end{enumerate}
}
%RESOLUTION
{}


\newproblem*{vad-21}{amb}{}
%STATEMENT
{En un Parque Nacional se contabilizan 15 linces. Si sabemos, por estudios previos, que mueren en promedio 1 de cada 10 individuos a lo largo de un año (ya sea por accidentes, caza de furtivos o por causas naturales):
\begin{enumerate}
\item ¿Cuál es la probabilidad de que en el Parque Nacional se contabilicen más de 2 muertes de linces en un año?
\item Suponiendo un periodo de 12 años, ¿cuál es la probabilidad de que en el Parque Nacional haya algún año
en el que mueran 2 linces?
\end{enumerate}
}
%SOLUTION
{}
%RESOLUTION
{}


\newproblem{vad-22}{med}{*}
%STATEMENT
{The Turner syndrome is a genetic abnormality of women characterized by having only an $X$ chromosome.
It affects 1 in 2000 women approximately.
Besides, approximately 1 in 10 women with the Turner abnormality, also suffers a narrowing of the aorta.
\begin{enumerate}
\item In a sample of 4000 women, what is the probability of having more than 3 women with the Turner syndrome?
And what is the probability of having some women with a narrowing of the aorta as a consequence of the Turner syndrome?
\item In a sample of 20 women with the Turner syndrome, what is the probability of having less than 3 with a narrowing of the aorta?
\end{enumerate}
}
%SOLUTION
{
\begin{enumerate}
\item If $X$ is the number of women with the Turner syndrome in the sample of 4000 women, then $X\sim
B(4000,1/2000)\approx P(2)$ and $P(X>3)=0.1429$.\\
If $Y$ is the number of wome with a narrowing of the aorta in the sample of 4000 women, then $Y\sim
B(4000,1/20000)\approx P(0.2)$ and $P(Y>0)=0.1813$.
\item If $Z$ is the number of women with a narrowing of the aorta in the sample of 20 women with the Turner syndrome, then $Z\sim B(20,1/10)$ and $P(Z<3)=0.6769$.
\end{enumerate}
}
%RESOLUTION
{}


\newproblem{vad-23}{amb}{}
%STATEMENT
{Por estudios previos se sabe que, en una comarca, hay dos tipos de larvas que parasitan, de forma completamente
independiente, los chopos, y que producen su muerte.
Si la larva de tipo $A$ está parasitando un 15\% de los chopos, y la $B$ un 30\%, y en una zona concreta de la comarca
hay 10 chopos:
\begin{enumerate}
\item ¿Qué probabilidad hay que de estén siendo parasitados por $A$ más de dos?
\item ¿Qué probabilidad hay de que estén libres de $B$ más de 8?
\item ¿Qué probabilidad hay de que más de 1 tenga los dos tipos de larva?
\item ¿Qué probabilidad hay de que más de 3 tengan algún tipo de larva?
\end{enumerate}
}
%SOLUTION
{
\begin{enumerate}
\item $X_A$ es el número de chopos parasitados por larvas del tipo $A$, entonces $X_A\sim B(10,0.15)$ y
$P(X_A>2)=0.1798$.
\item Si $X_{\overline{B}}$ es el número de chopos no parasitados por larvas del tipo $B$, entonces
$X_{\overline{B}}\sim B(10,0.7)$ y $P(X_{\overline{B}}>8)=0.1493$.
\item Si llamamos $X_{A\cap B}$ al número de chopos parasitados por larvas de ambos tipos, entonces $X_{A\cap B}\sim
B(10,0.045)$ y $P(X_{A\cap B}>1)=0.0717$.
\item Si llamamos $X_{A\cup B}$ al número de chopos parasitados por algún tipo de larva, entonces $X_{A\cup B}\sim
B(10,0.405)$ y $P(X_{A\cup B}>3)=0.6302$.
\end{enumerate}
}
%RESOLUTION
{}


\newproblem*{vad-24}{amb}{*}
%STATEMENT
{En un Parque Regional se contabilizan 10 parejas de buitre leonado. Si sabemos que el 70\% de las parejas de esta especie logran que alguna de sus crías sobreviva:
\begin{enumerate}
\item ¿Cuál es la probabilidad de que 8 parejas de buitre leonado del Parque logren que alguna de sus crías sobreviva?
\item ¿Cuál es la probabilidad de que alguna pareja logre que alguna de sus crías sobreviva?
\item Si sabemos que en dicho Parque Regional nidifican, en promedio, 8 parejas al año, ¿cuál es la probabilidad de que en un año concreto nidifiquen más de 6?
\end{enumerate}
}
%SOLUTION
{}
%RESOLUTION
{}


\newproblem*{vad-25}{gen}{}
%STATEMENT
{Al lanzar 100 veces una moneda, ¿cuál es la probabilidad de obtener entre 40 y 60 caras?
}
%SOLUTION
{}
%RESOLUTION
{}


\newproblem{vad-26}{psi}{}
%STATEMENT
{El trastorno de pánico aparece en 1 de cada 75 personas.
¿Cuál es la probabilidad de que en un grupo de 100 personas aparezca alguna con trastorno de pánico?
¿Cuál es el número esperado de personas con trastorno de pánico en ese grupo?
}
%SOLUTION
{Llamando $X$ al número de personas que sufren trastorno del pánico en el grupo de 100 personas, se tiene que $X\sim
B(100,1/75)$ y $P(X\geq 1)=0.7379$.}
%RESOLUTION
{}


\newproblem{vad-27}{med}{}
%STATEMENT
{It is knwon that 2 in 1000 patients are allergic to drug $A$, and 6 in 1000 are allergic to drug $B$.
Besides, 30\% of patients allergic to drug $B$ are also allergic to drug $A$.
If we apply both drugs to a sample of 500 patients,
\begin{enumerate}
\item What is the probability of no patients being allergic to drug $A$?
\item What is the probability of at least 2 patients being allergic to drug $B$?
\item What is the probability of less than 2 patients being allergic to both drugs?
\item What is the probability of some patient being allergic to some of the drugs?
\end{enumerate}
}
%SOLUTION
{
\begin{enumerate}
\item Naming $X_A$ to the number of patients being allergic to drug $A$ in the sample of 500 patients, we have that $X_A\sim B(500,0.002)\approx P(1)$ and $P(X_A=0)=0.3678$.
\item Naming $X_B$ to the number of patients being allergic to drug $B$ in the sample of 500 patients, we have that $X_B\sim B(500,0.006)\approx P(3)$ and $P(X_B\geq 2)=0.8009$.
\item Naming $X_{A\cap B}$ to the number of patients being allergic to both drugs in the sample of 500 patients, we have that $X_A\sim B(500,0.0018)\approx P(0.9)$ and $P(X_{A\cap B}<2)=0.7725$.
\item Naming $X_{A\cup B}$ to the number of patients being allergic to some of the drugs in the sample of 500 patients, we have that $X_A\sim B(500,0.0062)\approx P(3.1)$ and $P(X_{A\cup B}\geq 1)=0.9550$.
\end{enumerate}
}
%RESOLUTION
{}


\newproblem{vad-28}{gen}{}
%STATEMENT
{In a classroom there are 40 students of which 35\% smoke.
If we draw a random sample with replacement of 4 students, what is the probability of having at least a smoker student?
Compute the same probability if the random sampling is without repalacement.}
%SOLUTION
{If $X$ is the number of smoker students in a random sample with replacement of size $4$, then $X\sim
B(4,0.35)$ and $P(X\geq 1)=0.8215$.\\
If the random sampling is without replacement then $P(X\geq 1)= 0.8364$.}
%RESOLUTION
{}


\newproblem{vad-29}{med}{*}
%STATEMENT
{Se sabe que por término medio 2 de cada 10000 niños que nacen son albinos.
\begin{enumerate}
\item Si en una región nacen cada año 22000 niños ¿cuál es la probabilidad de que un año nazcan al menos 4 albinos?
\item ¿Cuál es la probabilidad de que en esa región, en un periodo de 10 años no nazca ningún niño albino?
\end{enumerate}
}
%SOLUTION
{
\begin{enumerate}
\item Llamando $X$ al número de niños albinos que nacen en un año, se tiene que $X\sim B(22000,2/10000)\approx
P(4.4)$ y $P(X\geq 4)=0.6406$.
\item Llamando $Y$ al número de niños albinos que nacen en 10 años, se tiene que $Y\sim B(220000,2/10000)\approx
P(44)$ y $P(Y=0)=0$.
\end{enumerate}
}
%RESOLUTION
{}


\newproblem{vad-30}{med}{*}
%STATEMENT
{Suponiendo una facultad en la que hay un 60\% de chicas y un 40\% de chicos:
\begin{enumerate}
\item  Si un año van 6 alumnos a hacer prácticas en un hospital, ¿qué probabilidad hay de que vayan más chicos que chicas?
\item En un período de 5 años, ¿cuál es la probabilidad de que más de 1 año no haya ido ningún chico?
\end{enumerate}
}
%SOLUTION
{\begin{enumerate}
\item Si $X$ es el número de chicos, $X\sim B(6,0.4)$  y $P(X\geq 4)=0.1792$.
\item Si $Y$ es el número de años que no ha ido ningún chico, $Y \sim B(5,0.0467)$ y $P(Y>1)=0.0199$.
\end{enumerate}
}
%RESOLUTION
{\begin{enumerate}
\item Si consideramos un total de $n=6$ alumnos, con un $60\%$ de chicas y un $40\%$ de chicos, la variable $X$ que es el número de alumnos chicos que van a hacer las prácticas de un total de 6, sigue una distribución binomial de 6 intentos y probabilidad de éxito igual a $0.4$: $X\sim B(6,0.4)$. Como nos piden la probabilidad de que haya más chicos que chicas, eso se consigue si el número de chicos es 4 o más. Por lo tanto, nos piden la probabilidad de que $X$ sea mayor o igual que 4:
\[
P(X \ge 4) = P(X = 4) + P(X = 5) + P(X = 6)
\]
Calculando estas probabilidades con la función de probabilidad de la variable binomial, tenemos
\begin{align*}
P(X = 4) &= \binom{6}{4}\cdot 0.4^4  \cdot (1-0.4)^{6-4}  = 12\cdot 0.4^4\cdot 0.6^2 = 0.1382,\\
P(X = 5) &= \binom{6}{5}\cdot 0.4^5  \cdot (1-0.4)^{6-5}  = 6\cdot 0.4^5 \cdot 0.6 = 0.0369,\\
P(X = 6) &= \binom{6}{6}\cdot 0.4^6  \cdot (1-0.4)^{6-6}  = 1\cdot 0.4^6 \cdot 0.6^0 = 0.0041
\end{align*}
Y sumando los tres resultados obtenidos:
\[
P(X \ge 4) = P(X = 4) + P(X = 5) + P(X = 6)= 0.1382+0.0369+0.0041=0.1792.
\]

\item Si consideramos, en un total de 5 años, la probabilidad de que más de un año no haya ido ningún chico, la variable a tener en cuenta $Y$ será el número de años en el total de 5 sin ningún chico, y de nuevo esta variable sigue una distribución binomial, esta vez con 5 intentos, cuyo éxito viene dado por la probabilidad de que en un año concreto no haya ningún chico: $Y \sim B(5,p)$, donde $p=P(X=0)$.
\[
p = P(X = 0) = \binom{6}{0}\cdot 0.4^0  \cdot (1-0.4)^{6-0}  = 1\cdot 0.4^0\cdot 0.6^6 = 0.0467.
\]
Por lo tanto, $Y \sim B(5,0.0467)$.

Además, nos piden la probabilidad de más de una año en el total de 5; es decir: 
\[
P(Y>1)=1-P(Y\leq 1) = 1-P(Y=0)-P(Y=1).
\]
Aplicando, de nuevo, la fórmula de la función de probabilidad de la binomial, obtenemos:
\begin{align*}
P(Y = 0) &= \binom{5}{0}\cdot 0.0467^0  \cdot (1 - 0.0467)^{5-0} = 1\cdot 0.0467^0\cdot 0.9533^5  = 0.7873,\\
P(Y = 1) &= \binom{5}{1}\cdot 0.0467^1  \cdot (1 - 0.0467)^{5-1} = 5\cdot 0.0467^1\cdot 0.9533^4  = 0.1928.
\end{align*}
Teniendo lo anterior en cuenta, la probabilidad que nos piden vale:
\[
P(Y>1)=1-P(Y=0)-P(Y=1)=1-0.7873-0.1928=0.0199.
\]
\end{enumerate}
}


\newproblem*{vad-31}{med}{*}
%STATEMENT
{La probabilidad de que en un grupo de 5 individuos mayores de 70 años todos padezcan arterioesclerosis cerebral es de $12,5$ por mil.
\begin{enumerate}
\item ¿Cuál es la probabilidad de padecer la enfermedad entre los mayores de 70 años?
\item En un grupo de 1000 personas, ¿cuál es la probabilidad de que padezcan la enfermedad más de 450?
\end{enumerate}
}
%SOLUTION
{}
%RESOLUTION
{}


\newproblem{vad-32}{gen}{*}
%STATEMENT
{¿Cuánto habría que restar a cada pregunta errada en un examen de tipo test de 5 preguntas con cuatro opciones y sólo
una correcta, para que un individuo que responda al azar tenga una puntuación esperada de 0? 
}
%SOLUTION
{$1/3$.}
%RESOLUTION
{Si llamamos $X$ al número de preguntas acertadas, está claro que $X$ sigue una distribución binomial $B(5,1/4)$ ya
que el examen tiene 4 preguntas, y la probabilidad de acertar cualquiera de ellas al azar es $1/4$ ya que hay cuatro
opciones y sólo una es la correcta.
}


\newproblem{vad-33}{psi}{}
%STATEMENT
{Se sabe que el $6.8$\% las personas presentan a lo largo de su adolescencia un trastorno de hiperactividad, de los
cuales tres cuartas partes son mujeres.
Si en la población hay el mismo número de hombres y mujeres, se pide:
\begin{enumerate}
\item Calcular la probabilidad de que en una muestra de tres hombres, haya alguno que haya tenido hiperactividad en su
adolescencia. 
\item Calcular la probabilidad de que en una muestra de 2 hombres y 2 mujeres, haya alguno que haya tenido
hiperactividad en su adolescencia. 
\end{enumerate}
}
%SOLUTION
{
\begin{enumerate}
\item Si llamamos $X$ al número de hombres que han tenido hiperactividad en su adolescencia en una muestra de 3
hombres, se tiene que $X\sim B(3,0.034)$ y $P(X\geq 1)=0.0986$.
\item Si llamamos $X_H$ al número de hombres que han tenido hiperactividad en su adolescencia en una muestra de 2
hombres y $X_M$ al número de mujeres que han tenido hiperactividad en su adolescencia en una muestra de 2 mujeres,
entonces $X_H\sim B(2,0.034)$ y $X_M(2,0.102)$. Entonces $P(X_H\geq 1\cup X_M\geq 1)=0.2475$.
\end{enumerate}
}
%RESOLUTION
{}


\newproblem{vad-34}{gen}{*}
%STATEMENT
{A un hospital llegan pacientes por la mañana a efectuarse extracciones de sangre. Se ha medido la frecuencia de llegada de los mismos en
intervalos de 15 minutos. La distribución de probabilidad (medida de forma frecuentista) se  muestra en la siguiente tabla:
\[
\begin{array}{c|c|c|c|c|c|c|c|}
  X   &  0  &  1  &  2   &  3   &  4   &  5  &  6   \\
\hline
 P(x) & 0.1 & 0.2 & 0.25 & 0.15 & 0.15 & 0.1 & 0.05 \\
\end{array}
\]
Se pide:
\begin{enumerate}
\item Calcular la probabilidad de que en un intervalo de 15 minutos lleguen 2 o más personas, y probabilidad de que lleguen menos de 8
personas.
\item ¿Cuál es el número medio esperado de personas que llegarán a sacarse sangre cada 15 minutos?
\item Suponiendo que el número de personas que llegan a sacarse sangre en 15 minutos sigue una distribución de Poisson de media la  
calculada en el apartado anterior, ¿cuál es la probabilidad de que llegue alguna en 15 minutos? ¿Y de que llegue alguna en 5 minutos? 
\end{enumerate}
} 
%SOLUTION
{Llamemos $J$ al suceso consitente en que una persona con la lesión sea joven, $C$ al suceso consistente en curarse, y $A$ y $B$ a los sucesos consistentes en aplicar las respectivas técnicas de
rehabilitación:
\begin{enumerate}
\item Llamando $X$ al número de personas que llegan en un intervalo de 15 minutos: $P(X\geq 2)=0.7$ y $P(X<8)=1$.
\item $\mu=2.55$ personas.
\item Suponiendo $X\sim P(2.55)$, $P(X\geq 1)=0.9219$.\\
Llamando $Y$ al número de personas que llegan en un intervalo de 5 minutos, $P(Y\geq 1)=0.5726$.
\end{enumerate}
}
%RESOLUTION
{\begin{enumerate}
\item Teniendo en cuenta que nos dan la distribución de probabilidad de la variable aleatoria discreta $X$ que expresa el número de
pacientes que llegan en 15 minutos, nos están pidiendo $P(X\geq 2)$ y $P(X<8)$. Estas probabilidades son:
\begin{align*}
P(X\geq 2) & =1-P(X<2)=1-P(X=0)-P(X=1)=1-0.2-0.3=0.7,\\
P(X<8)& =1-P(X\geq 8)=1.
\end{align*}
\item El número esperado es la media de la variable aleatoria:
\[
\mu =\sum xf(x)=0\cdot 0.1+1\cdot 0.2+2\cdot 0.25+3\cdot 0.15+4\cdot 0.15+5\cdot 0.1+6\cdot 0.05=2.55 \text{ personas}.
\]

\item Suponiendo que $X$ sigue una distribución de Poisson con $\lambda =2.55$, $X\sim P(2.55),$ la probabilidad de que llegue alguna
persona en 15 minutos vendrá dada por:
\[
P(X\geq 1)=1-P(X=0)=1-e^{-2.55}\dfrac{2.55^{0}}{0!}=0.9219.
\]

Para la segunda pregunta, teniendo en cuenta que el número medio de personas que llegan cada 15 minutos es $2.55$, en 5 minutos llegarán en
media $2.55/3 = 0.85$ personas, y, por tanto, tenemos una nueva variable aleatoria $Y$, que seguirá una distribución de Poisson con $\lambda
^{\prime }=0.85.$
\[
P(Y\geq 1)=1-P(Y=0)=1-e^{-0.85}\dfrac{0.85^{0}}{0!}=0.5726.
\]
\end{enumerate}
}


\newproblem{vad-35}{gen}{*}
%STATEMENT
{En las siguientes tablas, indicar razonadamente, en los caso que sea posible,
los valores de $h$ que deben ponerse en cada tabla para que se tenga una
distribución de probabilidad:
\[
\begin{array}{c|c}
x & f(x) \\
\hline
-2 & 0.3  \\
5 & h  \\
8 & 0.1
\end{array}
\qquad
\begin{array}{c|c}
x & f(x) \\
\hline
 1 & -0.2 \\
 3 & 0.7 \\
 4 & h
\end{array}
 \qquad
\begin{array}{c|c}
x & f(x) \\
\hline
 2 & h \\
 3 & 0.5 \\
 4 & 0.6
\end{array}
\]

En las tablas que constituyan una distribución de probabilidad:
\begin{enumerate}
\item Representar gráficamente la función de distribución.
\item Calcular media y desviación típica.
\item Calcular la mediana.
\item Si a los valores de $x$ se multiplican por una constante $k<0$, ¿cómo se ve afectada la media? ¿Y la desviación típica?
\end{enumerate}
} 
%SOLUTION
{La única tabla que puede ser una distribución de probabilidad es la primera para $h=0.6$.
\begin{enumerate}[start=2]
\item $\mu=3.2$ y $\sigma=3.516$.
\item $Me=5$.
\item $\mu_y=k\mu_x$ y $\sigma_y=|k|\sigma_x$.
\end{enumerate}
}
%RESOLUTION
{La segunda tabla no puede ser una distribución de probabilidad pues $f(1)=-0.2$ y la función de probabilidad no puede tomar valores
negativos. Por otro lado, la tercera tabla tampoco puede ser una distribución de probabilidad ya que la suma de todas las probabilidades
debe ser 1, y para ello debería ser $h=-0.1$, lo cual no es posible al no poder tomar valores negativos. Así pues la única tabla que puede
ser una distribución de probabilidad es la primera, y como la suma de todas las probabilidades tiene que ser 1, $0.3+h+0.1=1$, se deduce que
$h=0.6$. Trabajaremos pues, con la distribución
\[
\begin{array}{r|r}
 x  & f(x) \\
\hline
 -2 & 0.3  \\
 5  & 0.6  \\
 8  & 0.1  \\
\end{array}
\]

\begin{enumerate}
\item La función de distribución se define como $F(x_0)=P(X\leq x_0)$, y por tanto,  mide probabilidades acumuladas. Acumulando las
probabilidades de la tabla anterior tenemos
\[
\begin{array}{r|r|r}
 x  & \multicolumn{1}{c|}{f(x)} & \multicolumn{1}{c}{F(x)}\\
\hline
 -2 & 0.3 & 0.3 \\
 5  & 0.6 & 0.9 \\
 8  & 0.1 & 1 \\
\end{array}
\]
O lo que es lo mismo, expresado como una función a trozos
\[
F(x)=
\left\{%
\begin{array}{ll}
   0, & \hbox{si $x<-2$;} \\
   0.3, & \hbox{si $-2\leq x<5$;} \\
   0.9, & \hbox{si $5\leq x<8$;} \\
   1, & \hbox{si $x\geq 8$.} \\
\end{array}%
\right.
\]

La gráfica de esta función es la siguiente
\begin{center}
\includegraphics[scale=0.3]{grafica1}\hspace*{1cm}
\end{center}

\item Calculamos los estadísticos que nos piden
\begin{align*}
\mu &= \sum x_if(x_i)=-2\cdot0.3+5\cdot 0.6+8\cdot 0.1=3.2,\\
\sigma^2 &= \sum x_i^2f(x_i)-\mu^2=(-2)^2\cdot 0.3+5^2\cdot 0.6+8^2\cdot 0.1-3.2^2=22.6-10.24=12.36,\\
\sigma &= \sqrt{12.36}=3.516.
\end{align*}

\item La mediana es el valor que deja acumulada una probabilidad $0.5$, es decir, $F(med)=0.5$, y mirando en la función de distribución, el
valor donde se consigue acumular esta probabilidad es el 5.

\item Sea $Y=kX$ donde $k<0$. Por las propiedades de las transformaciones lineales de variables aleatorias, tenemos que $\mu_y=k\mu_x$, y
por tanto la media quedará también multiplicada por la constante $k$. Para la desviación típica tenemos que $\sigma_y=|k|\sigma_x$ y la
desviación típica quedará multiplicada por el valor absoluto de $k$.
\end{enumerate}
}


\newproblem{vad-36}{gen}{*}
%STATEMENT
{En una empresa el número de días al año que los empleados están de baja es, por término medio, 5. Suponiendo que un año tiene 240 días
laborables y que cada mes tiene 20, se pide:
\begin{enumerate}
\item Calcular el porcentaje de empleados que no faltarían más de 5 días al año.
\item Calcular la probabilidad de que un empleado falte algún día en un mes.
\item ¿Cual es la probabilidad de que en un año haya más de 2 meses en los que haya faltado alguna vez?
\end{enumerate}
} 
%SOLUTION
{
\begin{enumerate}
\item Llamando $X$ a la variable que mide el número de días de baja al año de cada empleado, $X\sim B(240,5/240)\approx P(5)$ y
$P(X\leq 5)=0.616$.
\item Llamando $Y$ a la variable que mide el número de días de baja al mes de cada empleado, $Y\sim B(20,5/240)$ y $P(Y\geq 1)=0.3437$.
\item Llamando $Z$ a la variable que mide el número de meses al año en que un empleado falta alguna vez, $Z\sim B(12,0.3437)$ y
$P(Z>2)=0.8379$.
\end{enumerate}
}
%RESOLUTION
{
\begin{enumerate}
\item Sea $X$ la variable que mide el número de días de baja al año de cada empleado. Entonces \mbox{$X\sim B(240,5/240)$}, pero como
$n=240>30$ y $p=5/240<0.1$, podemos aproximarla como una distribución Poisson $P(5)$. La probabilidad de que un empleado no falte más de 5
días al año es
\begin{align*}
P(X\leq 5)&= P(X=0)+P(X=1)+P(X=2)+P(X=3)+P(X=4)+P(X=5)= \\
&= e^{-5}\frac{5^0}{0!}+
e^{-5}\frac{5^1}{1!}+e^{-5}\frac{5^2}{2!}+e^{-5}\frac{5^3}{3!}
+e^{-5}\frac{5^4}{4!}+e^{-5}\frac{5^5}{5!}= \\
&= 0.0067+0.0337+0.0842+0.1404+0.1755+0.1755=0.616,
\end{align*}
es decir, un $61.6\%$.

\item Sea $Y$ la variable que mide el número de días de baja al mes de cada empleado. Entonces \mbox{$Y\sim B(20,5/240)$}, y la
probabilidad de que algún empleado falte algún día en un mes es
\begin{align*}
P(Y\geq1)&=1-P(Y<1)=1-P(Y=0)=
1-\binom{20}{0}\left(\frac{5}{240}\right)^0\left(1-\frac{5}{240}\right)^{20}= \\
&=1-\left(\frac{235}{240}\right)^{20}=0.3437.
\end{align*}

\item Sea ahora $Z$ la variable que mide el número de meses al año en que un empleado falta alguna vez. Entonces, como la probabilidad de
que un empleado falte alguna vez en un mes, según el apartado anterior es $0.3437$, tenemos que $Z\sim B(12,0.3437)$. Así pues, la
probabilidad que nos piden  es
\begin{align*}
P(Z>2)&=1-P(Z\leq 2)=1-P(Z=0)-P(Z=1)-P(Z=2)=\\
&= 1-\binom{12}{0}0.3437^0 0.6563^{12}-\binom{12}{1}0.3437^1 0.6563^{11}-
\binom{12}{2}0.3437^2 0.6563^{10}=\\
&=1-0,0064-0,0401-0,1156=0.8379.
\end{align*}
\end{enumerate}
}


\newproblem{vad-37}{med}{*}
%STATEMENT
{Sabiendo que la prevalencia de la isquemia cardíaca es del 1\%, y que la aplicación de un test diagnóstico para detectar la isquemia
cardíaca tiene una sensibilidad del 90\%, y una especificidad del 95\%. Calcular:
\begin{enumerate}
\item Los valores predictivos, tanto el positivo como el negativo.
\item La probabilidad de diagnóstico acertado.
\item Si tenemos un grupo de 10 enfermos de isquemia cardíaca, ¿cuál es la probabilidad de que diagnostiquemos la enfermedad a
menos de 8?
\end{enumerate}
} 
%SOLUTION
{
}
%RESOLUTION
{
}


\newproblem{vad-38}{med}{*}
%STATEMENT
{A diagnostic test for a disease returns 1\% of positive outcomes, and the positivie and negative predictive values are $0.95$ and $0.98$ respectively. 
\begin{enumerate}
\item Compute the prevalence of the disease.
\item Compute the sensitivity and the specificity of the test.
\item If the test is applied to 12 sick persons, what is the probability of getting at least a wrong diagnostic?
\item If the test is applied to 12 persons, what is the probability of getting a right diagnostic for all of them?
\end{enumerate}
} 
%SOLUTION
{
\begin{enumerate}
\item $P(D)=0.0293$.
\item Sensitivity $P(+|D)=0.3242$ and specificity $P(-|\overline D)=0.9995$. 
\item Let $X$ be the number of wrong diagnostics in 12 sick individuals, we have that $x\sim B(12,0.6758)$ and $P(X\geq 1)=1$. 
\item Let $Y$ be the number of right diagnostics in 12 individuals, we we have that $Y\sim B(12,0.9797)$ and $P(Y=12)=0.7818$. 
\end{enumerate}
}
%RESOLUTION
{
}


\newproblem{vad-39}{med}{*}
%STATEMENT
{La probabilidad de que en un grupo de 5 individuos mayores de 70 años todos padezcan arterioesclerosis cerebral es de $12.5$ por mil.
\begin{enumerate}
\item ¿Cuál es la probabilidad de padecer la enfermedad entre los mayores de 70 años?
\item En un grupo de 1000 personas, ¿cuál es la probabilidad de que padezcan la enfermedad más de 450?
\end{enumerate}
} 
%SOLUTION
{
}
%RESOLUTION
{
}


\newproblem{vad-40}{med}{*}
%STATEMENT
{Si sabemos, por estudios previos, que las cepas que provocarán la gripe del siguiente otoño-invierno afectarán a un 20\% de la
población:
\begin{enumerate}
\item ¿Cuál es la probabilidad de que en una población de 10000 habitantes queden infectados menos de 1900?
\item Suponiendo que se vacunan los 10000 habitantes y sabiendo, por estudios previos, que la vacuna inmuniza al 98\% de los vacunados,
¿Cuál es la probabilidad de que queden sin inmunizar menos de 180?
\item De nuevo, suponiendo que se han vacunado los 10000 habitantes y teniendo en cuenta que, por estudios previos, la vacuna produce
reacciones alérgicas en uno de cada 5000 casos, ¿cuál es la probabilidad de que se produzca alguna reacción alérgica en dicha población?
\end{enumerate}
} 
%SOLUTION
{
}
%RESOLUTION
{
}

