% Author Alfredo Sánchez Alberca (asalber@ceu.es)

\section{Probability}
\begin{enumerate}[leftmargin=*,resume]
\item Construct the sample space of the following random experiments:
\begin{enumerate}
\item Pick a random person and measure the gender and whether she or he is smoker or not. 
\item Pick a random person and measure the blood type and whether she or he is smoker or not.
\item Pick a random person and measure the gender, the blood type and whether she or he is smoker or not.
\end{enumerate}

\item There are two boxes with balls of different colors. 
The first box contains 3 white balls and 2 black balls, and the second one contains 2 blue balls, 1 red ball and 1 green
ball. 
Construct the sample space of the following random experiments:
\begin{enumerate}
\item Pick a random ball from every box. 
\item Pick two random balls from every box.
\end{enumerate}

\item The Morgan's laws state that given two events $A$ and $B$ from the same sample space, $\overline{A\cup B}=\bar A
\cap \bar B$ and $\overline{A\cap B}=\bar A \cup \bar B$.
Proof both statements graphically using Venn diagrams. 

\item In a laboratory there are 4 flasks with sulfuric acid and 2 with nitric acid, and in another laboratory there are
1 flask with sulfuric acid and 3 with nitric acid. 
A random experiment consist in picking two flask, one from every laboratory. 
Calculate the probability of the following events:

\begin{enumerate}
\item The two picked flasks are of sulfuric acid.
\item The two picked flasks are of nitric acid.
\item The tow picked flasks contains different acids.
\end{enumerate}
Calculate the same probabilities if the flask picked in the first laboratory is put in the second laboratory before
picking the flask from it. 

\item Let $A$ and $B$ be events of the same sample space, such that $P(A)=3/8$, $P(B)=1/2$, $P(A\cap B)=1/4$.
Calculate the following probabilities:
\begin{enumerate}
\item  $P(A\cup B)$.
\item  $P(\bar A)$ y $P(\bar B)$.
\item  $P(\bar A\cap \bar B)$.
\item  $P(A\cap \bar B)$.
\item  $P(A|B)$.
\item  $P(A|\bar B)$.
\end{enumerate}

\item In a hospital the probability of getting hepatitis in a blood transfusion from a unit of blood is $0.01$.
A patient gets two units of blood while staying at the hospital.
What is the probability of getting hepatitis?

\item Let $A$ and $B$ be two events of the same sample space, such that $P(A)=0.6$ and $P(A\cup B)=0.9.$
Calculate $P(B)$ with the following assumptions:
\begin{enumerate}
\item $A$ and $B$ are incompatible.
\item $A$ and $B$ are independent.
\end{enumerate}

\item A study about smoking has published that 40\% of smokers have a smoker father, 25\% have a smoker mother and 52\%
have al least one of the parents smoker.
We pick a random person from this population.
Answer the following questions: 
\begin{enumerate}
\item What is the probability of having a smoker mother if the father smokes?
\item What is the probability of having a smoker mother if the father doesn't smoke?
\item Are independent the events having a smoker father and having a smoker mother?
\end{enumerate}

\item The probability that an injury $A$ is repeated is $4/5$, the probability that another injury $B$ is repeated is
$1/2$, and the probability that both injuries are repeated is $1/3$.
Calculate the probability of the following events:
\begin{enumerate}
\item Only injury $B$ is repeated.
\item At least one injury is repeated.
\item Injury $B$ is repeated if injury $A$ has been repeated.
\item Injury $B$ is repeated if injury $A$ hasn't been repeated. 
\end{enumerate}

\item 
We know, from a research study, that 10\% of people over 50 years suffer a particular type or arthritis.
We have developed a new method to detect the disease and after clinical trials we observe that if we apply the method to
people with arthritis we get a positive result in 85\% of cases, while if we apply the method to people without
arthritis, we get a positive result in 4\% of cases.
Answer the following questions:
\begin{enumerate}
\item What is the probability of getting a positive result after applying the method to a random person?
\item If the result of applying the method to one person has been positive, what is the probability of having arthritis?
\end{enumerate}

% 
% \item In a digestive clinical  
% 
%  A partir de una investigación realizada, se sabe que el 10 % de las personas de 50 años sufren un
% tipo particular de artritis. Se ha desarrollado un procedimiento para detectar esta enfermedad, y
% por las pruebas realizadas se observa que si se aplica el procedimiento a individuos que padecen la
% enfermedad, da positivo en el 85 % de los casos, mientras que si se aplica a individuos sanos, da
% positivo en el 4 % de los casos. Se pide:
% a) Calcular la probabilidad de que realizado el procedimiento a una persona, el resultado sea
% positivo.
% b) Si el resultado de aplicar el procedimiento a una persona ha sido positivo, ¿Cuál es la proba-
% bilidad de que padezca la enfermedad?


\end{enumerate}

