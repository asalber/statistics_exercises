% Author Alfredo Sánchez Alberca (asalber@ceu.es)

\newproblem{reg-1}{gen}{}
% ENUNCIADO
{Dada la siguiente tabla de correlación:
\begin{center}
\begin{tabular}{|c||c|c|c|}
\hline
$X\setminus Y$ & 1 & 2 & 3 \\ \hline\hline
$\left[ -2,2\right) $ & 3 & 6 & 1 \\ \hline
$\left[ 2,6\right) $ & 4 & 7 & 3 \\ \hline
$\left[ 6,10\right) $ & 5 & 3 & 0 \\ \hline
\end{tabular}
\end{center}

Determinar:
\begin{enumerate}
\item  Las distribuciones marginales. Media, Moda y Mediana.
\item  Rectas de Regresión.
\item  Coeficiente de correlación lineal. Interpretar el resultado.
\end{enumerate}
}
%SOLUTION
{}
%RESOLUTION
{}


\newproblem{reg-2}{med}{}
%STATEMENT
{Una compañía de asistencia sanitaria hace un estudio del número de veces que, durante el último trimestre, han acudido sus asegurados a consultas de especialistas, en función de su edad. En la siguiente tabla se reflejan los resultados obtenidos:
\begin{center}
\begin{tabular}{|c||c|c|c|c|c|}
\hline
$\mbox{Edad}\setminus \mbox{Cons.}$ & 0 & 1 & 2 & 3 & 4 \\ \hline\hline
$\left[ 30,40\right) $ & 6 & 2 & 2 & 0 & 0  \\ \hline
$\left[ 40,50\right) $ & 4 & 3 & 6 & 4 & 1 \\ \hline
$\left[ 50,60\right) $ & 0 & 2 & 4 & 5 & 3 \\ \hline
$\left[ 60,70\right) $ & 0 & 0 & 3 & 4 & 5 \\ \hline
$\left[ 70,80\right) $ & 0 & 0 & 0 & 4 & 6 \\ \hline
\end{tabular}
\end{center}

Se pide:
\begin{enumerate}
\item Recta de regresión del número de consultas sobre la edad.
\item Coeficiente de correlación e interpretarlo.
\item ¿Cuántas consultas se espera que realice una persona de 52 años?¿Es fiable esta predicción?
\end{enumerate}
}
%SOLUTION
{}
%RESOLUTION
{}


\newproblem{reg-3}{far}{}
%STATEMENT
{A research study has been conducted to determine the loss of activity of a drug.
The table below shows the results of the experiment.

\begin{center}
\begin{tabular}{|l|r|r|r|r|r|}
\hline
Time (in years) & 1 & 2 & 3 & 4 & 5 \\ \hline
Activity (\%) & 96 & 84 & 70 & 58 & 52 \\ \hline
\end{tabular}
\end{center}

\begin{enumerate}
\item  Construct the linear regression model of activity on time.
\item  According to the linear model, when will the activity be 80\%?
When will the drug have lost all activity?
Which prediction is more reliable?
Justify the answer.
\end{enumerate}
}
%SOLUTION
{Naming $T$ the time and $A$ the drug activity:
\begin{enumerate}
\item $\bar t=3$ years, $\bar a=72\%$, $s_t^2=2$ years$^2$, $s_a^2=264\%^2$, $s_{ta}=-22.8$ years$\cdot\%$.\\
Regression line of activity on time: $a=-11.4t+106.2$.
\item Regression line of time on activity: $t=-0.086a+9.2182$.\\
$t(80)=2.3091$ years and $t(0)=9.2182$ years.
\end{enumerate}
}
%RESOLUTION
{}


\newproblem{reg-4}{amb}{}
%STATEMENT
{Las temperaturas medias mensuales (en $^\circ$C) y las precipitaciones totales mensuales (en mm) durante el año 2001 en Madrid fueron:
\begin{center}
\begin{tabular}{|l|r|r|r|r|r|r|r|r|r|r|r|r|}
\cline{2-13}
\multicolumn{1}{c|}{} &    Ene &    Feb &    Mar &    Abr &    May &    Jun &    Jul &    Ago &    Sep &    Oct &    Nov &    Dic \\
\hline
Temp.               &  $7.2$ &  $8.4$ & $12.2$ & $13.7$ & $16.7$ & $23.3$ & $24.2$ & $25.5$ & $20.4$ & $16.2$ &  $8.1$ &  $4.2$ \\
\hline
Prec.                & $73.6$ & $31.7$ & $72.1$ & $20.7$ & $37.1$ & $3.8$ & $3.3$ & $1.5$ & $23.1$ & $67.0$ & $12.4$ & $18.0$ \\
\hline
\end{tabular}
\end{center}
¿Existe relación lineal entre las precipitaciones y la temperatura?
De acuerdo a esta relación, ¿qué cantidad de precipitaciones se espera que haya un mes con una temperatura media de 15$^\circ$C?¿Es fiable esta predicción?
}
%SOLUTION
{}
%RESOLUTION
{}


\newproblem{reg-5}{gen}{}
% ENUNCIADO
{Se ha realizado un estudio comparativo de las puntuaciones obtenidas por los alumnos en un test de ingreso en la
universidad ($X$), y el número de asignaturas aprobadas en el primer curso ($Y$). Los resultados obtenidos se expresan en
la siguiente tabla:

\begin{center}
\begin{tabular}{|c||c|c|c|c|c|}
\hline
$X\setminus Y$ & 0 & 1 & 2 & 3 & 4 \\ \hline\hline
$\left[ 0,10\right) $ & 2 & 2 & 1 & 0 & 0 \\ \hline
$\left[ 10,20\right) $ & 1 & 1 & 2 & 2 & 0 \\ \hline
$\left[ 20,30\right) $ & 0 & 1 & 3 & 4 & 1 \\ \hline
$\left[ 30,40\right) $ & 0 & 0 & 2 & 2 & 6 \\ \hline
\end{tabular}
\end{center}

Se desea calcular:
\begin{enumerate}
\item Recta de regresión de $X$ sobre $Y.$
\item Coeficiente de correlación e interpretación del mismo.
\item Si la universidad en cuestión sólo contara con alumnos que al menos logren aprobar dos asignaturas, ¿qué número
de preguntas respondidas correctamente exigirá en el test?
\end{enumerate}
}
%SOLUTION
{
\begin{enumerate}
\item $\bar x=23$ puntos, $\bar y=2.4$ asignaturas, $s_x^2=116$ puntos$^2$, $s_y^2=1.5733$ asignaturas$^2$,
$s_x=10.7703$ puntos, $s_y=1.2453$ asignaturas y $s_{xy}=9.8$ puntos$\cdot$asignaturas.\\
Recta de regresión de $X$ sobre $Y$: $x=6.2288y+8.0508$.
\item $r=0.73$, lo que quiere decir que hay buena relación lineal entre las puntuaciones y las asignaturas aprobadas y
además es creciente (a mayor puntuación en el test, más asignaturas aprobadas).
\end{enumerate}
}
%RESOLUTION
{}


\newproblem{reg-6}{nut}{*}
%STATEMENT
{The table below shows the two-dimensional frequency distribution of a sample of 80 persons in a study about the
relation between the blood cholesterol ($X$) in mg/dl and the high blood pressure ($Y$) in mmHg.
\[
\begin{array}{|c||c|c|c||c|}
\hline
X\setminus Y & [110,130) & [130,150) & [150,170) & n_x \\
\hline\hline
[170,190)   &           &     4     &           & 12\\
\hline
[190,210)   &    10     &    12     &     4     &   \\
\hline
[210,230)   &     7     &           &     8     &   \\
\hline
[230,250)   &     1     &           &           & 18\\
\hline\hline
n_y          &           &    30     &    24    &    \\
\hline
\end{array}
\]

\begin{enumerate}
\item Complete the table.
\item Construct the linear regression model of cholesterol on pressure.
\item Use the linear model to calculate the expected cholesterol for a person with pressure 160 mmHg.
\item According to the linear model, what is the expected pressure for a person with cholesterol 270 mg/dl?
\end{enumerate}

Use the following sums:
$\sum x_i=16960$ mg/dl, $\sum y_j=11160$ mmHg, $\sum x_i^2=3627200$ (mg/dl)$^2$, $\sum y_j^2=1576800$ mmHg$^2$ y
$\sum x_iy_j=2378800$ mg/dl$\cdot$mmHg.
}
%SOLUTION
{
\begin{enumerate}
\item Frequency table
\[
\begin{array}{|c||c|c|c||c|}
\hline
X\setminus Y & [110,130) & [130,150) & [150,170) & n_x \\
\hline\hline
[170,190)   &     8     &     4     &     0     & 12 \\
\hline
[190,210)   &    10     &    12     &     4     & 26 \\
\hline
[210,230)   &     7     &     9     &     8     & 24 \\
\hline
[230,250)   &     1     &     5     &    12     & 18 \\
\hline\hline
n_y          &   26     &    30     &    24     & 80 \\
\hline
\end{array}
\]
\item $\bar x=212$ mg/dl, $\bar y=139.5$ mmHg, $s_x^2=396$ (mg/dl)$^2$, $s_y^2=249.75$ mmHg$^2$ y $s_{xy}=161$
mg/dl$\cdot$mmHg. Regression line of cholesterol on pressure: $x=122.0721+0.6446y$.
\item  $x(160)=225.2152$ mg/dl.
\item Regression line of pressure on cholesterol: $y=0.4066x+53.3081$.\\
$y(270)=163.0808$ mmHg.
\end{enumerate}
}
%RESOLUTION
{}


\newproblem{reg-7}{nut}{*}
%STATEMENT
{A dietary center is testing a new diet in a sample of 12 persons.
The data below are the number of days of diet and the weight loss (in Kg) until them for every person.
\begin{center}
(33 , 3.9), (51 , 5.9), (30 , 3.2), (55 , 6.0), (38 , 4.9), (62 , 6.2),\\
(35 , 4.5), (60 , 6.1), (44 , 5.6), (69 , 6.2), (47 , 5.8), (40 , 5.3)
\end{center}
\begin{enumerate}
\item Draw the scatter plot. According to the point cloud, what type of regression model explains better the relation
between the weight loss and the days of diet?
\item Construct the linear regression model and the logarithmic regression model of the weight loss on the number of days of diet.
\item Use the best model to predict the weight that will lose a person after 40 and 100 days of diet.
Are these predictions reliable?
\end{enumerate}
Use the following sums ($X$=days of diet and $Y$=weight loss): $\sum x_i=564$ days, $\sum \log(x_i)=45.8086$
$\log(\mbox{days})$, $\sum y_j=63.6$ kg, $\sum x_i^2=28234$ days$^2$, $\sum \log(x_i)^2=175.6603$ $\log(\mbox{days})^2$, $\sum y_j^2=347.7$ kg$^2$, $\sum x_iy_j=3108.5$ days$\cdot$kg, $\sum \log(x_i)y_j=245.4738$ $\log(\mbox{days})\cdot$kg.
}
%SOLUTION
{Naming $Z=\log X$.
\begin{enumerate}[start=2]
\item $\bar x=47$ days, $\bar y=5.3$ kg, $s_x^2=143.833$ days$^2$, $s_y^2=0.885$ kg$^2$, $s_{xy}=9.942$ days$\cdot$kg.\\
Linear model: $y=0.069x+2.051$.\\
$\bar z=3.82$ $\log(\mbox{days})$, $s_z^2=0.07$ $\log^2(\mbox{days})$, $s_{yz}=0.22$ $\log(\mbox{days})\cdot\mbox{kg}$.\\
Logarithmic model: $y=3.4\log y-7.67$.
\item Linear model: $r^2=0.78$, logarithmic model: $r^2=0.86$.\\
Predictions with the logarithmic model: $y(40)=4.86$ kg and $y(100)=7.98$ kg.
The predictions are reliable since the coefficient of determination is high, although the prediction for 100 days is less reliable for being out of the range of observed values in the sample.
\end{enumerate}
}
%RESOLUTION
{}


\newproblem{reg-8a}{far}{*}
%STATEMENT
{In a study about the effect of different doses of a medicament, 2 patients got 2 mg and took 5 days to cure, 4
patients got 2 mg and took 6 days to cure, 2 patients got 3 mg ant took 3 days to cure, 4 patients got 3 mg and took 5
days to cure, 1 patient got 3 mg and took 6 days to cure, 5 patients got 4 mg and took 3 days to cure and 2 patients got
4 mg and took 5 days to cure.
\begin{enumerate}
\item Construct the joint frequency table.
\item Get the marginal frequency distributions and compute the main statistics for each variable.
\item Compute the covariance and interpret it.
\end{enumerate}
}
%SOLUTION
{Let $X$ be the dose and $Y$ to the curation time:
\begin{enumerate}[start=3]
\item $\bar x=3.05$ mg, $\bar y=4.55$ days, $s_x^2=0.648$ mg$^2$, $s_y^2=1.448$ days$^2$, $s_x=0.805$ mg, $s_y=1.203$
days and $s_{xy}=-0.678$ mg$\cdot$days, so there is a decreasing relation.
\end{enumerate}
}
%RESOLUTION
{}


\newproblem{reg-8b}{far}{*}
%STATEMENT
{Al realizar un estudio sobre la dosificación de un cierto medicamento, se trataron 6 pacientes con dosis diarias de 2
mg, 7 pacientes con 3 mg y otros 7 pacientes con 4 mg. De los pacientes tratados con 2 mg, 2 curaron al cabo de 5 días,
y 4 al cabo de 6 días. De los pacientes tratados con 3 mg diarios, 2 curaron al cabo de 3 días, 4 al cabo de 5 días y 1
al cabo de 6 días. Y de los pacientes tratados con 4 mg diarios, 5 curaron al cabo de 3 días y 2 al cabo de 5 días.

Se pide:
\begin{enumerate}
\item Dar el coeficiente de correlación e interpretación.
\item Determinar el tiempo esperado de curación para una dosis de 5 mg diarios.
\end{enumerate}
}
%SOLUTION
{Llamando $X$ a la dosis e $Y$ al tiempo de curación:
\begin{enumerate}
\item $\bar x=3.05$ mg, $\bar y=4.55$ días, $s_x^2=0.648$ mg$^2$, $s_y^2=1.448$ días$^2$, $s_x=0.805$ mg, $s_y=1.203$
días y $s_{xy}=-0.678$ mg$\cdot$días.\\
$r=-0.7$, que quiere decir que hay buena relación lineal entre la dosis y el tiempo de curación, y además es
decreciente (a mayor dosis, menor tiempo de curación).
\item Recta de regresión del tiempo de curación sobre la dosis: $y=-1.046x+7.741$.\\
$y(5)=2.511$ días.
\end{enumerate}
}
%RESOLUTION
{}


\newproblem{reg-9}{nut}{*}
%STATEMENT
{Después de tomar un litro de vino se ha medido la concentración de alcohol en la sangre en distintos instantes,
obteniendo:
\[
\begin{tabular}{|c|c|c|c|c|c|c|}
\hline
Tiempo después (minutos) & 30 & 60 & 90 & 120 & 150 & 180 \\ \hline
Concentración (gramos/litro) & 1.6 & 1.7 & 1.5 & 1.1 & 0.7 & 0.2 \\
\hline
\end{tabular}
\]

Se pide:
\begin{enumerate}
\item Calcular la recta de regresión de la concentración en función del tiempo.
\item ¿Qué concentración de alcohol habrá a los 100 minutos?
\item Si la concentración máxima de alcohol en la sangre que permite la ley para poder conducir es 0.8 g/l, ¿cuánto tiempo habrá que esperar después de tomarse un litro de vino para poder conducir sin infringir la ley?
\end{enumerate}
}
%SOLUTION
{}
%RESOLUTION
{}


\newproblem{reg-10}{gen}{}
%STATEMENT
{For two variables $X$ and $Y$ we have
\begin{itemize}
\item[--] The regression line of $Y$ on $X$ is $y-x-2=0$.
\item[--] The regression line of $X$ on $Y$ is $y-4x+22=0$.
\end{itemize}
Calculate:
\begin{enumerate}
\item The means $\bar x$ and $\bar y$.
\item The correlation coefficient.
\end{enumerate}
}
%SOLUTION
{
\begin{enumerate}
\item $\bar x=8$ and $\bar y=10$.
\item $r=0.5$.
\end{enumerate}
}
%RESOLUTION
{}


\newproblem{reg-11}{gen}{}
%STATEMENT
{The means of two variables $X$ and $Y$ are $\bar x=2$ and $\bar y=1$, and the correlation coefficient is 0.
\begin{enumerate}
\item  Predict the value of $Y$ for $x=10$.
\item  Predict the value of $X$ for $y=5$.
\item  Plot both regression lines.
\end{enumerate}
}
%SOLUTION
{
\begin{enumerate}
\item $y(10)=1$.
\item $x(5)=2$.
\end{enumerate}
}
%RESOLUTION
{}


\newproblem{reg-12}{gen}{*}
%STATEMENT
{En un estudio para relacionar la longitud de la línea de la vida de la mano izquierda y la duración de la vida de una
persona se han obtenido datos de 50 personas con los siguientes resultados ($X$=longitud de la línea en cm, $Y$=edad al
morir en años):
\[
\sum y=3333 \quad \sum y^2=231933 \quad \sum x=459.9 \quad \sum x^2=4308.57 \quad \sum xy=30949.
\]
A la vista de estos resultados, ¿cuanto vivirá, por termino medio, una persona con una línea de longitud 7.5 cm?
¿Es fiable esta estimación?  }
%SOLUTION
{$\bar x=9.198$ cm, $\bar y=66.66$ años, $s_x^2=1.568$ cm$^2$, $s_y^2=195.104$ años$^2$ y $s_{xy}=6.393$
cm$\cdot$años.\\
Recta de regresión de la edad al morir sobre la longitud de la línea de la vida: $y=4.077x+29.158$.\\
$y(7.5)=59.736$ años.\\
$r^2=0.13$, lo que quiere decir que casi no hay relación lineal entre las variables y la predicción anterior no es
fiable.}
%RESOLUTION
{}


\newproblem{reg-13}{gen}{*}
%STATEMENT
{En el estudio de regresión lineal con dos variables $X$ e $Y$ se sabe que $\overline{x}=30$, $\overline{y}=70$ y el
coeficiente de correlación lineal es $0.8$.
También se sabe que para $x=42$ el valor que predice la recta de regresión para $y$ es 78.

Se pide:
\begin{enumerate}
\item Calcular el valor de $x$ que se predice cuando $y=74$.
\item Explicar razonadamente en cuál de las dos variables es más representativa la media.
\end{enumerate}
}
%SOLUTION
{
\begin{enumerate}
\item Recta de regresión de $X$ sobre $Y$: $x=0.96x-37.2$.\\
$x(74)=33.84$.
\item $cv_x=0.0408\sqrt{s_{xy}}>cv_y=0.0146\sqrt{s_{xy}}$ y por tanto es más representativa la media de $Y$ pues tiene
menor dispersión relativa.
\end{enumerate}
}
%RESOLUTIONl
{}


\newproblem{reg-14}{gen}{*}
%STATEMENT
{The values of two variables $S$ and $T$ measured in 10 individuals are
\begin{center}
(-1.5, 2.25), (0.8, 0.64), (-0.2, 0.04), (-0.8, 0.64), (0.4, 0.16),\\
(0.2, 0.04), (-2.1, 4.41), (-0.4, 0.16), (1.5, 2.25), (2.1, 4.41).
\end{center}

\begin{enumerate}
\item Compute the covariance of $S$ and $T$.
\item Can we affirm that $S$ and $T$ are independent?
Justify the answer.
\item Use the regression line to predict the value of $S$ for $t=2$.
\end{enumerate}
}
%SOLUTION
{
\begin{enumerate}
\item $\bar s=0$, $\bar t=1.5$ and $s_{st}=0$.
\item We can affirm that there is no linear relationship between $S$ and $T$, but we can not affirm thar are independent.
\item $s(2)=0$.
\end{enumerate}
}
%RESOLUTION
{}


\newproblem{reg-15}{amb}{}
%STATEMENT
{En un experimento se ha medido el número de bacterias por unidad de volumen en un cultivo, cada hora transcurrida,
obteniendo los siguientes resultados:
\begin{center}
\begin{tabular}{c|ccccccccc}
Horas & 0 & 1 & 2 & 3 & 4 & 5 & 6 & 7 & 8  \\
\hline
Nº Bacterias & 25 & 28 & 47 & 65 & 86 & 121 & 190 & 290 & 362
\end{tabular}
\end{center}

Se pide:
\begin{enumerate}
\item Dibujar el diagrama de dispersión.
Según este diagrama, ¿qué tipo de modelo explicaría mejor la relación entre le número de bacterias y las horas
transcurridas?
\item Dibujar el diagrama de dispersión tomando una escala logarítmica para el número de bacterias.
\item Según el modelo anterior, ¿Cuántas bacterias tendríamos al cabo de 3 horas y media?
¿Y al cabo de 10 horas?
¿Son fiables estas predicciones?
\item ¿Cuánto tiempo tendría que transcurrir para que en el cultivo hubiese 100 bacterias?
\end{enumerate}
}
%SOLUTION
{Llamando $X$ a las horas, $Y$ a las bacterias y $Z$ al logaritmo neperiano de las bacterias:
\begin{enumerate}[start=3]
\item $\bar x=4$ horas, $\bar z=4.5149$ log(bacterias), $s_x^2=6.6667$ horas$^2$, $s_z^2=0.8361$ log$^2$(bacterias) y
$s_{xz}=2.3466$ horas$\cdot$log(bacterias).\\
Modelo lineal del logaritmo de las bacterias sobre las horas: $z=0.3520x+3.1070$.\\
Modelo exponencial de las bacterias sobre las horas: $y=e^{0.3520x+3.1070}$.\\
$y(3.5)=76.6254$ bacterias y $y(10)=755.0986$ bacterias.
\item Modelo lineal de las horas sobre el logaritmo de las bacterias: $x=2.8218z-8.7403$.\\
Modelo logarítmico de las horas sobre las bacterias: $x=2.8218\log y-8.7403$.\\
$x(100)=4.25$ horas.
\end{enumerate}
}
%RESOLUTION
{}


\newproblem{reg-16}{amb}{}
%STATEMENT
{Para evaluar la percepción de los ciudadanos sobre la contaminación atmosférica, se ha realizado un estudio en el que se ha medido en 12 ciudades la concentración media de CO (en mg/m$^3$ diarios), y la percepción mediana en la calidad el aire (en una muestra de individuos de tamaño fijo), medida en la escala MM=Muy Mala, M=Mala, A=Aceptable, B=Buena y MB=Muy Buena.
Los resultados obtenidos fueron:
\begin{center}
($12.8$ , A), ($11.6$ , A), ($9.8$ , B), ($10.3$ , MB), ($15.7$ , MM), ($18.2$ , M),\\
($11.8$ , B), ($16.7$ , M), ($14.5$ , M), ($12.1$ , A), ($19.4$ , MM), ($7.9$ , MB)
\end{center}
¿Existe relación entre la percepción de los habitantes de estas ciudades y la concentración de monóxido de carbono en la atmósfera de las mismas?
}
%SOLUTION
{}
%RESOLUTION
{}


\newproblem{reg-17}{med}{}
%STATEMENT
{En un estudio sobre la influencia del tabaco en los embarazos se ha medido en una muestra de 20 madres el número medio de cigarrillos diarios que fumaban las madres y el peso del recién nacido, obteniendo los siguientes resultados
\begin{center}
\begin{tabular}{|l|c|c|c|c|c|c|c|c|c|c|c|c|c|c|c|}
\hline
Cigarrillos &  2  &  3  & 10  &  8  & 12  &  6  &  6  &  5  &  4  &  9  & 14  &  3  &  7  & 8 &  2  \\
\hline
Peso (kg)  & 3.1 & 3.3 & 2.5 & 3.3 & 2.6 & 3.1 & 3.0 & 3.4 & 3.4 & 2.7 & 2.5 & 3.7 & 3.1 & 3 & 3.6 \\
\hline
\end{tabular}
\end{center}

Se pide:
\begin{enumerate}
\item Construir el modelo de regresión logarítmico del peso sobre el número de cigarrillos.
\item Según este modelo, ¿cuanto pesará el recién nacido si la madre fumaba 15 cigarrillos diários?
Es fiable esta predicción.
\item ¿Es mejor el modelo lineal a la hora de hacer predicciones?
\end{enumerate}
}
%SOLUTION
{}
%RESOLUTION
{}


\newproblem{reg-18}{med}{*}
%STATEMENT
{The table below contains the age in years and the systolic pressure in mmHg of 15 individuals.
\[
\begin{array}{|c|c|c|c|c|c|}
\hline
\text{Age} (x) & 20 & 30 & 40 & 50 & 60 \\
\hline
& 121 & 131 & 132 & 136 & 134\\
\text{Systolic pressure} (y) & 130 & 125 & 129 & 128 & 142 \\
& 125 & 128 & 131 & 134 & 137\\
\hline
\end{array}
\]

\begin{enumerate}
\item What percentage of the systolic pressure variance is explained by the age according to the linear regression model?
\item According to that regression model, what age corresponds to an individual with a systolic pressure of 133 mmHg.
Is reliable this prediction? 
Justify the answer. 
\end{enumerate}
Use the following sums for the computations:
$\sum x_i=600$ years, $\sum y_j=1963$ mmHg, $\sum x_i^2=27000$ years$^2$, $\sum y_j^2=257287$ mmHg$^2$ and $\sum x_iy_j=79400$ years$\cdot$mmHg.
}
%SOLUTION
{
\begin{enumerate}
\item $\bar x=40$ years, $\bar y=130.867$ mmHg, $s_x^2=200$ years$^2$, $s_y^2=26.295$ mmHg$^2$ y $s_{xy}=58.667$
years$\cdot$mmHg.\\
$r^2=0.654$, thus $65.4\%$ of the systolic pressure variance is explained by the age according to the linear regression model.
\item Regression line of age on systolic pressure: $x=2.231y-251.978$.\\
$x(133)=44.745$ years. This prediction is moderately reliable since the determination coefficient is moderately high. 
\end{enumerate}
}
%RESOLUTION
{}


\newproblem{reg-19}{qui}{*}
%STATEMENT
{Se ha realizado un estudio de regresión para ver la relación que existe entre la velocidad de transformación de una determinada sustancia química en una reacción y la temperatura a la que se realiza dicha reacción (manteniendo las cantidades de reactivos constantes).
Según una recta de regresión, a 10 ºC le correspondería una velocidad de 5 gr/min, y a 30 ºC le correspondería una velocidad de 15 gr/min.
Y según la otra recta, a una velocidad de 8 gr/min le correspondería una temperatura de 17 ºC, y a una velocidad de 16 gr/min le correspondería una temperatura de \mbox{31 ºC}. Se pide:
\begin{enumerate}
\item Calcular las ecuaciones de las rectas de regresión.
\item Calcular las medias de ambas variables.
\item Calcular el coeficiente de determinación. ¿Podemos decir que las predicciones del enunciado son fiables? Justificar la respuesta.
\end{enumerate}
}
%SOLUTION
{}
%RESOLUTION
{}


\newproblem{reg-20}{amb}{*}
%STATEMENT
{En un estudio ambiental de una comunidad autónoma se afirma que el número de hectáreas quemadas en los últimos 6 años está relacionado con la cantidad de precipitación media caída en la comunidad, en litros por metro cuadrado. Los datos que han manejado son:
\begin{center}
\begin{tabular}{|l|l|l|}
\hline
\multicolumn{1}{|c|}{Año} & \multicolumn{1}{c|}{Hectáreas quemadas} & \multicolumn{1}{c|}{Precipitación (l/m$^2$)} \\
\hline
\multicolumn{1}{|c|}{2000} & \multicolumn{1}{c|}{1250} & \multicolumn{1}{c|}{420} \\
\hline
\multicolumn{1}{|c|}{2001} & \multicolumn{1}{c|}{1400} & \multicolumn{1}{c|}{380} \\
\hline
\multicolumn{1}{|c|}{2002} & \multicolumn{1}{c|}{850} & \multicolumn{1}{c|}{460} \\
\hline
\multicolumn{1}{|c|}{2003} & \multicolumn{1}{c|}{1650} & \multicolumn{1}{c|}{370} \\
\hline
\multicolumn{1}{|c|}{2004} & \multicolumn{1}{c|}{900} & \multicolumn{1}{c|}{410} \\
\hline
\multicolumn{1}{|c|}{2005} & \multicolumn{1}{c|}{1700} & \multicolumn{1}{c|}{310} \\
\hline
\end{tabular}
\end{center}

\begin{enumerate}
\item Calcular la recta de regresión del número de hectáreas quemadas en función de la precipitación media anual.
\item ¿Es el modelo lineal un buen modelo de ajuste para la nube de puntos? Justificar la respuesta.
\end{enumerate}
}
%SOLUTION
{}
%RESOLUTION
{}


\newproblem{reg-21}{far}{}
%STATEMENT
{Se desea comprobar si el número de ventas de un fármaco depende del descuento que se aplique sobre él.
Para ello se ha medido el número de ventas en farmacias que aplican distintos descuentos obteniendo la siguiente muestra:
\begin{center}
\begin{tabular}{|c||c|c|c|c|c|c|c|c|c|c|c|c|c|c|}
\hline Descuento (\%) & 20 & 16 & 15 & 10 & 12 & 11 & 16 & 8 & 18 & 12 & 12 & 10 & 15 & 14 \\
\hline Ventas & 98 & 46 & 40 & 15 & 21 & 19 & 50 & 8 & 71 & 24 & 21 & 16 & 39 & 32 \\
\hline
\end{tabular}
\end{center}
Se pide:
\begin{enumerate}
\item Construir los modelos exponencial y logarítmico.
\item ¿Cuál de ellos expresa mejor la relación entre el descuento y las ventas?
\item ¿Qué descuento tendremos que aplicar si queremos vender al menos 50 fármacos?
\end{enumerate}
}
%SOLUTION
{}
%RESOLUTION
{}


\newproblem{reg-22}{amb}{*}
% ENUNCIADO
{Para ver si un aditivo para la gasolina mejora la combustión aumentando la emisión de dióxido de carbono, se ha hecho un
estudio en el que se ha medido la cantidad de aditivo añadida a cada litro de gasolina y el porcentaje de CO$_2$ emitido
por un mismo motor, obteniendo la siguiente muestra: \[
\begin{array}{|l|rrrrrrrr|}
\hline
\mbox{Aditivo (cl/l)} &  0.2 &  0.4 &  0.6 &  0.8 &  1.0 &  1.2 &  1.4 &  1.6 \\
\hline
\mbox{CO$_2$ (\%)}         & 11.2 & 12.0 & 12.7 & 13.3 & 13.5 & 13.7 & 13.8 & 13.9 \\
\hline
\end{array}
\]
Se pide:
\begin{enumerate}
\item Calcular el modelo de regresión lineal y logarítmico del CO$_2$ sobre el aditivo.
¿Cuál de los dos modelos es mejor?
\item Según el mejor de los modelos anteriores, ¿cuánto CO$_2$ se producirá para $0.5$ cl de aditivo? ¿y para 2 cl?
¿Son fiables estas predicciones?
\item La normativa sobre emisión de gases exige que el porcentaje mínimo de CO$_2$ en la combustión debe superar al menos el $12.5\%$.
¿Cuánto aditivo es necesario para garantizar esto?
\end{enumerate}
}
%SOLUTION
{}
%RESOLUTION
{}


\newproblem{reg-23}{amb}{*}
%STATEMENT
{La siguiente tabla muestra los datos de emisiones de CO$_2$ y CH$_4$ (en kg/hab) y el producto interior bruto per cápita (en miles US\$) de varios países en el último año:
\[
\begin{array}{|l|r|r|r|}
\hline
\mbox{País} & \mbox{CO}_2 & \mbox{CH}_4 & \mbox{PIB}\\
\hline\hline
\mbox{Austria}     & 7.60 & 0.97 & 38.40\\ \hline
\mbox{España}      & 6.73 & 0.81	& 30.12\\ \hline
\mbox{Francia}     & 5.71 & 0.94	& 33.19\\ \hline
\mbox{EEUU}        &19.40 & 1.72	&	45.84\\ \hline
\mbox{Alemania}    & 9.80 & 0.83	& 34.18\\ \hline
\mbox{Canadá}      &15.60 & 3.08	& 38.43\\ \hline
\mbox{Italia}      & 7.29 & 0.58	& 30.44\\ \hline
\mbox{Japón}       &	9.44 & 0.16	& 33.58\\ \hline
\mbox{Australia}   &17.48 & 6.36	& 36.26\\ \hline
\mbox{Reino Unido} & 8.99 & 0.76	& 35.13\\ \hline
\end{array}
\qquad
\begin{array}{|l|r|r|r|}
\hline
\mbox{País} & \mbox{CO}_2 & \mbox{CH}_4 & \mbox{PIB}\\
\hline\hline
\mbox{Bolivia}     & 1.05 & 3.44	& 40.13\\ \hline
\mbox{Niger}       &	0.1	 & 0.12	&	 0.67\\ \hline
\mbox{Senegal}     &	0.35 & 0.76 &  1.69\\ \hline
\mbox{Pakistán}    & 0.65 & 0.59	&  2.59\\ \hline
\mbox{Filipinas}   &	0.83 & 0.46	&  3.38\\ \hline
\mbox{Perú}        & 0.94 & 0.75	&  7.80\\ \hline
\mbox{Túnez}      & 2.17 & 0.48	&  7.47\\ \hline
\mbox{Nepal}       & 0.13 & 0.90	&  1.21\\ \hline
\mbox{Nicaragua}   & 0.7	 & 0.32	&  2.62\\ \hline
\mbox{Mauritania}  & 0.97 & 0.85	&  2.01\\ \hline
\end{array}
\]
Utilizando los datos sin agrupar, calcular el modelo de regresión logarítmico que explique las emisiones de CO$_2$ en función del PIB y utilizarlo para predecir las emisiones de un país con 10 mil US\$ de PIB.
¿Es fiable la predicción?
}
%SOLUTION
{}
%RESOLUTION
{}


\newproblem{reg-24}{med}{*}
%STATEMENT
{A blood bank keeps plasma at a temperature of 0ºF.
When it is required for a blood transfusion, it is heated in an oven at a constant temperature of 120ºF.
In an experiment it has been measured the temperature of plasma at different times during the heating.
The results are in the table below.
\begin{center}
\begin{tabular}{lrrrrrrrr}
\toprule
Time (min)	& 5 & 8 & 15 & 25 & 30 & 37 & 45 & 60\\
Temperature (ºF) & 25 & 50 & 86 & 102 & 110 & 114 & 118 & 120\\
\bottomrule
\end{tabular}
\end{center}
\begin{enumerate}
\item Plot the scatter plot.
Which type of regression model do you think explains better relationship between temperature and time?
\item Which transformation should we apply to the variables to have a linear relationship?
\item Compute the logarithmic regression of the temperature on time.
\item According to the logarithmic model, what will the temperature of the plasma be after 15 minutes of heating?
Is this prediction reliable? Justify your answer.
\end{enumerate}

Use the following sums ($X$=Time and $Y$=Temperature): $\sum x_i=225$ min, $\sum \log(x_i)=24.5289$ $\log(\mbox{min})$, $\sum y_j=725$ ºF, $\sum \log(y_j)=35.2051$ $\log(\mbox{ºF})$, $\sum x_i^2=8833$ min$^2$, $\sum \log(x_i)^2=80.4703$ $\log(\mbox{min})^2$, $\sum y_j^2=74345$ ºF$^2$, $\sum \log(y_j)^2=157.1023$ $\log(\mbox{ºF})^2$, $\sum x_iy_j=24393$ min$\cdot$ºF, $\sum x_i\log(y_j)=1048.0142$ min$\cdot \log(\mbox{ºF})$, $\sum \log(x_i)y_j=2431.7096$ $\log(\mbox{min})$ºF, $\sum \log(x_i)\log(y_j)=111.1165$ $\log(\mbox{min})\log(\mbox{ºF})$.
}
%SOLUTION
{
\begin{enumerate}
\item A logarithmic model.
\item Apply a logarithmic transformation to time, $z=\log(x)$.
\item $\bar z=3.0661$ $\log(\mbox{min})$, $s_z^2=0.6577$ $\log^2(\mbox{min})$, $\bar y=90.625$ ºF, $s_y^2=1080.2344$ ºF$^2$ and $s_{zy}=26.0969$ $\log(\mbox{min})$ºF.\\
Logarithmic model of temperature on time: $y=-31.0325+39.6781\log(x)$.
\item $y(15)=76.4176$ ºF. $r^2=0.9586$, that is close to 1, so the prediction is reliable.
\end{enumerate}
}
%RESOLUTION
{}


\newproblem{reg-25}{far}{*}
%STATEMENT
{The concentration of a drug in blood, in mg/dl, depends on time, in hours, according to the data below.
\begin{center}
\begin{tabular}{lrrrrrrr}
\toprule
Time & 2 & 3 & 4 & 5 & 6 & 7 & 8\\
Drug concentration & 25 & 36 & 48 & 64 & 86 & 114 & 168\\
\bottomrule
\end{tabular}
\end{center}
\begin{enumerate}
\item Construct the linear regression model of drug concentration on time.
\item Construct the exponential regression model of drug concentration on time.
\item Use the best regression model to predict the drug concentration after $4.8$ hours? Is this prediction reliable?
\end{enumerate}

Use the following sums ($C$=Drug concentration and $T$=time): $\sum t_i=35$ h, $\sum \log(t_i)=10.6046$
$\log(\mbox{h})$, $\sum c_j=541$ mg/dl, $\sum \log(c_j)= 29.147$ $\log(\mbox{mg/dl})$, $\sum t_i^2=203$ h$^2$,
$\sum \log(t_i)^2=17.5206$ $\log(\mbox{h})^2$, $\sum c_j^2=56937$ (mg/dl)$^2$, $\sum \log(c_j)^2=124.0131$
$\log(\mbox{mg/dl})^2$, $\sum t_ic_j=3328$ h$\cdot$mg/dl, $\sum t_i\log(c_j)=154.3387$
h$\cdot\log(\mbox{mg/dl})$, $\sum \log(t_i)c_j=951.6961$ $\log(\mbox{h})\cdot$mg/dl, $\sum
\log(t_i)\log(c_j)=46.08046$ $\log(\mbox{h})\cdot\log(\mbox{mg/dl})$.
}
%SOLUTION
{Naming $Z=\log(C)$.
\begin{enumerate}
\item $\bar t=5$ hours, $\bar c=77.2857$ mg/dl, $s_t^2=4$ hours$^2$,  $s_c^2=2160.7755$ (mg/dl)$^2$, $s_{tc}=89$ hours(mg/dl).\\
Linear model of $C$ on $T$: $c=-33.9643+22.25t$.\\
$r^2=0.9165$.
\item  $\bar z=4.1639$ $\log$(mg/dl), $s_z^2=0.3785$ $\log^2$(mg/dl),
$s_{tz}=1.2291$ hours$\cdot\log$(mg/dl).\\
Exponential model of $C$ on $T$: $c=e^{0.3073t+2.6275}$.\\
$r^2=0.9979$.
\item $c(4.8)= 60.498$ mg/dl and is quite reliable since the coefficient of determination is close to 1.
\end{enumerate}
}
%RESOLUTION
{En el primer apartado de este problema debemos trabajar con el modelo exponencial de la concentración en función del
tiempo, por lo que vamos a tener que calcular la recta de regresión de $z=\ln C$ en función de $t$. Además, en el
segundo apartado debemos trabajar con el modelo lineal de $t$ en función de $C$. Por lo tanto, la tabla con los
sumatorios precisos es:
\[
\begin{array}{|l|r|r|r|r|r|r|r|}
\hline
t_i & c_i & t_i^2 & c_i^2 & t_i \cdot c_i & z_i=\ln_i & z_i^2 & t_i \cdot z_i \\
\hline
2 & 25 & 4 & 625 & 50 & 3.219 & 10.362 & 6.438 \\
\hline
3 & 36 & 9 & 1296 & 108 & 3.584 & 12.845 & 10.752 \\
\hline
4 & 48 & 16 & 2304 & 192 & 3.871 & 14.985 & 15.484 \\
\hline
5 & 64 & 25 & 4096 & 320 & 4.159 & 17.297 & 20.795 \\
\hline
6 & 86 & 36 & 7396 & 516 & 4.454 & 19.838 & 26.724 \\
\hline
7 & 114 & 49 & 12996 & 798 & 4.736 & 22.430 & 33.152 \\
\hline
8 & 168 & 64 & 28224 & 1344 & 5.124 & 26.255 & 40.992 \\
\hline
\sum= 35 & 541 & 203 & 56937 & 3328 & 29.147 & 124.012 & 154.337 \\
\hline
\end{array}
\]

\begin{enumerate}
\item Para el modelo exponencial de la concentración en función del tiempo tenemos en cuenta que:
\[
C = a \cdot e^{bt}  \Leftrightarrow \ln C = \ln \left( {a \cdot e^{bt} } \right) = \ln a + bt
\]
Por lo tanto, si $z=\ln C$, entonces:
\[
z=\ln a +bt
\]
Y el modelo exponencial se transforma en un modelo lineal de $z$ en función de $t$.

Por otra parte, sabemos que la recta de regresión de $z$ en función de $t$ viene dada por:
\[
z-\bar z = \frac{s_{tz}}{s_t^2}(t-\bar t)
\]
Y teniendo en cuenta los sumatorios obtenidos:
\begin{align*}
\bar t &= \frac{\sum t_i}{n} = \frac{35}{7} = 5,\\
\bar z &= \frac{\sum z_i}{n} = \frac{29.147}{7} = 4.164,\\
s_t ^2  &= \frac{\sum t_i^2}{n}-\bar t^2 = \frac{203}{7}-5^2 = 4,\\
s_z ^2  &= \frac{\sum z_i^2}{n}-\bar z^2 = \frac{124.012}{7}-4.164^2 = 0.38,\\
s_{tz}  &= \frac{\sum t_i z_i}{n}-\bar t \cdot \bar z = \frac{154.337}{7}-5 \cdot 4.164 = 1.228.
\end{align*}

Donde la media de $t$ viene dada en horas, su varianza en horas al cuadrado, la media de $z$ no tiene unidades ($z$ es
un logaritmo neperiano), tampoco las tiene su varianza, y la covarianza tiene las unidades de $t$, es decir horas.

Con todo ello, la ecuación de la recta de regresión de $z$ en función $t$ vale:
\[
z-4.164 = \frac{1.228}{4}(t-5)\Leftrightarrow z=2.629+0.307\cdot t
\]

Por lo tanto, teniendo en cuenta que:
\[
z=\ln a +b \cdot t=2.629+0.307 \cdot t
\]
obtenemos fácilmente que $b= 0.307$, y para $a$ despejamos tomando exponenciales:
\[
\ln a= 2.629\Leftrightarrow a=e^{2.629} =13.860.
\]
Con todo ello, cuando $t_0=4.8$ horas, el valor obtenido para $C_0$ (en mg/dl) vale:
\[
C(4.8)=13.860 e^{0.307 \cdot 4.8}=60.498 \text{ mg/dl}.
\]

Para ver si es fiable o no la predicción, calculamos el coeficiente de determinación (o el coeficiente de correlación):
\[
r^2  = \frac{{s_{tz} ^2 }}{{s_t ^2 s_z ^2 }} = \frac{{1.228^2 }}{{4 \cdot 0.377}} = 0.999,
\]
luego, mediante el modelo exponencial estamos explicando un $99.9\%$ de la variabilidad de la nube de puntos, y el
modelo exponencial es muy bueno. Por lo tanto, si el modelo es muy bueno y además la predicción la realizamos en
$t_0=4.8$, que está dentro del rango en el que hemos calculado el modelo, sin duda la predicción también será muy
fiable.

\item Para este nuevo apartado debemos predecir el tiempo que debe transcurrir para que la concentración sea de 100
mg/dl mediante un modelo lineal. Por lo tanto necesitamos la recta de regresión del tiempo en función de la concentración:
\[
t-\bar t = \frac{s_{tC}}{s_C^2}(C-\bar C)
\]
Mediante los sumatorios obtenidos en la tabla del comienzo, calculamos:
\begin{align*}
\bar C &= \frac{\sum C_i}{n} = \frac{541}{7} = 77.286,\\
s_C ^2 &= \frac{\sum z_i ^2}{n}-\bar z^2 = \frac{56937}{7}-77.286^2 = 2160.731,\\
s_{tC} &= \frac{\sum t_i C_i}{n}-\bar t \cdot \bar C = \frac{3328}{7}-5 \cdot 77.286 = 90.000.
\end{align*}
Donde la media de $C$ viene dada en mg/dl, su varianza en (mg/dl)$^2$, y la covarianza en horas$\cdot$(mg/dl).

Sustituyendo todo en la ecuación de la recta obtenemos:
\[
t-5 = \frac{90.000}{2160.731}(C-77.286)\Leftrightarrow t=0.0417 \cdot C+ 1.781
\]

Por lo tanto, si $C_0= 100$ entonces $t_0=5.951$ horas.
Para ver si la predicción es adecuada, de nuevo calculamos el coeficiente de determinación:
\[
r^2 = \frac{s_{tC}^2}{s_t ^2 s_C ^2} = \frac{90.000^2}{4 \cdot 2160.731} = 0.937.
\]
Lo cual nos confirma que el modelo lineal, aunque peor que el exponencial, sigue siendo un muy buen modelo.
Si a eso unimos que estamos realizando la predicción dentro del rango de concentraciones en las que lo hemos calculado,
concluimos que sí que será fiable.
\end{enumerate}
}


\newproblem{reg-26}{fis}{}
%STATEMENT
{The activity of a radioactive substance depends on time according to the data in the table below.
\[
\begin{array}{lrrrrrrrr}
\toprule
t\mbox{ (hours)} & 0 & 10 & 20 & 30 & 40 & 50 & 60 & 70 \\
A\mbox{ ($10^7$ disintegrations/s)} & 25.9 & 8.16 & 2.57 & 0.81 & 0.25 & 0.08 & 0.03 & 0.01\\
\bottomrule
\end{array}
\]

\begin{enumerate}
\item Represent graphically the data of radioactivity as a function of time.
Which type of regression model explains better the relationship between radioactivity and time?
\item Represent graphically the data of radioactivity as a function of time in a semi-logarithmic paper.
\item Compute the regression line of the logarithm of radioactivity on time.
\item Taking into account that radioactivity decay follows the formula
\[
A(t) = A_0 e^{-\lambda t}
\]
where $A_0$ is the number of disintegrations at the beginning and $\lambda$ is a disintegration constant, different for each radioactive substance, use the slope of the previous regression line to compute the disintegration constant for the substance.
\end{enumerate}
Use the following sums ($X$=Time and $Y$=Radioactivity): $\sum x_i=280$ hours, $\sum y_j=37.81$ $10^7$ disintegrations/s, $\sum \log(y_j)=-5.9371$  $\log(10^7 \mbox{ disintegrations/s})$, $\sum x_i^2=14000$ hours$^2$, $\sum y_j^2=744.7265$ $(10^7$ disintegrations/s$)^2$, $\sum \log(y_j)^2=57.7369$  $\log(10^7 \mbox{ disintegrations/s})^2$, $\sum x_iy_j=173.8$ hours$\cdot 10^7$ disintegrations/s, $\sum x_i\log(y_j)=-680.9447$ hours$\cdot \log(10^7 \mbox{disintegrations/s})$.
}
%SOLUTION
{Naming $Z=\log(Y)$.
\begin{enumerate}[start=3]
\item $\bar x=35$ hours, $\bar z=-0.7421$ $10^7$ disintegrations/s, $s_x^2=525$ hours$^2$, $s_z^2=6.6664$ $(10^7$ disintegrations/s$)^2$ and $s_{xz}=-59.1434$ hours$\cdot 10^7$ disintegrations/s.\\
Regression line of the logarithm of radioactivity on time: $z=-0.1127x+3.2008$.
\item $\lambda=0.1127$.
\end{enumerate}
}
%RESOLUTION
{}


\newproblem{reg-27}{qui}{}
%STATEMENT
{For oscillations of small amplitude, the oscillation period $T$ of a pendulum is given by the formula
\[
T = 2\pi\sqrt{\frac{L}{g}}
\]
where $L$ is the length of the pendulum and $g$ is the gravitational constant.
In order to check if the previous formula is satisfied, an experiment has been conducted where it has been measured the oscillation period for different lengths of the pendulum.
The measurements are shown in the table below.

\[
\begin{array}{lrrrrr}
\toprule
L\text{ (cm)} & 52.5 & 68.0 & 99.0 & 116.0 & 146.0 \\
P\text{ (seg)} & 1.449 & 1.639 & 1.999 & 2.153 & 2.408\\
\bottomrule
\end{array}
\]

\begin{enumerate}
\item Represent graphically the data of the period versus the length of the pendulum.
Does a linear model fit well to the points cloud?
\item Represent graphically the data of the period versus the length in a logarithmic paper.
Which type of model fits better to the points cloud?
\item Compute the regression line of the logarithm of period on the logarithm of length.
\item Taking in to account the independent term of the previous regression line, compute the value of $g$.
\end{enumerate}
}
%SOLUTION
{Let $X$ be the logarithm of length and $Y$ to the logarithm of period.
\begin{enumerate}[start=3]
\item $\bar x=4.5025$ $\log$(cm), $\bar y=0.6407$ $\log$(s), $s_x^2=0.1353$ $\log^2$(cm), $s_y^2=0.0339$ $\log^2$(s), $s_{xy}=0.0677$ $\log(\mbox{cm})\log(\mbox{s})$.\\
Regression line of $Y$ on $X$: $y=0.5006x-1.6132$.
\item $g=994.4579$ cm/s$^2$.
\end{enumerate}
}
%RESOLUTION
{}


\newproblem{reg-28}{far}{}
%STATEMENT
{En análisis colorimétrico, es frecuente utilizar la fracción de luz que absorbe una determinada sustancia disuelta
como una medida de la concentración con la que dicha sustancia está presente en la disolución, siempre y cuando se
utilice luz monocromática y la misma longitud recorrida por la luz en cada una de las mediciones.
Si llamamos $I_0$ a la intensidad de luz incidente, $I$ a la intensidad de luz transmitida y $C$ a la concentración de
la sustancia analizada, en un experimento de análisis colorimétrico realizado con Mn y una longitud de onda de 525 nm,
se han obtenido los siguientes datos, donde la concentración de Mn viene dada en mg por cada 100 ml de disolución:
\[
\begin{array}{|l|r|r|r|r|}
\hline
C & 1.00 & 2.00 & 3.00 & 4.00\\
\hline
I/I_0 & 0.418 & 0.149 & 0.058 & 0.026\\
\hline
\end{array}
\]

Se pide:
\begin{enumerate}
\item Representar los datos considerando $I/I_0$ en función de función de $C$.
A la vista de la nube de puntos, ¿qué modelo de regresión sería el más adecuado para expresar la relación entre las
variables?
\item Representar los datos pero en papel semilogarítmico.
\item Calcular la ecuación de la recta de regresión del logaritmo neperiano de $I/I_0$ frente a $C$.
\end{enumerate}
}
%SOLUTION
{Llamando $C$ a la concentración e $Z$ al logaritmo neperiano de $I/I_0$.
\begin{enumerate}[start=3]
\item $\bar c=2.5$ mg/100ml, $\bar z=-2.3183$, $s_c^2=1.25$ (mg/100ml)$^2$, $s_z^2=1.0788$ y $s_{cz}=-1.1595$.\\
Recta de regresión de $Z$ sobre $C$: $z=-0.9276c+0.0007$.
\end{enumerate}
}
%RESOLUTION
{}


\newproblem{reg-29}{psi}{}
%STATEMENT
{Se han recogido por medio de unos cuestionarios los niveles de estrés y energía de 14 mujeres durante un año. A partir
de las respuestas del cuestionario se han asignado puntuaciones a cada una de ellas de manera que a mayor puntuación
mayor grado de estrés y energía. Los datos recogidos son:
\[
\begin{array}{rcccccccccccccc}
\hline
\mbox{Edad}   & 21 & 31 & 19 & 21 & 30 & 20 & 22 & 23 & 45 & 24 & 26 & 19 & 25 & 21\\
\mbox{Estrés} & 25 & 19 & 20 & 19 & 24 &  6 & 29 & 25 & 49 &  0 & 10 & 25 & 13 & 23\\
\mbox{Energía}& 25 & 20 & 45 & 60 & 50 & 50 & 10 & 60 & 40 & 60 & 50 & 60 & 85 & 50\\
\hline
\end{array}
\]

Se pide:
\begin{enumerate}
\item Dibujar un diagrama de dispersión que refleje la relación entre el estrés y la energía.
\item ¿Existe relación lineal entre el estrés y la energía?
¿Y entre el estrés y la edad?
Justificar la respuesta.
\item ¿Qué efecto tendría sobre el coeficiente de correlación lineal de la edad y el estrés la eliminación del individuo
de 45 años? Justificar la respuesta.
\item Calcular el coeficiente de correlación de Spearman entre estrés y energía e interpretarlo.
¿Coinciden las conclusiones con las que se deducen del coeficiente de correlación lineal?
\end{enumerate}
}
%SOLUTION
{Llamando $X$ a la edad, $Y$ al estrés y $Z$ a la energía:
\begin{enumerate}[start=2]
\item $r^2_{yz}=0.14$, lo que indica que casi no hay relación entre el estrés y la energía y $r^2_{xy}=0.31$ lo que
indica que hay una ligera relación entre el estrés y la edad.
\item El coeficiente de correlación lineal disminuye hasta valer casi 0, lo que indica que la relación lineal entre el
estrés y la edad del apartado anterior se debe a este dato atípico, así que, realmente no hay relación entre estrés y
edad.
\item $r_s=-0.41$ lo que indica que hay una ligera relación decreciente entre energía y estrés.
\end{enumerate}
}
%RESOLUTION
{}

\newproblem{reg-30}{psi}{}
%STATEMENT
{Para comprobar el efecto de la herencia genética sobre la inteligencia se desarrolló un estudio en el que se midió el
coeficiente intelectual de varias parejas de gemelos, obteniendo los siguientes resultados:
\[
(128, 132)\ (116, 112)\ (86, 98)\ (65, 81)\ (104,96)\ (111,111)\ (101, 105)\ (72,75)
\]
Calcular el coeficiente de determinación lineal e interpretarlo.
¿Tiene sentido calcular el coeficiente de correlación?
}
%SOLUTION
{Llamando $X$ al coeficiente intelectual del primer hermano e $Y$ al del segundo: $\bar x=97.875$, $\bar y=101.25$,
$s_x^2=418.3594$, $s_y^2=288.4375$, $s_{xy}=326.5313$ y $r^2=0.8836$, lo que indica que existe bastante relación
lineal entre el coeficiente intelectual de los gemelos. No tiene sentido el coeficiente de correlación lineal porque es
indiferente el orden en que tomemos a los gemelos.
}
%RESOLUTION
{}


\newproblem{reg-31}{psi}{}
%STATEMENT
{En un estudio sobre la búsqueda visual se realiza un prueba que consiste en presentarle a un sujeto una matriz de $n$
símbolos y pedirle que pulse rápidamente un botón si entre los símbolos se encuentra uno concreto, u otro botón
diferente si no aparece dicho símbolo.
El tiempo de respuesta de cada participante (en centésimas de segundo) y el número de símbolos de cada matriz aparecen
en la siguiente tabla:
\[
\begin{array}{|l|c|ccccccccc|}
\hline
\mbox{Matrices con} & n & 4 & 5 & 6 & 7 & 8 & 9 & 10 & 11 & 12\\
\cline{2-11}
\mbox{el símbolo} & T & 22 & 24 & 23 & 31 & 33 & 45 & 42 & 46 & 50\\
\hline
\mbox{Matrices sin} & n & 4 & 5 & 6 & 7 & 8 & 9 & 10 & 11 & \\
\cline{2-11}
\mbox{el símbolo} & T & 25 & 24 & 32 & 35 & 43 & 49 & 52 & 56 &\\
\hline
\end{array}
\]

Se pide:
\begin{enumerate}
\item Construir la recta de regresión del tiempo de respuesta sobre el número de símbolos para las matrices con el
símbolo y también para las matrices sin el símbolo.
\item ¿En qué matrices, las que tienen el símbolo o las que no, explica mejor el tiempo de respuesta el número de símbolos?
Justificar la respuesta.
\item Según los modelos anteriores, ¿cuánto tiempo tardará en responder una persona elegida al azar en una matriz de 20
símbolos que contenga al símbolo?
¿Y si no lo contuviese?
\end{enumerate}
}
%SOLUTION
{Llamando $X$ al número de símbolos e $Y$ al tiempo de respuesta:
\begin{enumerate}
\item Matrices con el símbolo: $\bar x=8$ símbolos, $\bar y=35.1111$ seg, $s_x^2=6.6667$ símbolos$^2$, $s_y^2=104.4321$
seg$^2$, $s_{xy}=25.4446$ símbolos$\cdot$seg.\\
Recta de regresión del tiempo sobre el número de símbolos: $y=3.8333x+4.4444$.
Matrices sin el símbolo: $\bar x=7.5$ símbolos, $\bar y=39.5$ seg, $s_x^2=5.25$ símbolos$^2$, $s_y^2=132.25$
seg$^2$, $s_{xy}=26$ símbolos$\cdot$seg.\\
Recta de regresión del tiempo sobre el número de símbolos: $y=4.9525x+2.3571$.
\item $r^2=0.9292$ en las matrices con el símbolo y $r^2=0.9736$ en las matrices sin el símbolo, así que el número de
símbolos explica un poco mejor el tiempo de respuesta en las matrices sin el símbolo.
\item $y(20)=81.11$ seg si la matriz contiene el símbolo y $y(20)=101.4$ seg si la matriz no contiene el símbolo.
\end{enumerate}
}
%RESOLUTION
{}


\newproblem{reg-32}{gen}{}
%STATEMENT
{A study to determine the relation between the age and the physical strength gave the scatter plot below.
\begin{center}
\resizebox{0.7\textwidth}{!}{% Created by tikzDevice version 0.9 on 2016-03-02 18:10:58
% !TEX encoding = UTF-8 Unicode
\begin{tikzpicture}[x=1pt,y=1pt]
\definecolor{fillColor}{RGB}{255,255,255}
\path[use as bounding box,fill=fillColor,fill opacity=0.00] (0,0) rectangle (505.89,361.35);
\begin{scope}
\path[clip] ( 36.00, 36.00) rectangle (493.89,337.35);
\definecolor{fillColor}{RGB}{5,161,230}

\path[fill=fillColor] ( 52.96, 57.31) circle (  2.25);

\path[fill=fillColor] ( 77.90,123.26) circle (  2.25);

\path[fill=fillColor] (115.31,179.07) circle (  2.25);

\path[fill=fillColor] (152.72,229.80) circle (  2.25);

\path[fill=fillColor] (190.13,270.38) circle (  2.25);

\path[fill=fillColor] (215.07,300.82) circle (  2.25);

\path[fill=fillColor] (227.54,305.90) circle (  2.25);

\path[fill=fillColor] (252.48,300.82) circle (  2.25);

\path[fill=fillColor] (277.41,295.75) circle (  2.25);

\path[fill=fillColor] (302.35,280.53) circle (  2.25);

\path[fill=fillColor] (314.82,270.38) circle (  2.25);

\path[fill=fillColor] (352.23,260.24) circle (  2.25);

\path[fill=fillColor] (377.17,250.09) circle (  2.25);

\path[fill=fillColor] (402.11,250.09) circle (  2.25);

\path[fill=fillColor] (439.52,239.94) circle (  2.25);

\path[fill=fillColor] (476.93,229.80) circle (  2.25);
\end{scope}
\begin{scope}
\path[clip] (  0.00,  0.00) rectangle (505.89,361.35);
\definecolor{drawColor}{RGB}{0,0,0}

\node[text=drawColor,anchor=base,inner sep=0pt, outer sep=0pt, scale=  1.20] at (264.94,345.21) {\bfseries Scatter plot of Strengh on Age};

\node[text=drawColor,anchor=base,inner sep=0pt, outer sep=0pt, scale=  1.00] at (264.94,  4.80) {Age};

\node[text=drawColor,rotate= 90.00,anchor=base,inner sep=0pt, outer sep=0pt, scale=  1.00] at ( 12.00,186.67) {Weight lifted (kg)};
\end{scope}
\begin{scope}
\path[clip] (  0.00,  0.00) rectangle (505.89,361.35);
\definecolor{drawColor}{RGB}{0,0,0}

\path[draw=drawColor,line width= 0.4pt,line join=round,line cap=round] ( 52.96, 36.00) -- (489.40, 36.00);

\path[draw=drawColor,line width= 0.4pt,line join=round,line cap=round] ( 52.96, 36.00) -- ( 52.96, 32.99);

\path[draw=drawColor,line width= 0.4pt,line join=round,line cap=round] (115.31, 36.00) -- (115.31, 32.99);

\path[draw=drawColor,line width= 0.4pt,line join=round,line cap=round] (177.66, 36.00) -- (177.66, 32.99);

\path[draw=drawColor,line width= 0.4pt,line join=round,line cap=round] (240.01, 36.00) -- (240.01, 32.99);

\path[draw=drawColor,line width= 0.4pt,line join=round,line cap=round] (302.35, 36.00) -- (302.35, 32.99);

\path[draw=drawColor,line width= 0.4pt,line join=round,line cap=round] (364.70, 36.00) -- (364.70, 32.99);

\path[draw=drawColor,line width= 0.4pt,line join=round,line cap=round] (427.05, 36.00) -- (427.05, 32.99);

\path[draw=drawColor,line width= 0.4pt,line join=round,line cap=round] (489.40, 36.00) -- (489.40, 32.99);

\node[text=drawColor,anchor=base,inner sep=0pt, outer sep=0pt, scale=  0.80] at ( 52.96, 21.60) {10};

\node[text=drawColor,anchor=base,inner sep=0pt, outer sep=0pt, scale=  0.80] at (115.31, 21.60) {15};

\node[text=drawColor,anchor=base,inner sep=0pt, outer sep=0pt, scale=  0.80] at (177.66, 21.60) {20};

\node[text=drawColor,anchor=base,inner sep=0pt, outer sep=0pt, scale=  0.80] at (240.01, 21.60) {25};

\node[text=drawColor,anchor=base,inner sep=0pt, outer sep=0pt, scale=  0.80] at (302.35, 21.60) {30};

\node[text=drawColor,anchor=base,inner sep=0pt, outer sep=0pt, scale=  0.80] at (364.70, 21.60) {35};

\node[text=drawColor,anchor=base,inner sep=0pt, outer sep=0pt, scale=  0.80] at (427.05, 21.60) {40};

\node[text=drawColor,anchor=base,inner sep=0pt, outer sep=0pt, scale=  0.80] at (489.40, 21.60) {45};

\path[draw=drawColor,line width= 0.4pt,line join=round,line cap=round] ( 36.00, 47.16) -- ( 36.00,300.82);

\path[draw=drawColor,line width= 0.4pt,line join=round,line cap=round] ( 36.00, 47.16) -- ( 32.99, 47.16);

\path[draw=drawColor,line width= 0.4pt,line join=round,line cap=round] ( 36.00, 97.89) -- ( 32.99, 97.89);

\path[draw=drawColor,line width= 0.4pt,line join=round,line cap=round] ( 36.00,148.63) -- ( 32.99,148.63);

\path[draw=drawColor,line width= 0.4pt,line join=round,line cap=round] ( 36.00,199.36) -- ( 32.99,199.36);

\path[draw=drawColor,line width= 0.4pt,line join=round,line cap=round] ( 36.00,250.09) -- ( 32.99,250.09);

\path[draw=drawColor,line width= 0.4pt,line join=round,line cap=round] ( 36.00,300.82) -- ( 32.99,300.82);

\node[text=drawColor,anchor=base east,inner sep=0pt, outer sep=0pt, scale=  0.80] at ( 31.20, 44.41) {10};

\node[text=drawColor,anchor=base east,inner sep=0pt, outer sep=0pt, scale=  0.80] at ( 31.20, 95.14) {20};

\node[text=drawColor,anchor=base east,inner sep=0pt, outer sep=0pt, scale=  0.80] at ( 31.20,145.87) {30};

\node[text=drawColor,anchor=base east,inner sep=0pt, outer sep=0pt, scale=  0.80] at ( 31.20,196.60) {40};

\node[text=drawColor,anchor=base east,inner sep=0pt, outer sep=0pt, scale=  0.80] at ( 31.20,247.34) {50};

\node[text=drawColor,anchor=base east,inner sep=0pt, outer sep=0pt, scale=  0.80] at ( 31.20,298.07) {60};

\path[draw=drawColor,line width= 0.4pt,line join=round,line cap=round] ( 36.00, 36.00) --
	(493.89, 36.00) --
	(493.89,337.35) --
	( 36.00,337.35) --
	( 36.00, 36.00);
\end{scope}
\end{tikzpicture}
}
\end{center}

\begin{enumerate}
\item Calculate the linear coefficient of determination for the whole sample.
\item Calculate the linear coefficient of determination for the sample of people younger than 25 years old.
\item Calculate the linear coefficient of determination for the sample of people older than 25 years old.
\item For which age group the relation between age and strength is stronger?
\end{enumerate}

Use the following sums ($X$=Age and $Y=$Weight lifted).
\begin{itemize}[label=--]
\item Whole sample: $\sum x_i=431$ years, $\sum y_j=769$ kg, $\sum x_i^2=13173$ years$^2$, $\sum y_j^2=39675$
kg$^2$ and $\sum x_iy_j=21792$ years$\cdot$kg.
\item Young people: $\sum x_i=123$ years, $\sum y_j=294$ kg, $\sum x_i^2=2339$ years$^2$, $\sum y_j^2=14418$
kg$^2$ and $\sum x_iy_j=5766$ years$\cdot$kg.
\item Old people: $\sum x_i=308$ years, $\sum y_j=475$ kg, $\sum x_i^2=10834$ years$^2$, $\sum y_j^2=25257$
kg$^2$ and $\sum x_iy_j=16026$ years$\cdot$kg.
\end{itemize}
}
%SOLUTION
{
\begin{enumerate}
\item  $\bar{x}=26.9375$ years, $s_x^2=97.6836$ years$^2$, $\bar{y}=48.0625$ kg, $s_y^2=169.6836$ kg$^2$, $s_{xy}=67.3164$ years$\cdot$kg and $r^2=0.2734$.
\item $\bar x=17.5714$ years, $\bar y=42$ kg, $s_x^2=25.3878$ years$^2$, $s_y^2=295.7143$ kg$^2$, $s_{xy}=85.7143$ years$\cdot$kg and $r^2=0.9786$.
\item $\bar x=34.2222$ years, $\bar y=52.7778$ kg, $s_x^2=32.6173$ years$^2$, $s_y^2=20.8395$ kg$^2$,
$s_{xy}=-25.5062$ years$\cdot$kg and $r^2=0.9571$.
\item The linear relation between age and physical strength is stronger in young people.
\end{enumerate}
}
%RESOLUTION
{}


\newproblem{reg-33}{psi}{}
%STATEMENT
{Para evaluar la capacidad de aprendizaje en la realización de una tarea, se ha medido el tiempo que tarda en
realizarse una tarea en sucesivas repeticiones de la misma. Los resultados obtenidos son:
\[
\begin{array}{lcccccccccc}
\hline
\mbox{Repetición} & 1 & 2 & 3 & 4 & 5 & 6 & 7 & 8 & 9 & 10\\
\mbox{Tiempo (min)} & 80 & 65 & 56 & 50 & 48 & 43 & 41 & 38 & 37 & 35 \\
\hline
\end{array}
\]

Se pide:
\begin{enumerate}
\item Dibujar el diagrama de dispersión.
\item En vista del diagrama de dispersión, construir el modelo de regresión más adecuado del tiempo en función de las
repeticiones.
\item ¿Qué porcentaje de la variabilidad del tiempo explican las repeticiones?
\item ¿Cuanto tiempo tardará por término medio en la 5 repetición de la tarea?
\end{enumerate}
}
%SOLUTION
{Llamando $X$ a las repeticiones, $Y$ al tiempo y $Z$ al logaritmo neperiano del tiempo, se tien:
\begin{enumerate}[start=2]
\item $\bar x=5.5$ repeticiones, $\bar z= 3.8644$ ln(min), $s_x^2=8.25$ repeticiones$^2$, $s_z^2=0.0637$ ln$^2$(min) y
$s_{xz}=-0.7014$ repeticiones$\cdot$ln(min).\\
Recta de regresión del logaritmo del tiempo sobre las repeticiones: $z=-0.085x+4.3320$.\\
Modelo exponencial del tiempo sobre las repeticiones: $y=e^{-0.085x+4.3320}$.
\item $R^2=0.9364$, es decir, un $93.64\%$.
\item $y(5)=49.74$ min.
\end{enumerate}
}
%RESOLUTION
{}


\newproblem{reg-34}{psi}{}
%STATEMENT
{En un estudio se ha preguntado a un grupo de personas sobre su ideología política $X$ (izquierda, centro o derecha) y
su opinión sobre la subida o bajada de impuestos $Y$, obteniendo la siguiente tabla de frecuencias:
\begin{center}
\begin{tabular}{|l|c|c|c|}
\hline
$X\backslash Y$ & Bajada & Mantenimiento & Subida \\
\hline
Izquierda & 2 & 6 & 8 \\
\hline
Centro & 3 & 4 & 3 \\
\hline
Derecha & 6 & 5 & 3 \\
\hline
\end{tabular}
\end{center}
¿Se puede concluir que existe relación entre la ideología y la opinión sobre la subida o bajada de impuestos?
Justificar la respuesta.
}
%SOLUTION
{$\chi^2=4.4$ y $C=0.49$ lo que indica que existe bastante relación entre las variables.}
%RESOLUTION
{}


\newproblem{reg-35}{psi}{}
%STATEMENT
{Un estudio sobre 100 personas concluye que 26 personas son fumadores y bebedores habituales, 12 son bebedores pero no
fumadores, 18 son fumadores pero no bebedores y 44 no beben ni fuman habitualmente. Según estos datos, ¿podemos decir
que existe relación entre el tabaco y la bebida? Justificar la respuesta.
}
%SOLUTION
{$\chi^2=14.83$ y $C=0.36$ lo que indica que hay una relación moderada entre los hábitos de fumar y beber.}
%RESOLUTION
{}


\newproblem{reg-36}{psi}{}
%STATEMENT
{En un estudio en el que participaron las 8 universidades de una región se ha valorado la excelencia docente e
investigadora, estableciendo los siguientes rankings (de mejor a peor):
\begin{center}
\begin{tabular}{lcccccccc}
Ranking docencia & 3 & 4 & 8 & 5 & 2 & 1 & 6 & 7\\
Ranking investigación & 6 & 5 & 4 & 3 & 7 & 8 & 1 & 2\\
\end{tabular}
\end{center}
¿Se puede decir que existe relación entre la excelencia docente e investigadora? Justificar la respuesta.
}
%SOLUTION
{$r_s=-0.83$, lo que indica una fuerte relación inversa entre la excelencia docente y la excelencia investigadora.}
%RESOLUTION
{}


\newproblem{reg-37}{fis}{*}
% ENUNCIADO
{En un grupo de pacientes se analiza el efecto de una sustancia dopante en el tiempo de respuesta a un determinado estímulo. Para ello, se
suministra en sucesivas dosis, de 0 a 70 mg, la misma cantidad de dopante a todos los miembros del grupo, y se anota el tiempo medio de
respuesta al estímulo, expresado en centésimas de segundo.
\[
\begin{array}{l|r|r|r|r|r|r|r|r}
X \text{ (mg)} & 0 & 10 & 20 & 30 & 40 & 50 & 60 & 70 \\
\hline
Y\ (10^{-2}\text{s}) & 28 & 46 & 62 & 81 & 100 & 132 & 195 & 302 \\
\end{array}
\]

\begin{enumerate}
\item Representar la nube de puntos. A la vista de la representación, ¿crees que el modelo lineal es el que mejor explica el tipo de
relación entre las variables?

\item Calcular la recta de regresión del tiempo en función de la cantidad de dopante, y utilizarla para predecir el tiempo de reacción medio
para una cantidad de dopante de 25 mg.

\item Hacer la misma predicción del apartado anterior con el modelo exponencial. ¿Qué predicción es mejor?

\item Si para el estímulo estudiado los tiempos de reacción superiores a un segundo se consideran peligrosos para la salud, ¿a partir de qué
nivel debería regularse, e incluso prohibirse, la administración de la sustancia dopante?

\end{enumerate}
}
%SOLUTION
{\begin{enumerate}[start=2]
\item Recta de regresión de $Y$ sobre $X$: $y=3.44x-2.25$. $y(25)=83.82$ centésimas de segundo.
\item Recta de regresión de $X$ sobre $Y$: $x=0.25y+5.57$. $x(100)=30.46$ mg.
\end{enumerate}
}
%RESOLUTION
{\begin{enumerate}
\item El diagrama de dispersión de $Y$ sobre $X$ es el siguiente
\[
\includegraphics[scale=0.5]{dispersion}\qquad
\]
A la vista del diagrama se puede decir que el modelo lineal no es el que mejor se ajustaría a la nube de puntos, sino posiblemente el exponencial.

\item La recta de regresión de $Y$ sobre $X$, tiene ecuación
\[
y=\bar y+\frac{s_{xy}}{s_{x}^2}(x-\bar x).
\]
Calculamos primero los estadísticos que necesitamos en la ecuación:
\begin{align*}
\bar x & = \frac{\sum x_{i}}{n}=\frac{0+10+\cdots+70}{8}=\frac{280}{8}=35,  \\
s_{x}^2 & = \frac{\sum x_{i}^2}{n}-\bar x^2 = \frac{0^2+10^2+\cdots+70^2}{8}-35^2=\frac{14000}{8}-35^2=7261.6875,  \\
s_{x} & = \sqrt{7261.6875}=22.91,  \\
\bar y & = \frac{\sum y_{j}}{n}=\frac{28+46+\cdots+302}{8}= \frac{946}{8}=118.25,  \\
s_{y}^2 & = \frac{\sum y_{j}^2}{n}-\bar y^2 = \frac{28^2+16^2+\cdots+302^2}{8}-118.25^2=\frac{169958}{8}-13983.0625=7261.6875,  \\
s_{y} & = \sqrt{7261.6875}=85.22,  \\
s_{xy} & = \frac{\sum x_{i}y_{j}}{N}-\bar x\bar y = \frac{0\cdot 28+10\cdot 46+\cdots +70\cdot 302}{8}-35\cdot 118.25 =\\
& = \frac{47570}{8}-4138.75=1807.5.  \\
\end{align*}
Sustituyendo en la ecuación anterior estos estadísticos calculados obtenemos la recta de regresión de $Y$ sobre $X$.
\[
y=118.25+\frac{1807.5}{525}(x-35)=3.44x-2.25.
\]
Según esta recta, el tiempo de reacción medio para una cantidad dopante de 25 mg sería
\[
y(25)=3.44\cdot 25-2.25=83.82 \textrm{ centésimas de segundo}.
\]

\item Para ver la cantidad dopante que le corresponde 1 segundo de tiempo de reacción, necesitamos utilizar la recta de regresión de $X$ sobre $Y$. La ecuación de esta recta es
\[
x=\bar x+\frac{s_{xy}}{s_{y}^2}(y-\bar y)= 35+\frac{1807.5}{7261.6875}(y-118.25)=0.25y+5.57.
\]
Y ana vez que tenemos la recta de regresión, para estimar la dosis correspondiente a 1 segundo, hacemos la predicción para $y=100$ centésimas de segundo (ya que las unidades de $X$ son centésimas de segundo $10^{-2}$ s). Sustituyendo en la ecuación anterior tenemos
\[ x(100)=0.25\cdot 100+5.57=30.46 \textrm{ mg}. \]
Como la relación entre $X$ e $Y$ es creciente ($s_{xy}>0$), a partir de $30.46$ mg el tiempo de reacción sería superior a un segundo, y por tanto, peligroso para la salud.
\end{enumerate}
}


\newproblem{reg-38}{fis}{*}
% ENUNCIADO
{La artrosis reumatoide es una enfermedad reumática que aparece con frecuencia en las personas mayores. Uno de los índices más utilizados
para ver el grado de actividad de la enfermedad es el RADAI (Rheumatoid Arthritis Disease Activity Index), que mide el grado de actividad en
una escala de 0 (mínima actividad) a 3 (máxima actividad). Para ver de qué manera influye la edad en el grado de actividad de la enfermedad
se ha seleccionado un grupo de personas mayores y se ha medido el índice RADAI en ellos, obteniendo la siguiente tabla de frecuencias:
\begin{center}
\begin{tabular}{|c|c|c|c|c|}
\hline
 RADAI$\backslash$Edad & 40-50 & 50-60 & 60-70 & 70-80 \\
\hline
         0-1          &   8   &   6   &   2   &   1   \\
\hline
         1-2          &   4   &   7   &   5   &   2   \\
\hline
         2-3          &   0   &   2   &   6   &   7   \\
\hline
\end{tabular}
\end{center}
Se pide:
\begin{enumerate}
\item Estudiar si existe relación lineal entre la edad y el RADAI.
\item Calcular la recta de regresión del RADAI sobre la edad. Según la recta, ¿cuánto aumentaría el grado de actividad de la enfermedad por
cada año que pasa?
\item Si se considera que los pacientes don un RADAI de 2 o superior necesitan ayuda en sus actividades diarias, ¿a qué edad se empezaría a
necesitar esta ayuda?
\end{enumerate}
}
%SOLUTION
{Llamando $X$ a la variable que mide la edad e $Y$ a la que mide el RADAI.
\begin{enumerate}
\item $r=0.59$ que indica una relación lineal moderada.
\item Recta de regresión de $Y$ sobre $X$: $y=0.0442x-1.1575$. Cada año que pase la actividad de la enfermedad aumentará $0.0442$ puntos en
el RADAI.
\item A los $63.4$ años.
\end{enumerate}
}
%RESOLUTION
{Llamemos $X$ a la variable que mide la edad e $Y$ a la que mide el RADAI.
\begin{enumerate}
\item Para ver si existe relación lineal entre $X$ e $Y$, podemos calcular el
coeficiente de correlación lineal, pero para ello necesitamos la media y
desviación típica de cada variable y la covarianza. Antes de calcular estos
estadísticos, obtenemos las distribuciones marginales de cada variable a
partir de la tabla:
\[
\begin{array}{|c|c|c|c|c|c|}
\hline
 Y\backslash X & 40-50 & 50-60 & 60-70 & 70-80 & n_y \\
\hline
         0-1          &   8   &   6   &   2   &   1  & 17 \\
\hline
         1-2          &   4   &   7   &   5   &   2  & 18 \\
\hline
         2-3          &   0   &   2   &   6   &   7  & 15 \\
\hline
         n_x        &  12   &  15   &  13   &  10  & 50 \\
\hline
\end{array}
\]

A partir de aquí calculamos los estadísticos anteriores:
\begin{align*}
\overline{x} &=
\frac{\sum_{i}^{}x_{i}n_{i}}{n} = \frac{45\cdot 12+55\cdot 15+65\cdot
13+75\cdot 10}{50} = \frac{2960}{50} = 59.2,  \\
\overline{y} &=
\frac{\sum_{j}^{}y_{j}n_{j}}{n} = \frac{0.5\cdot 17+1.5\cdot 18+2.5\cdot
15}{50} = \frac{73}{50} = 1.46,  \\
s_{x}^{2} &= \frac{\sum_{i}^{}x_{i}^2n_{i}}{n}-\overline{x}^2 =
\frac{45^2\cdot 12+55^2\cdot 15+65^2\cdot
13+75\cdot 10}{50}-59.2^2 = \\
&= \frac{180850}{50}-3504.64 = 112.36,  \\
s_{x} &= \sqrt{112.36} = 10.6,  \\
s_{y}^{2} &= \frac{\sum_{j}^{}y_{j}^2n_{j}}{n}-\overline{y}^2 =
\frac{0.5^2\cdot 17+1.5^2\cdot 18+2.5^2\cdot
15}{50}-1.46^2 =\\
&= \frac{138.5}{50}-2.1316 = 0.6384,  \\
s_{y} &= \sqrt{0.6384} = 0.8,  \\
s_{xy} &=
\frac{\sum_{ij}^{}x_{i}y_{j}n_{ij}}{n}-\overline{x}\overline{y} =
\frac{45\cdot 0.5\cdot 8+55\cdot 1.5\cdot 4+ \cdots +75\cdot 2.5\cdot
7}{50}-59.2\cdot 1.46 =\\
&= \frac{4570}{50}-86.432 = 4.968.
\end{align*}

Con estos datos, el coeficiente de correlación lineal es
\[
r=\frac{s_{xy}}{s_xs_y}=\frac{4.968}{10.6\cdot 0.8}=0.59,
\]
que indica que existe relación lineal aunque no demasiado fuerte, sino más bien
moderada.

\item La recta de regresión de $Y$ sobre $X$ es
\[
 y=\overline{y}+\frac{s_{xy}}{s_{x}^{2}}(x-\overline{x})=1.46+\frac{4.968}{112.36}(x-59.2)=
 0.0442x-1.1575.
\]
El aumento del grado de actividad del RADAI por cada año que pasa nos lo da el
coeficiente de regresión de $Y$ sobre $X$, que es la pendiente de la recta de
regresión que hemos calculado, es decir, 0.0442 por cada año.

\item Para predecir a qué edad se empezaría a necesitar ayuda, necesitamos
calcular la recta de regresión de $X$ sobre $Y$, que tiene ecuación
\[
 x=\overline{x}+\frac{s_{xy}}{s_{y}^{2}}(y-\overline{y})=5.92+\frac{4.968}{0.6384}(y-1.46)=
 7.782y+47.83.
\]
Sustituyendo $y$ por 2 en esta ecuación tenemos
\[
x(2)=7.782\cdot 2+47.83=63.4 \textrm{ años}.
\]
\end{enumerate}}




\newproblem{reg-39}{fis}{*}
% ENUNCIADO
{A basketball team is testing a new stretching program to reduce the injuries during the league.
The data below show the daily number of minutes doing stretching exercises and the number of injuries along the league.
\begin{center}
\begin{tabular}{lrrrrrrrr}
\toprule
Stretching minutes & 0 & 30 & 10 & 15 & 5 & 25 & 35 & 40\\
Injuries & 4 & 1 & 2 & 2 & 3 & 1 & 0 & 1\\
\bottomrule
\end{tabular}
\end{center}
\begin{enumerate}
\item Construct the regression line of the number of injuries on the time of stretching.
\item How much is the reduction of injuries for every minute of stretching?
\item How many minutes of stretching are require for having no injuries? Is reliable this prediction?
\end{enumerate}

Use the following sums ($X$=Number of minutes stretching, and $Y$=Number of injuries):
$\sum x_i = 160$ min, $\sum y_j=14$ injuries, $\sum x_i^2= 4700$ min$^2$, $\sum y_j^2=36$ injuries$^2$ and $\sum
x_iy_j=160$  min$\cdot$injuries.
}
%SOLUTION
{
\begin{enumerate}
\item Regression line of $Y$ on $X$: $y-0.08x+3.35$.
\item For each minute more of stretching there will be $0.08$ injuries less.
\item To having no injuries at least $38.26$ minutes of stretching are required. $r=-0.91$, and the prediction is quite reliable.
\end{enumerate}
}
%RESOLUTION
{Llamemos $X$ a la variable que mide el tiempo de estiramiento, e $Y$ a la que
mide el número de lesiones en cada jugador.
\begin{enumerate}
\item La recta de regresión de $Y$ sobre $X$, tiene ecuación
\[
y=\overline{y}+\frac{s_{xy}}{s_{x}^2}(x-\overline{x}).
\]
Calculamos primero los estadísticos que necesitamos en la ecuación:
\begin{align*}
\overline{x} & = \frac{\sum x_{i}}{N}=\frac{0+30+\cdots+40}{8}=\frac{160}{8}=20,  \\
s_{x}^2 & = \frac{\sum x_{i}^2}{N}-\overline{x}^2 =
\frac{0^2+30^2+\cdots+40^2}{8}-20^2=\frac{4700}{8}-20^2=187.5,  \\
s_{x} & = \sqrt{187.5}=13.69,  \\
\overline{y} & = \frac{\sum y_{j}}{N}=\frac{4+1+\cdots+1}{8}=
\frac{14}{8}=1.75,  \\
s_{y}^2 & = \frac{\sum y_{j}^2}{N}-\overline{y}^2 =
\frac{4^2+1^2+\cdots+1^2}{8}-1.75^2=\frac{36}{8}-1.75^2=1.4375,  \\
s_{y} & = \sqrt{1.4375}=1.2,  \\
s_{xy} & = \frac{\sum x_{i}y_{j}}{N}-\overline{x}\overline{y} =
\frac{0\cdot 4+30\cdot 1+\cdots +40\cdot 1}{8}
-20\cdot 1.75 =\\
& = \frac{160}{8}-20\cdot 1.75=-15.  \\
\end{align*}
Sustituyendo en la ecuación anterior estos estadísticos calculados obtenemos la
recta de regresión de $Y$ sobre $X$.
\[
y=1.75-\frac{15}{187.5}(x-20)=-0.08x+3.35.
\]

El incremento que experimenta la variable $Y$ por cada unidad que se
incrementa la variable $X$ según la recta de regresión, es su pendiente o
coeficiente de regresión de $Y$ sobre $X$, que en este caso es $-0.08$. Así,
pues, por cada minuto más de estiramiento se espera tener $0.08$ lesiones
menos.

\item Para predecir el número de minutos que debería estirar un jugador que
quiere tener 0 lesiones, debemos calcular antes la recta de regresión de
tiempo de estiramiento sobre número de lesiones. La ecuación de esta recta
es
\[
x=\overline{x}+\frac{s_{xy}}{s_{y}^2}(y-\overline{y})=
20-\frac{15}{1.4375}(y-1.75)=-10.43y+38.26.
\]

Una vez que tenemos la recta de regresión, para estimar el valor de $X$ para $y=0$, basta con sustituir $y$ por $0$ en esta ecuación y
obtenemos \[ x(0)=-10.43\cdot 0+38.26=38.26. \]

Por último, para ver si esta estimación es fiable, calculamos el coeficiente de
correlación lineal
\[
r=\frac{s_{xy}}{s_{x}s_{y}}=\frac{-15}{13.69\cdot 1.2}=-0.91.
\]
Como el coeficiente de correlación lineal está próximo a -1, el modelo lineal es un
buen modelo y por tanto sus predicciones serán fiables.
\end{enumerate}
}


\newproblem{reg-40}{gen}{*}
% ENUNCIADO
{Un profesor está interesado en analizar la relación existente entre la nota que esperan obtener los alumnos en los exámenes de su
asignatura con la que de verdad obtienen una vez corregidos dichos exámenes. La tabla muestra la nota esperada y la obtenida para 10 alumnos
diferentes:
\[
\begin{array}{ccc}
\hline
\text{Alumno} & \text{Nota esperada} & \text{Nota obtenida} \\
\hline \hline
   1    &      3.0      &      5.1      \\
\hline
   2    &      6.0      &      4.8      \\
\hline
   3    &      7.0      &      6.0      \\
\hline
   4    &      8.0      &      4.2      \\
\hline
   5    &      3.0      &      5.2      \\
\hline
   6    &      9.0      &      7.5      \\
\hline
   7    &      2.0      &      3.6      \\
\hline
   8    &      5.0      &      3.0      \\
\hline
   9    &      8.0      &      6.5      \\
\hline
   10   &      2.0      &      0.8      \\
\hline
\end{array}
\]

\begin{enumerate}
\item Calcular la recta de regresión de la nota obtenida en función de la nota esperada.
\item Calcular el coeficiente de correlación lineal e interpretarlo.
\item ¿Cuál es la nota que esperaba obtener un alumno que en realidad saca un 4.0?
\end{enumerate}
}
%SOLUTION
{Llamando $X$ a la nota esperada e $Y$ a la nota real obtenida:
\begin{enumerate}
\item Recta de regresión de $Y$ sobre $X$: $y=0.485\,x+2.0994$.
\item $r=0.6786,$ lo que indica una relación creciente moderada.
\item $x(4)=4.6639$ puntos.
\end{enumerate}
}
%RESOLUTION
{Llamemos $X$ a la nota esperada e $Y$ a la nota real obtenida.
\begin{enumerate}
\item La ecuación de la recta de regresión de $Y$ sobre $X$ es
\[
y=\bar y+\frac{s_{xy}}{s_{x}^2}(x-\bar x).
\]
Calculamos primero los estadísticos que necesitamos en la ecuación:
\begin{align*}
\bar x &= \frac{\sum x_{i}}{n}=\frac{53}{10}=5.3,  \\
s_{x}^2 &= \frac{\sum x_{i}^2}{n}-\bar x^2 = \frac{345}{10}-5.3^2=6.41,  \\
s_{x} &=\sqrt{6.41}=2.5318,\\
\bar y &= \frac{\sum y_{j}}{n}=\frac{46.7}{10}=4.67,  \\
s_{y}^2 &= \frac{\sum y_{j}^2}{n}-\bar y^2 =
\frac{250.83}{10}-4.67^2=3.2741,  \\
s_{y}&=\sqrt{3.2741}=1.8094,\\
s_{xy} &= \frac{\sum x_{i}y_{j}}{n}-\bar x\bar y = \frac{278.6}{10}-5.3\cdot 4.67 =3.109.  \\
\end{align*}
Sustituyendo en la ecuación anterior estos estadísticos calculados obtenemos la recta de regresión de $Y$ sobre $X$.
\[
y=4.67+\frac{3.109}{6.41}(x-5.3)=0.485\,x+2.0994.
\]

\item El coeficiente de correlación lineal es
\[
r=\frac{s_{xy}}{s_{x}s_{y}}=\frac{3.109}{2.5318\cdot 1.8094}=0.6786,
\]
y según este valor, podemos decir que existe una dependencia creciente moderada.

\item Para predecir la nota esperada por un alumno que saca un 4, necesitamos la recta de regresión de $X$ sobre $Y$. La ecuación de esta
recta es
\[
x=\bar x+\frac{s_{xy}}{s_{y}^2}(y-\bar y)= 5.3+\frac{3.109}{3.2741}(y-4.67)=0.9496\,y+0.8655.
\]
Sustituyendo $y$ por 4 en esta ecuación obtenemos la predicción deseada
\[
x(4)=0.9496\cdot 4+0.8655=4.6639.
\]
\end{enumerate}
}


\newproblem{reg-41}{fis}{*}
% ENUNCIADO
{A researcher is studying the relation between the obesity and the response to pain.
The obesity is measured as the percentage over the ideal weight, and the response to pain as the nociceptive flexion pain threshold.
The results of the study appears in the table below.
\[
\begin{array}{lrrrrrrrrrr}
\toprule
\mbox{Obesity} & 89 & 90 & 77 & 30 & 51 & 75 & 62 & 45 & 90 & 20\\
\mbox{Pain threshold} & 10 & 12 & 11.5 & 4.5 & 5.5 & 7 & 9 & 8 & 15 & 3\\
\bottomrule
\end{array}
\]
\begin{enumerate}
\item According to the scatter plot, what model explains better the relation of the response to pain on the obesity, the linear or the logarithmic model?
\item According to the best regression model, what is the response to pain expected for a person with an obesity of 50\%?
Is this prection reliable?
\item According to the best regression model, what is the expected obesity for a person with a pain threshold of
10? Is this prediction reliable?
\end{enumerate}
Use the following sums ($X$=Obesity and $Y$=Pain threshold): $\sum x_i=629$, $\sum \log(x_i)=40.4121$, $\sum y_j=85.5$,
$\sum \log(y_j)=20.4679$, $\sum x_i^2=45445$, $\sum \log(x_i)^2=165.6795$, $\sum y_j^2=854.75$, $\sum\log(y_j)^2=44.08906$, $\sum x_iy_j=6124$, $\sum x_i\log(y_j)=1390.14$, $\sum \log(x_i)y_j=360.0725$, $\sum\log(x_i)\log(y_j)=84.80687$.
}
%SOLUTION
{
\begin{enumerate}[start=2]
\item $\bar{x}=62.9$, $s_x^2=588.09$, $\bar{y}=8.55$, $s_y^2=12.3725$ and $s_{xy}=74.605$.\\
Regression line of response to pain on obesity: $y=0.5705+0.1269x$. $r^2=0.765$.\\
$\overline{\log(x)}=4.0412$, $s_{\log(x)}^2=0.2366$ and $s_{\log(x)y}=1.45491$.
Logarithmic model of response to pain on obesity: $y=-16.303+6.150\log(x)$.
$r^2=0.7232$.\\
Using the linear model: $y(50)=6.9155$.
\item Regression line of obesity on response to pain: $x=11.344+6.030y$.\\
$x(10)=71.644$.
\end{enumerate}
}
%RESOLUTION
{
}


\newproblem{reg-42}{med}{*}
%STATEMENT
{Se realiza un estudio para establecer una ecuación mediante la cual se pueda utilizar la concentración de estrona en saliva para predecir
la concentración del esteroide en plasma libre. Se extrajeron los siguientes datos de 10 varones sanos:
\[
\begin{array}{|lrrrrrrrrrr|}
\hline
\text{Estrona} & 1.4 & 7.5 & 8.5 & 9.0 & 9.0 & 11 & 13 & 14 & 14.5 & 16\\
\text{Esteroide} & 30.0 & 25.0 & 31.5 & 27.5 & 39.5 & 38.0 & 43.0 & 49.0 & 55.0 & 48.5\\
\hline
\end{array}
\]

\begin{enumerate}
\item Comprobar la idoneidad del modelo lineal de regresión. Si el modelo es apropiado, hallar la recta de regresión de la concentración de
estrona en función de la concentración de esteroide.
\item Si un individuo presenta una concentración de estrona en saliva de 10, ¿qué concentración de esteroide en plasma libre
predeciría el modelo de regresión lineal?
\item Para los dos primeros individuos, calcular los errores que se comenten al utilizar el modelo de regresión lineal para
predecir la concentración de estrona. Razonar a que se deben estos errores.
\end{enumerate}
}
%SOLUTION
{
}
%RESOLUTION
{
}


\newproblem{reg-43}{med}{*}
%STATEMENT
{En una análisis de niños sanos se deseaba establecer si existía relación lineal entre la edad (en años) del niño y el ángulo de Clarke (en
grados), obteniéndose en una muestra de 7 niños los valores que aparecen a continuación:
\[
\begin{array}{|l|r|r|r|r|r|r|r|}
\hline
\text{Edad} & 3 & 4 & 5 & 6 & 7 & 8 & 9 \\
\hline
\text{Ángulo de Clarke} & 24 & 26 & 30 & 31 & 34 & 32 & 33\\
\hline
\end{array}
\]
\begin{enumerate}
\item Calcular la ecuación de la recta de regresión del Ángulo de Clarke en función de la edad.
\item ¿Qué tanto por ciento de la variabilidad de la nube de puntos explicamos con el modelo lineal? ¿Se puede considerar un modelo
bueno?
\item Calcular el coeficiente de correlación de Spearman e interpretarlo. ¿Está en consonancia con el coeficiente de correlación lineal?
\end{enumerate}
}
%SOLUTION
{
}
%RESOLUTION
{
}


\newproblem{reg-44}{far}{*}
%STATEMENT
{A study tries to determine the relationship between two substances $X$ and $Y$ in blood.
The concentrations of these substances have been measured in seven individuals (in $\mu$g/dl) and the results are shown in the table below.
\[
\begin{array}{rrrrrrrr}
\toprule
X & 2.1 & 4.9 & 9.8 & 11.7 & 5.9 & 8.4 & 9.2 \\
Y & 1.3 & 1.5 & 1.7 & 1.8 & 1.5 & 1.7 & 1.7 \\
\bottomrule
\end{array}
\]

\begin{enumerate}
\item Are $Y$ and $X$ linearly related?
\item Are $Y$ and $X$ potentially related?
\item Use the best of the previous regression models to predict the concentration in blood of $Y$ for $x=8$ $\mu$gr/dl.
Is this prediction reliable.
Justify your answer.
\end{enumerate}

Use the following sums:
$\sum x_i=52$ $\mu$g/dl, $\sum \log(x_i)=13.1955$ $\log(\mu\mbox{g/dl})$, $\sum y_j=11.2$ $\mu\mbox{g/dl}$, $\sum \log(y_j)=3.253$ $\log(\mu\mbox{g/dl})$, $\sum x_i^2=451.36$ $(\mu\mbox{g/dl})^2$, $\sum \log(x_i)^2=26.9397$ $\log(\mu\mbox{g/dl})^2$, $\sum y_j^2=18.1$ $(\mu\mbox{g/dl})^2$, $\sum \log(y_j)^2=1.5878$ $\log(\mu\mbox{g/dl})^2$, $\sum x_iy_j=86.57$ $(\mu\mbox{g/dl})^2$, $\sum x_i\log(y_j)=26.3463$ $\mu\mbox{g/dl}\cdot\log(\mu\mbox{g/dl})$, $\sum \log(x_i)y_j=21.7087$ $\log(\mu\mbox{g/dl})\cdot\mu\mbox{g/dl}$, $\sum \log(x_i)\log(y_j)=6.5224$ $\log(\mu\mbox{g/dl})^2$.
}
%SOLUTION
{\begin{enumerate}
\item Linear relation: $r^2=0.9696$, so there is a strong linear relation.
\item Potential relation: $r^2=0.9688$, so there is a strong potential relation, however the linear relation is a little bit stronger.
\item $y(8)=1.6296$ $\mu$gr/dl.
\end{enumerate}
}
%RESOLUTION
{Para el modelo lineal se tiene
\begin{enumerate}
\item Para ver si existe relación lineal entre $Y$ y $X$ se calcula el coeficiente de determinación lineal:
\begin{align*}
\bar x &= \frac{\sum x_i}{n} = \frac{2.1+\cdots+9.2}{7} = \frac{52}{7} = 7.4286 \text{ $\mu$gr/dl},\\
s_x^2 &= \frac{\sum x_i^2}{n}-\bar x^2 = \frac{2.1^2+\cdots+9.2^2}{7} -7.4286^2= \frac{451.36}{7}-55.1841 = 9.2963 \text{ $(\mu$gr/dl)$^2$},\\
\bar y &= \frac{\sum y_j}{n} = \frac{1.3+\cdots+1.7}{7} = \frac{11.2}{7} = 1.6 \text{ $\mu$gr/dl},\\
s_y^2 &= \frac{\sum y_j^2}{n}-\bar y^2 = \frac{1.3^2+\cdots+1.7^2}{7} -1.6^2= \frac{18.1}{7}-2.56 = 0.0257 \text{ $(\mu$gr/dl)$^2$},\\
s_{xy} &= \frac{\sum x_iy_j}{n}-\bar x\bar y = \frac{2.1\cdot1.3+\cdots+9.2\cdot1.7}{7}-7.4286\cdot1.6 = \frac{86.57}{7}-11.8858 = 0.4814 \text{ $(\mu$gr/dl)$^2$},\\
r^2 &= \frac{s_{xy}^2}{s_x^2 s_y^2} = \frac{0.4814^2}{9.2963\cdot 0.0257} = 0.9696.
\end{align*}
Como el coeficiente de determinación lineal está muy próximo a 1, podemos concluir que existe una relación lineal muy fuerte entre $X$ e $Y$.

\item Del mismo modo, para ver si existe relación potencial entre $Y$ y $X$ se calcula el coeficiente de determinación potencial.
Teniendo en cuenta que la ecuación del modelo potencial $y=ax^b$ se puede convertir en lineal aplicando el logarítmo tanto a $X$ como a $Y$,
$\ln y = \ln a + b\ln x$, el coeficiente de determinación potencial entre $X$ y $Y$ es el mismo que el coeficiente de determinación lineal entre $\ln(X)$ y $\ln(Y)$.
Así pues, para calcularlo primero construimos las variables $U=\ln(X)$ y $V=\ln(Y)$:
\[
\begin{array}{rrrrrrrr}
   \hline
U = \ln X & 0.7419 & 1.5892 & 2.2824 & 2.4596 & 1.7750 & 2.1282 & 2.2192 \\
  V = \ln Y & 0.2624 & 0.4055 & 0.5306 & 0.5878 & 0.4055 & 0.5306 & 0.5306 \\
   \hline
\end{array}\]

Y el coeficiente de determinación potencial vale:
\begin{align*}
\bar u &= \frac{\sum u_i}{n} = \frac{0.7419+\cdots+2.2192}{7} = \frac{13.1955}{7} = 1.8851 \text{ $\ln(\mu$gr/dl)},\\
s_u^2 &= \frac{\sum u_i^2}{n}-\bar u^2 = \frac{0.7419^2+\cdots+2.2192^2}{7} -1.8851^2= \frac{26.9397}{7}-3.5536 = 0.295 \text{ $\ln^2(\mu$gr/dl)},\\
\bar v &= \frac{\sum v_j}{n} = \frac{0.2624+\cdots+0.5306}{7} = \frac{3.253}{7} = 0.4647 \text{ $\ln(\mu$gr/dl)},\\
s_v^2 &= \frac{\sum v_j^2}{n}-\bar v^2 = \frac{0.2624^2+\cdots+0.5306^2}{7} -0.4647^2= \frac{1.5878}{7}-0.2159 = 0.0109 \text{ $\ln^2(\mu$gr/dl)},\\
s_{uv} &= \frac{\sum u_iv_j}{n}-\bar u\bar v = \frac{0.7419\cdot0.2624+\cdots+2.2192\cdot0.5306}{7}-1.8851\cdot0.4647 =\\
&= \frac{6.5224}{7}-0.876 = 0.0557 \text{ $\ln^2(\mu$gr/dl)},\\
r^2 &= \frac{s_{uv}^2}{s_u^2 s_v^2} = \frac{0.0557^2}{0.295\cdot 0.0109} = 0.9688.
\end{align*}

Así pues, el modelo potencial también es muy buen modelo para explicar la relación entre $Y$ y $X$ aunque es un poco mejor el lineal.

\item Como el modelo lineal es un poco mejor que el potencial, hay que hacer la predicción con el modelo lineal. Para ello se calcula la recta
de regresión de $Y$ sobre $X$, que tiene ecuación
\[
y = \bar y +\frac{s_{xy}}{s_x^2}(x-\bar x) = 1.6 + \frac{0.4814}{9.2963}(x-7.4286) = 0.0518x+1.2153.
\]

Finalmente, la concentración de $Y$ para $x=8$ $\mu$gr/dl será
\[
y = 0.0518\cdot 8+1.2153 = 1.6296 \text{ $\mu$gr/dl}.
\]
\end{enumerate}
}


\newproblem*{reg-45}{gen}{}
%STATEMENT
{Give some examples of:
\begin{enumerate}
\item Non related variables.
\item Variables that are increasingly related.
\item Variables that are decreasingly related.
\end{enumerate}
}


\newproblem{reg-46}{far}{*}
%STATEMENT
{The gene of a rat species has been modified to help the metabolization of cholesterol in blood.
To check the effectiveness of this genetic modification two samples of rats were drawn, ones with the gene modified and the others not, and they were fed with the same diet with different concentrations of palm oil during one month. 
The following sums summarize the results:

\textbf{Palm oil quantity in gr} (the same in both samples)\\
$\sum x_i=640.6467$, $\sum x_i^2=23508.6387$,\\
\textbf{Cholesterol level in blood in mg/dl of non genetically modified rats}\\
$\sum y_j=2945.8545$, $\sum y_j^2=439517.5975$, $\sum x_iy_j=98156.0658$.\\
\textbf{Cholesterol level in blood in mg/dl of genetically modified rats}\\
$\sum y_j=2126.5899$, $\sum y_j^2=226824.5373$, $\sum x_iy_j=69517.3648$.

\begin{enumerate}
\item In which sample the regression line of cholesterol on the palm oil quantity fits better?
\item According to the regression line, what level of cholesterol is expected for a genetically modified rat with a diet of 25 gr of palm oil?
And for a non genetically modified rat?
\item What amount of palm oil must be supplied to a non genetically modified rat to have a cholesterol level smaller than 150 mg/dl?
Is this prediction reliable?
\end{enumerate}
}
%SOLUTION
{\begin{enumerate}
\item Palm oil quantity: $\bar x=32.0323$ gr,  $s^2_x=149.3614$ gr$^2$.\\
Non genetically modified rats: $\bar y=147.2927$ mg/dl,  $s^2_y=280.7332$ (mg/dl)$^2$, $s_{xy}=189.6733$ gr$\cdot$mg/dl and $r^2=0.858$.\\
Genetically modified rats: $\bar y=106.3295$ mg/dl,  $s^2_y=35.265$ (mg/dl)$^2$, $s_{xy}=69.8861$ gr$\cdot$mg/dl and $r^2=0.9273$.\\
Thus, the regression line fits better in genetically modified rats since the coef. of determination is greater.
\item Regression line of $Y$ on $X$ in non genetically modified rats: $y=106.615+1.2699x$. Prediction: $y(25)=138.3624$\\
Regression line of $Y$ on $X$ in genetically modified rats: $y=91.3416+0.4679x$. Prediction: $y(25)=103.0391$.
\item Regression line of $X$ on $Y$ in non genetically modified rats: $x=-67.4838+0.6756y$. Prediction: $x(150)=33.8615$. The prediction is very reliable since the coef. of determination is close to 1.
\end{enumerate}
}
%RESOLUTION
{
}

\newproblem{reg-47}{med}{*}
%STATEMENT
{A study tries to determine the effect of smoking during the pregnancy in the weight of newborns. The table below shows the daily number of cigarretes smoked by mothers ($X$) and the weight of the newborn
(all of them are males) ($Y$).

\[
\begin{array}{lrrrrrrrrrrrr}
   \hline
\mbox{Daily num cigarettes} & 10.00 & 14.00 & 8.00 & 11.00 & 7.00 & 6.00 & 2.00 & 5.00 & 9.00 & 9.00 & 4.00 & 6.00 \\ 
  \mbox{Weight (kg)} & 2.55 & 2.44 & 2.68 & 2.65 & 2.71 & 2.85 & 3.45 & 2.93 & 2.67 & 2.59 & 3.02 & 2.72 \\ 
   \hline
\end{array}
\]

\begin{enumerate}
\item Give the equation of the regression line of the weight of newborns on the daily number of cigarettes and interpret the slope.
\item Which regression model is better to predict the weight of newborns, the logarithmic or the exponential?
\item Use the best of the two previous regression models to predict the weight of a newborn whose mother smokes 12 cigarettes a day. Is this prediction reliable?
\end{enumerate}

Use the following sums for the computations:\\
$\sum x_i=91$ cigarettes, $\sum \log(x_i)=23.0317$
$\log(\mbox{cigarettes})$, $\sum y_j=33.26$ kg,
$\sum \log(y_j)=12.1857$ $\log(\mbox{kg})$,\\
$\sum x_i^2=809$ cigarettes$^2$, $\sum \log(x_i)^2=47.196$
$\log(\mbox{cigarettes})^2$, $\sum y_j^2=92.9708$ kg$^2$,
$\sum \log(y_j)^2=12.4665$ $\log(\mbox{kg})^2$,\\
$\sum x_iy_j=243.61$ cigarettes$\cdot$kg,
$\sum x_i\log(y_j)=89.3984$ cigarettes$\cdot\log(\mbox{kg})$,
$\sum \log(x_i)y_j=62.3428$ $\log(\mbox{cigarettes})$kg,
$\sum \log(x_i)\log(y_j)=22.8753$
$\log(\mbox{cigarettes})\log(\mbox{kg})$.
}
%SOLUTION
{ 
\begin{enumerate}
   \item $\bar x=7.5833$ cigarettes, $s_x^2=9.9097$ cigarettes$^2$.\\ 
   $\bar y=2.7717$ kg, $s_y^2=0.0654$ kg$^2$.\\
     $s_{xy}=-0.7176.$ cigarettes$\cdot$kg\\
     Regression line: $y=-0.0724x + 3.3208$.
   \item $\overline{\log(x)}=1.9193$ log(cigarettes), $s_{\log(x)}^2=0.2492$ log(cigarettes)$^2$.\\
     $\overline{\log(y)}=1.0155$ log(kg), $s_{\log(y)}^2=0.0077$ log(kg)$^2$.\\
     $s_{x\log(y)}=-0.2508$ cigarettes$\cdot$log(kg), $s_{\log(x)y}=-0.1245$ log(cigarettes)$\cdot$kg\\
     Logarithmic coef. determination: $r^2=0.9499$\\
     Exponential coef. determination: $r^2=0.8268$\\
     Therefore, the logarithmic models fits better the data and is better to predict the weight.
   \item Logarithmic regression model: $y=3.7301+-0.4994\log(x)$.\\
     Prediction: $y(12)=2.4892$ kg. The coefficient of determination is high but the sample size small, so the prediction is not enterely reliable.
\end{enumerate}
}
%RESOLUTION
{
}