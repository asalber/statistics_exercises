% Author Alfredo Sánchez Alberca (asalber@ceu.es)

\section{Regression and correlation}
\begin{enumerate}[leftmargin=*,resume]
\item Give some examples of:
\begin{enumerate}
\item Non related variables.
\item Variables that are increasingly related.
\item Variables that are decreasingly related.
\end{enumerate}

\item In an study about the effect of different doses of a medicament, 2 patients got 2 mg and took 5 days to cure, 4
patients got 2 mg and took 6 days to cure, 2 patients got 3 mg ant took 3 days to cure, 4 patients got 3 mg and took 5
days to cure, 1 patient got 3 mg and took 6 days to cure, 5 patients got 4 mg and took 3 days to cure and 2 patients got
4 mg and took 5 days to cure. 
\begin{enumerate}
\item Construct the joint frequency table.
\item Get the marginal frequency distributions and compute the main statistics for every variable. 
\item Compute the covariance and interpret it. 
\end{enumerate}

\item The table below shows the two-dimensional frequency distribution of a sample of 80 persons in a study about the
relation between the blood cholesterol ($X$) in mg/dl and the high blood pressure ($Y$).
\[
\begin{array}{|c||c|c|c||c|}
\hline
X\setminus Y & [110,130) & [130,150) & [150,170) & n_x \\
\hline\hline
[170,190)   &           &     4     &           & 12\\
\hline
[190,210)   &    10     &    12     &     4     &   \\
\hline
[210,230)   &     7     &           &     8     &   \\
\hline
[230,250)   &     1     &           &           & 18\\
\hline\hline
n_y          &           &    30     &    24    &    \\
\hline
\end{array}
\]

\begin{enumerate}
\item Complete the table.
\item Construct the linear regression model of cholesterol on pressure. 
\item Use the linear model to calculate the expected cholesterol for a person with pressure 160 mmHg. 
\item According to the linear model, what is the expected pressure for a person with cholesterol 270 mg/dl?
\end{enumerate}

Use the following sums:
$\sum x_i=16960$ mg/dl, $\sum y_j=11160$ mmHg, $\sum x_i^2=3627200$ (mg/dl)$^2$, $\sum y_j^2=1576800$ mmHg$^2$ y
$\sum x_iy_j=2378800$ mg/dl$\cdot$mmHg.

\item A research study has been conducted to determine the loss of activity of a drug.
The table below shows the results of the experiment.

\begin{center}
\begin{tabular}{|l|r|r|r|r|r|}
\hline
Time (in years) & 1 & 2 & 3 & 4 & 5 \\ \hline
Activity (\%) & 96 & 84 & 70 & 58 & 52 \\ \hline
\end{tabular}
\end{center}

\begin{enumerate}
\item  Construct the linear regression model of activity on time.
\item  According to the linear model, when will the activity be 80\%? When will the drug have lost all activity?
\end{enumerate}

\item A basketball team is testing a new stretching program to reduce the injuries during the league. 
The data below show the daily number of minutes doing stretching exercises and the number of injuries along the league. 
\begin{center}
\begin{tabular}{lrrrrrrrr}
\toprule
Stretching minutes & 0 & 30 & 10 & 15 & 5 & 25 & 35 & 40\\
Injuries & 4 & 1 & 2 & 2 & 3 & 1 & 0 & 1\\
\bottomrule
\end{tabular}
\end{center}
\begin{enumerate}
\item Construct the regression line of the number of injuries on the time of stretching. 
\item What is the reduction of injuries for every minute of stretching? 
\item How many minutes of stretching are require for having no injuries? Is reliable this prediction?
\end{enumerate}

Use the following sums ($X$=Number of minutes stretching, and $Y$=Number of injuries):
$\sum x_i = 160$ min, $\sum y_j=14$ injuries, $\sum x_i^2= 4700$ min$^2$, $\sum y_j^2=36$ injuries$^2$ and $\sum
x_iy_j=160$  min$\cdot$injuries.

\item For two variables $X$ and $Y$ we have
\begin{itemize}
\item[--] The regression line of $Y$ on $X$ is $y-x-2=0$.
\item[--] The regression line of $X$ on $Y$ is $y-4x+22=0$.
\end{itemize}
Calculate:
\begin{enumerate}
\item The means $\bar x$ and $\bar y$.
\item The correlation coefficient.
\end{enumerate}

\item The means of two variables $X$ and $Y$ are $\bar x=2$ and $\bar y=1$, and the correlation coefficient is 0.
\begin{enumerate}
\item  Predict the value of $Y$ for $x=10$.
\item  Predict the value of $X$ for $y=5$.
\item  Plot both regression lines.
\end{enumerate}

\item A study to determine the relation between the age and the physical strength gave the scatter plot
below.
\begin{center}
\resizebox{0.7\textwidth}{!}{% Created by tikzDevice version 0.9 on 2016-03-02 18:10:58
% !TEX encoding = UTF-8 Unicode
\begin{tikzpicture}[x=1pt,y=1pt]
\definecolor{fillColor}{RGB}{255,255,255}
\path[use as bounding box,fill=fillColor,fill opacity=0.00] (0,0) rectangle (505.89,361.35);
\begin{scope}
\path[clip] ( 36.00, 36.00) rectangle (493.89,337.35);
\definecolor{fillColor}{RGB}{5,161,230}

\path[fill=fillColor] ( 52.96, 57.31) circle (  2.25);

\path[fill=fillColor] ( 77.90,123.26) circle (  2.25);

\path[fill=fillColor] (115.31,179.07) circle (  2.25);

\path[fill=fillColor] (152.72,229.80) circle (  2.25);

\path[fill=fillColor] (190.13,270.38) circle (  2.25);

\path[fill=fillColor] (215.07,300.82) circle (  2.25);

\path[fill=fillColor] (227.54,305.90) circle (  2.25);

\path[fill=fillColor] (252.48,300.82) circle (  2.25);

\path[fill=fillColor] (277.41,295.75) circle (  2.25);

\path[fill=fillColor] (302.35,280.53) circle (  2.25);

\path[fill=fillColor] (314.82,270.38) circle (  2.25);

\path[fill=fillColor] (352.23,260.24) circle (  2.25);

\path[fill=fillColor] (377.17,250.09) circle (  2.25);

\path[fill=fillColor] (402.11,250.09) circle (  2.25);

\path[fill=fillColor] (439.52,239.94) circle (  2.25);

\path[fill=fillColor] (476.93,229.80) circle (  2.25);
\end{scope}
\begin{scope}
\path[clip] (  0.00,  0.00) rectangle (505.89,361.35);
\definecolor{drawColor}{RGB}{0,0,0}

\node[text=drawColor,anchor=base,inner sep=0pt, outer sep=0pt, scale=  1.20] at (264.94,345.21) {\bfseries Scatter plot of Strengh on Age};

\node[text=drawColor,anchor=base,inner sep=0pt, outer sep=0pt, scale=  1.00] at (264.94,  4.80) {Age};

\node[text=drawColor,rotate= 90.00,anchor=base,inner sep=0pt, outer sep=0pt, scale=  1.00] at ( 12.00,186.67) {Weight lifted (kg)};
\end{scope}
\begin{scope}
\path[clip] (  0.00,  0.00) rectangle (505.89,361.35);
\definecolor{drawColor}{RGB}{0,0,0}

\path[draw=drawColor,line width= 0.4pt,line join=round,line cap=round] ( 52.96, 36.00) -- (489.40, 36.00);

\path[draw=drawColor,line width= 0.4pt,line join=round,line cap=round] ( 52.96, 36.00) -- ( 52.96, 32.99);

\path[draw=drawColor,line width= 0.4pt,line join=round,line cap=round] (115.31, 36.00) -- (115.31, 32.99);

\path[draw=drawColor,line width= 0.4pt,line join=round,line cap=round] (177.66, 36.00) -- (177.66, 32.99);

\path[draw=drawColor,line width= 0.4pt,line join=round,line cap=round] (240.01, 36.00) -- (240.01, 32.99);

\path[draw=drawColor,line width= 0.4pt,line join=round,line cap=round] (302.35, 36.00) -- (302.35, 32.99);

\path[draw=drawColor,line width= 0.4pt,line join=round,line cap=round] (364.70, 36.00) -- (364.70, 32.99);

\path[draw=drawColor,line width= 0.4pt,line join=round,line cap=round] (427.05, 36.00) -- (427.05, 32.99);

\path[draw=drawColor,line width= 0.4pt,line join=round,line cap=round] (489.40, 36.00) -- (489.40, 32.99);

\node[text=drawColor,anchor=base,inner sep=0pt, outer sep=0pt, scale=  0.80] at ( 52.96, 21.60) {10};

\node[text=drawColor,anchor=base,inner sep=0pt, outer sep=0pt, scale=  0.80] at (115.31, 21.60) {15};

\node[text=drawColor,anchor=base,inner sep=0pt, outer sep=0pt, scale=  0.80] at (177.66, 21.60) {20};

\node[text=drawColor,anchor=base,inner sep=0pt, outer sep=0pt, scale=  0.80] at (240.01, 21.60) {25};

\node[text=drawColor,anchor=base,inner sep=0pt, outer sep=0pt, scale=  0.80] at (302.35, 21.60) {30};

\node[text=drawColor,anchor=base,inner sep=0pt, outer sep=0pt, scale=  0.80] at (364.70, 21.60) {35};

\node[text=drawColor,anchor=base,inner sep=0pt, outer sep=0pt, scale=  0.80] at (427.05, 21.60) {40};

\node[text=drawColor,anchor=base,inner sep=0pt, outer sep=0pt, scale=  0.80] at (489.40, 21.60) {45};

\path[draw=drawColor,line width= 0.4pt,line join=round,line cap=round] ( 36.00, 47.16) -- ( 36.00,300.82);

\path[draw=drawColor,line width= 0.4pt,line join=round,line cap=round] ( 36.00, 47.16) -- ( 32.99, 47.16);

\path[draw=drawColor,line width= 0.4pt,line join=round,line cap=round] ( 36.00, 97.89) -- ( 32.99, 97.89);

\path[draw=drawColor,line width= 0.4pt,line join=round,line cap=round] ( 36.00,148.63) -- ( 32.99,148.63);

\path[draw=drawColor,line width= 0.4pt,line join=round,line cap=round] ( 36.00,199.36) -- ( 32.99,199.36);

\path[draw=drawColor,line width= 0.4pt,line join=round,line cap=round] ( 36.00,250.09) -- ( 32.99,250.09);

\path[draw=drawColor,line width= 0.4pt,line join=round,line cap=round] ( 36.00,300.82) -- ( 32.99,300.82);

\node[text=drawColor,anchor=base east,inner sep=0pt, outer sep=0pt, scale=  0.80] at ( 31.20, 44.41) {10};

\node[text=drawColor,anchor=base east,inner sep=0pt, outer sep=0pt, scale=  0.80] at ( 31.20, 95.14) {20};

\node[text=drawColor,anchor=base east,inner sep=0pt, outer sep=0pt, scale=  0.80] at ( 31.20,145.87) {30};

\node[text=drawColor,anchor=base east,inner sep=0pt, outer sep=0pt, scale=  0.80] at ( 31.20,196.60) {40};

\node[text=drawColor,anchor=base east,inner sep=0pt, outer sep=0pt, scale=  0.80] at ( 31.20,247.34) {50};

\node[text=drawColor,anchor=base east,inner sep=0pt, outer sep=0pt, scale=  0.80] at ( 31.20,298.07) {60};

\path[draw=drawColor,line width= 0.4pt,line join=round,line cap=round] ( 36.00, 36.00) --
	(493.89, 36.00) --
	(493.89,337.35) --
	( 36.00,337.35) --
	( 36.00, 36.00);
\end{scope}
\end{tikzpicture}
}
\end{center}

\begin{enumerate}
\item Calculate the linear coefficient of determination for the whole sample.
\item Calculate the linear coefficient of determination for the sample of people younger than 25 years old. 
\item Calculate the linear coefficient of determination for the sample of people older than 25 years old.
\item Which model explains better the relation between the age and the strength? 
\end{enumerate}

Use the following sums ($X$=Age and $Y=$Weight lifted).
\begin{itemize}[label=--]
\item Whole sample: $\sum x_i=431$ years, $\sum y_j=769$ Kg, $\sum x_i^2=13173$ years$^2$, $\sum y_j^2=39675$
Kg$^2$ and $\sum x_iy_j=21792$ years$\cdot$Kg.
\item Young people: $\sum x_i=123$ years, $\sum y_j=294$ Kg, $\sum x_i^2=2339$ years$^2$, $\sum y_j^2=14418$
Kg$^2$ and $\sum x_iy_j=5766$ years$\cdot$Kg.
\item Old people: $\sum x_i=308$ years, $\sum y_j=475$ Kg, $\sum x_i^2=10834$ years$^2$, $\sum y_j^2=25257$
Kg$^2$ and $\sum x_iy_j=16026$ years$\cdot$Kg.
\end{itemize}

\item A dietary center is testing a new diet in sample of 12 persons. 
The data below are the number of days of diet and the weight loss (in Kg) until them for every person.  
\begin{center}
(33 , 3.9), (51 , 5.9), (30 , 3.2), (55 , 6.0), (38 , 4.9), (62 , 6.2),\\
(35 , 4.5), (60 , 6.1), (44 , 5.6), (69 , 6.2), (47 , 5.8), (40 , 5.3)
\end{center}
\begin{enumerate}
\item Draw the scatter plot. According to the point cloud, what type of regression model explains better the relation
between the weight loss and the days of diet?
\item Construct the linear regression model and the logarithmic regression model of the weight loss on the number of
days of diet.
\item Use the best model to predict the weight that will lose a person after 100 days of diet. 
Is this prediction reliable?
\end{enumerate}
Use the following sums ($X$=days of diet and $Y$=weight loss): $\sum x_i=564$ days, $\sum \log(x_i)=45.8086$
$\log(\mbox{days})$, $\sum y_j=63.6$ Kg, $\sum x_i^2=28234$ days$^2$, $\sum \log(x_i)^2=175.6603$ $\log(\mbox{days})^2$, 
$\sum y_j^2=347.7$ Kg$^2$, $\sum x_iy_j=3108.5$ days$\cdot$Kg, $\sum \log(x_i)y_j=245.4738$ $\log(\mbox{days})\cdot$Kg.

\item The concentration of a drug in blood, in mg/dl, depends on time, in hours, according to the data below.
\begin{center}
\begin{tabular}{lrrrrrrr}
\toprule
Drug concentration & 2 & 3 & 4 & 5 & 6 & 7 & 8\\
Hours & 25 & 36 & 48 & 64 & 86 & 114 & 168\\
\bottomrule
\end{tabular}
\end{center}
\begin{enumerate}
\item Construct the linear regression model of drug concentration on time. 
\item Construct the exponential regression model of drug concentration on time. 
\item Use the best regression model to predict the drug concentration after $4.8$ hours? Is this prediction reliable?
Justify the answer.
\end{enumerate}

Use the following sums ($C$=Drug concentration and $T$=time): $\sum c_i=35$ mg/dl, $\sum \log(c_i)=10.6046$
$\log(\mbox{mg/dl})$, $\sum t_j=541$ hours, $\sum \log(t_j)= 29.147$ $\log(\mbox{hours})$, $\sum c_i^2=203$ (mg/dl)$^2$,
$\sum \log(c_i)^2=17.5206$ $\log(\mbox{mg/dl})^2$, $\sum t_j^2=56937$ hours$^2$, $\sum \log(t_j)^2=124.0131$
$\log(\mbox{hours})^2$, $\sum c_it_j=3328$ mg/dl$\cdot$hours, $\sum c_i\log(t_j)=154.3387$
mg/dl$\cdot\log(\mbox{hours})$, $\sum \log(c_i)t_j=951.6961$ $\log(\mbox{mg/dl})\cdot$hours, $\sum
\log(c_i)\log(t_j)=46.08046$ $\log(\mbox{mg/dl})\cdot\log(\mbox{hours})$.

\item A researcher is studying the relation between the obesity and the response to pain. 
Te obesity is measured as the percentage over the ideal weight, and the response to pain as the nociceptive flexion pain
threshold.
The results of the study appears in the table below.
\[
\begin{array}{lrrrrrrrrrr}
\toprule
\mbox{Obesity} & 89 & 90 & 75 & 30 & 51 & 75 & 62 & 45 & 90 & 20\\
\mbox{Pain threshold} & 10 & 12 & 4 & 4.5 & 5.5 & 7 & 9 & 8 & 15 & 3\\
\bottomrule
\end{array}
\]
\begin{enumerate}
\item According to the scatter plot, what model explains better the relation of the response to pain on the
obesity?
\item According to the best regression model, what is the response to pain expected for a person with an obesity of
50\%? Is this prection reliable?
\item According to the best regression model, what is the expected obesity for a person with a pain threshold of
10? Is this prediction reliable?
\end{enumerate}
Use the following sums ($X$=Obesity and $Y$=Pain threshold): $\sum x_i=627$, $\sum \log(x_i)=40.3858$, $\sum y_j=78$,
$\sum \log(y_j)=19.4119$, $\sum x_i^2=45141$, $\sum \log(x_i)^2=165.4516$, $\sum y_j^2=738.5$, $\sum
\log(y_j)^2=40.0458$, $\sum x_iy_j=5538.5$, $\sum x_i\log(y_j)=1306.051$, $\sum \log(x_i)y_j=327.3887$, $\sum
\log(x_i)\log(y_j)=80.1831$.

\end{enumerate}

